% Dieses Werk ist unter einer Creative Commons Lizenz vom Typ Namensnennung-
% Weitergabe unter gleichen Bedingungen 3.0 Unported zugänglich. Um eine Kopie
% dieser Lizenz einzusehen, konsultieren Sie
% http://creativecommons.org/licenses/by-sa/3.0/ oder wenden Sie sich brieflich
% an Creative Commons, 444 Castro Street, Suite 900, Mountain View, California,
% 94041, USA.

\documentclass[8pt,a4paper,german,twoside]{book}

\synctex=1

\usepackage{extsizes}
\usepackage{makeidx}
\usepackage[german]{babel}
\makeindex
\usepackage[utf8]{inputenc}
\usepackage[T1]{fontenc}
\usepackage{textcomp}
\usepackage{gensymb}
\usepackage{graphicx}
\usepackage{amssymb}
\usepackage{amsmath}
\usepackage{stmaryrd}
%\usepackage{gastex}
%\usepackage{pstricks, pst-node, pst-plot}
\usepackage{epsfig}
\usepackage[left=3cm,right=3cm,top=3cm,bottom=3cm]{geometry}
\usepackage[colorlinks=true,urlcolor=blue,linkcolor=black,backref,pagebackref]{hyperref}

\hypersetup{
	pdftitle = {KogSys Zusammenfassung}
}

\begin{document}

%*************** new commands ***********************************************************

\newcommand{\myvector}[1]{\stackrel{\to}{#1}}
\newcommand{\myprob}[2]{P(#1 \, | \, #2)}
\newcommand{\myvectwo}[2]{\left( \begin{array}{c} #1 \\ #2 \end{array} \right)}
\newcommand{\myvecqfour}[4]{\left( \begin{array}{c} #1 \\ #2 \\ #3 \\ #4 \end{array} \right)}
\newcommand{\myvecfour}[4]{\left( \begin{array}{cc} #1 & #2 \\ #3 & #4 \end{array} \right)}
\newcommand{\myvecfourright}[4]{\left( \begin{array}{rr} #1 & #2 \\ #3 & #4 \end{array} \right)}
\newcommand{\myvecthree}[3]{\left( \begin{array}{c} #1 \\ #2 \\ #3 \end{array} \right)}
\newcommand{\myvecnine}[9]{\left( \begin{array}{ccc} #1 & #2 & #3 \\ #4 & #5 & #6 \\ #7 & #8 & #9 \end{array} \right)}
\newcommand{\myvecnineright}[9]{\left( \begin{array}{rrr} #1 & #2 & #3 \\ #4 & #5 & #6 \\ #7 & #8 & #9 \end{array} \right)}
\newcommand{\quaternion}{\textbf{q}}
%\newcommand{\mypic}[2]{\begin{center} \includegraphics[width=#1 cm]{pics/#2.eps}\end{center}}
%\newcommand{\mypictwo}[4]{\begin{center} \includegraphics[width=#1 cm]{pics/#2.eps} \\[0,2cm] \includegraphics[width=#3 cm]{pics/#4.eps} \end{center}}
\newcommand{\mypic}[2]{\begin{center} \includegraphics[width=#1 cm]{bilder/#2.png}\end{center}}
\newcommand{\mypictwo}[4]{\begin{center} \includegraphics[width=#1 cm]{bilder/#2.png} \\[0,2cm] \includegraphics[width=#3 cm]{bilder/#4.png} \end{center}}
%\newcommand{\href}[2]{#2}

%*************** new commands (end) ******************************************************

\setlength{\parindent}{0 pt}
\pagestyle{headings}

%=========================================================================================
%============= Beginn des Dokuments ======================================================
%=========================================================================================

\tableofcontents

\pagestyle{headings}

% !TeX root = summary.tex

\section{Signalverarbeitung}

\subsection{Grundlegendes}

\begin{description}
\item[Erfassen/Messen von Signalen:] \quad
\begin{itemize}
\item Signal als Funktion: Akustik $f(t)$; Bilder $f(x,y)$; Energie als Funktion der Zeit oder des Raumes; Energie: Lautstärke, Helligkeit, Grauwertintensität; Farbe, Stereo, Bildsequenzen
\item Eigenschaften: Hinlänglich glatt; $0 \leq f(x,y) < \infty$ (wertbeschränkt)
\end{itemize}
\item[Abtastung und Sampling\index{Sampling}:] \quad
\begin{itemize}
\item Man messe das Signal $f(x)$ an verschiedenen Punkten $x$
\item Punkte $x$, in diskreten Abständen, an meist äquidistanten Stellen eines Abtastrasters
\item Akustik: $G = f(0), f(\Delta), f(2 \Delta), \dots, f((N-1)\Delta)$
\item Bild: $$G = \left[ \begin{array}{ccc} f(0,0) & \cdots & f(0,N-1) \\ \vdots & \ddots & \vdots \\ f(N-1,0) & \cdots & f(N-1,N-1) \end{array} \right]$$
\item Meist: $N = 2^m$
\item Bild: Rechtwinklig, schiefwinklig, sechseckige Raster
\item Wie oft? Wie großes Raster? $\to$ Sampling Theorem
\end{itemize}
\item[Quantisierung\index{Quantisierung}:] \quad
\begin{itemize}
\item Im Computer müssen Messwerte quantisiert werden. Die Anzahl der Quantisierungsstufen bestimmt Auflösung.
\item Feinere Auflösung: Bessere Qualität, mehr Speicher
\item Dynamische Abtastung: Starke Übergänge: Fein rastern - grob quantisieren; schwache Übergänge: Grob rastern, fein quantisieren
\end{itemize}
\item[Digitalisierung von Signalen:] \quad
\begin{itemize}
\item CD, Video (DV, DVD), Digitaler Rundfunk (DAB), ISDN
\item Vorteile: Qualität (Bits sind Bits, verlustfreie Übertragung); Kompression; mehrfacher Nutzen von Kommunikationskanälen (Time Division Multiple Access)
\end{itemize}
\item[Diracfunktion\index{Diracfunktion}:]\quad
\begin{itemize}
\item Definition: $$\int\limits_{- \infty}^{+ \infty} f(x) \delta(x - x_0) dx = f(x_0) \quad \textrm{und} \quad \int\limits_{- \infty}^{+ \infty} \delta(x - x_0) dx = \int\limits_{x_0^-}^{x_0^+} \delta(x - x_0) dx = 1$$
\item Hat Fläche 1, an einer beliebig kleinen Umgebung
\item $A \delta(x - x_0)$ Impuls mit Stärke $A$ an der Stelle $x = x_0$
\end{itemize}
\end{description}

\subsection{Faltung\index{Faltung}}

In der Mathematik und besonders in der Funktionalanalysis beschreibt die Faltung einen mathematischen Operator, welcher für zwei Funktionen $f$ und $g$ eine dritte Funktion liefert. Diese gibt eine Art "{}Überlappung"{} zwischen $f$ und einer gespiegelten und verschobenen Version von $g$ an.
Definition: $$\begin{array}{cl} (f * g)(t) = \int\limits_{- \infty}^{+ \infty} f(t - \tau) g(\tau) d \tau & \quad \textrm{(kontinuierlich)} \\ (f * g)[i] = \sum\limits_{j = - \infty}^{\infty} f[i - j] g[j] & \quad \textrm{(diskret)} \end{array}$$
Bedeutung: \\
Eine anschauliche Deutung der Faltung ist die Gewichtung einer Funktion mit einer anderen. Der Funktionswert der Gewichtsfunktion an einer Stelle $t$ gibt an, wie stark der um $t$ zurückliegende Wert der gewichteten Funktion in den Wert der Ergebnisfunktion eingeht. \\
Faltung der Zeitfunktion: $$F(f_1(t) * f_2(t)) = F_1(\omega) \cdot F_2(\omega)$$
Multiplikation der Zeitfunktionen: $$F(f_1(t) \cdot f_2(t)) = F_1(\omega) * F_2(\omega)$$
Faltung zweier Funktionen entspricht dem Filtern eines Signals. \\[0,1cm]
Beipiel mit den folgenden zwei Funktionen:
$$f(x) = \left\{ \begin{array}{ccc} A & \textrm{ für } & x \in [-L;L] \\ 0 & \textrm{sonst} & \end{array} \right. \quad \textrm{und} \quad g(x) = \left\{ \begin{array}{ccc} A & \textrm{ für } & x \in [0;2L] \\ 0 & \textrm{sonst} & \end{array} \right.$$

% Bild der beiden Funktionen
\mypic{10}{faltung1}

Die Faltung der beiden Funktionen sieht folgendermaßen aus:

% Bild der Faltung
\mypic{5}{faltung2}

\subsection{Fouriertransformation\index{Fouriertransformation}}

\subsubsection*{Idee der Fouriertransformation}
\begin{itemize}
\item Zerlegung eines Signals in eine Summe von komplexen Sinus- und Cosinusfunktionen.
\item Unterschiedliche Frequenzen.
\item Darstellung: Amplituden und Phasen von Frequenz
\end{itemize}

\subsubsection*{Fouriertransformation}
\begin{itemize}
\item Fouriertransfromation: $$F(\omega) = \int\limits_{- \infty}^{\infty} f(t) e^{-i \omega t} dt$$
\item Inverse Fouriertransformation: $$f(t) = \frac{1}{2 \pi} \int\limits_{- \infty}^{\infty} F(\omega) e^{+i \omega t} d \omega$$
\item Betrag des Fourierspektrums: $$|F(\omega)| = \sqrt{\Re(F(\omega))^2 + \Im(F(\omega))^2}$$
\item Phase des Spektrums: $$\phi(F(\omega)) = \textrm{arctan}\frac{\Im(F(\omega))}{\Re(F(\omega))}$$
\end{itemize}

\subsubsection*{Eigenschaften}
\begin{itemize}
\item Linearität: $$F(c_1f_1 + c_2f_2) = c_1F(f_1) + c_2F(f_2)$$
\item Differentiation: $$F(f^{(n)}) = (i \omega)^n F(f)$$
\item Verschiebung: $$\begin{array}{cl} F(f(t-T)) = e^{-i \omega T} F(f) & \quad \textrm{(Zeit)} \\ F(e^{-i \omega_0 t} f(t)) = F(\omega - \omega_0) & \quad \textrm{(Frequenz)} \end{array}$$
\end{itemize}

\subsubsection*{Zusammenhänge}
\begin{center}
\begin{tabular}{rcl}
Signal & $\Rightarrow$ & Transformierte \\ Transformierte & $\Leftarrow$ & Signal \\[0,1cm] diskret & $\Leftrightarrow$ & periodisch \\ reell & $\Leftrightarrow$ & gerade \\ imaginär & $\Leftrightarrow$ & ungerade
\end{tabular}
\end{center}

\subsubsection*{Typische Fouriertransformationen}
\begin{itemize}
\item Sinusfunktion: $$f(x) = a \sin(2 \pi \alpha x) \quad \Leftrightarrow \quad F(\omega) = \frac{a}{2}i \delta(\omega + \alpha) - \frac{a}{2}i \delta(\omega - \alpha)$$
\item Cosinusfunktion: $$f(x) = a \cos(2 \pi \alpha x) \quad \Leftrightarrow \quad F(\omega) = \frac{a}{2} \delta(\omega + \alpha) + \frac{a}{2} \delta(\omega - \alpha)$$
\end{itemize}
(Sinus- und cosinusförmige Signale werden im Frequenzbereich als zwei Impulse wiedergegeben.)
\begin{itemize}
\item Impulszug: $$f(x) = \sum\limits_{\nu = -\infty}^{\infty} \delta(x - \nu T) \quad \Leftrightarrow \quad F(\omega) = \frac{1}{T} \sum\limits_{\nu = -\infty}^{\infty} \delta(\omega - \frac{\nu}{T})$$
\end{itemize}

\subsubsection*{Anmerkungen}
\begin{itemize}
\item Periodisch unendlich ausgedehntes Signal wird im Fourierbereich als endliche Bandbreite dargestellt.
\item Endliches Signal kann zu unendlichem Spektrum führen. Approximation nötig!
\item Je stärker die Übergänge im Signal, desto höher die Frequenz der Komponenten, um so breiter das Spektrum.
\end{itemize}

\subsection{Fourierreihen\index{Fourierreihen}}

\subsubsection*{Fourierreihen und -transformationen}
Problem: Beide kontinuierlich, schwer digital darzustellen
\begin{itemize}
\item Diskrete Zeit (\textsl{discrete time}) Fouriertransformation: \\ Eingabe: diskret, aperiodisch \\ Ergebnis: periodisch, kontinuierlich
\item Diskrete Fouriertransformation (DFT, FFT): \\ Eingabe: periodisch, diskret \\ Ergebnis: diskret, periodisch \\ Spezialfall der Z-Transformation. Periodische Eingabe? Man schneidet ein Fenster aus und tut so, als ob es so periodisch unendlich fortgesetzt wird.
\end{itemize}

\subsubsection*{Fourierreihenzerlegung\index{Fourierreihenzerlegung}}
\begin{itemize}
\item Periodisches Signal
\item Signal: $$c_k = \frac{1}{2L} \int\limits_{-L}^{+L} f(x) e^{-i \pi \frac{k}{L} x} dx \quad \textrm{wobei} \quad f(x) = \sum\limits_{k=- \infty}^{+ \infty} c_k e^{i \pi \frac{k}{L} x}$$
\item Komplexe Schreibweise: $$e^{ix} = \cos(x) + i \sin(x)$$
\item Parameter: Amplitude, Phase, Frequenz (Periode)
\end{itemize}

\subsubsection*{Fourierreihe für Rechteckfunktion}
$$f(x) = \frac{1}{2} a_0 + \sum\limits_{n=1}^{\infty} \left( a_n \cos( \frac{2 \pi}{T} nx) + b_n \sin (\frac{2 \pi}{T} nx) \right)$$
Beispiel für Fourierreihenentwicklung: \\
Für $f(x) = \textrm{sign}(\sin(x))$ gilt:
\begin{eqnarray*}
a_n &=& \frac{1}{\pi} \int\limits_{- \pi}^{+ \pi} f(x) \cdot \cos(nx) dx = 0 \\
b_n &=& \frac{1}{\pi} \int\limits_{- \pi}^{+ \pi} f(x) \cdot \sin(nx) dx = \frac{2}{n \pi} (1 - (-1)^n)
\end{eqnarray*}
und somit $$f(x) = \frac{4}{\pi} \left( \sin(x) + \frac{\sin(3x)}{3} + \frac{\sin(5x)}{5} + \cdots \right)$$

\subsection{Aliasing\index{Aliasing}}
In der Signalverarbeitung treten Alias-Effekte beim Digitalisieren analoger Signale auf. \\
Damit das Ursprungssignal korrekt wiederhergestellt werden kann, dürfen im abgetasteten Signal nur Frequenzanteile vorkommen, die weniger als halb so groß wie die Abtastfrequenz sind. Wird dieses Abtasttheorem verletzt, werden Frequenzanteile, die größer sind als die halbe Abtastfrequenz als niedrigere Frequenzen interpretiert. Die hohen Frequenzen geben sich sozusagen als jemand anderes aus, daher die Bezeichnung Alias. Die halbe Abtastfrequenz wird als Nyquist-Frequenz bezeichnet. \\
Falls es nicht zu vermeiden ist, dass hohe Frequenzen im Eingangssignal vorhanden sind, wird das Eingangssignal zur Unterdrückung von Alias-Effekten durch einen Tiefpass gefiltert (Anti-Aliasing-Filter), wobei die aktive Filterwirkung dieses Abschneidens der hohen Frequenzen eindeutiger mit Höhensperre, Höhenfilter, High Cut und Treble Cut beschrieben wird.

\subsubsection*{Abtast/Sampling Theorem}
$$\Delta x \leq \frac{1}{2 \omega}$$
\begin{itemize}
\item Die Samplingfrequenz muss mindestens zweimal so gross sein als die höchste im Signal vorkommende Frequenz $\omega$ (\textsl{Cutoff-frequency}).
\item Um Aliasing zu vermeiden muss das Abtasttheorem eingehalten werden.
\item Ist das Abtasttheorem eingehalten, kann ein Signal durch inverse Fouriertransformation vollständig rekonstruiert werden.
\end{itemize}

\subsubsection*{Abtasten in der Praxis}
\begin{itemize}
\item Abtasten nur in endlichem Intervall möglich.
\item Sprektrum lokal (in kleinem Intervall interessant)
\begin{itemize}
\item Multiplikation des Signals mit Fensterfunktion
\item Faltung des Fenstersprektrums mit Signalspektrum
\item Ungenauigkeiten, genaue Rekonstruktion nicht mehr möglich
\end{itemize}
\item Ausnahme: Signal ist periodisch und bandbegrenzt
\end{itemize}

\subsubsection*{Behebung von Aliasing}
Möglichkeiten zur Behebung von Aliasing:
\begin{itemize}
\item Samplefrequenz erhöhen. Nachteil: erhöhtes Datenaufkommen
\item Bandbegrenzen des Originalsignals durch Tiefpassfilter (Anti-Aliasing Filter). Nachteil: Informationsverlust
\end{itemize}

\subsection{Korrelation\index{Korrelation}}
\begin{itemize}
\item Eindimensionale Kreuzkorrelation: $$R_{f,g}(m) = \sum\limits_{i} f(i) g(i - m)$$
\item Zweidimensionale Kreuzkorrelation: $$R_{f,g}(m) = \sum\limits_i \sum\limits_j f(i,j) g(i-m,j-n)$$
\item Autokorrelation = Kreuzkorrelation mit sich selbst:
\begin{eqnarray*}
A(m) &=& \sum\limits_i f(i) f(i-m) \\ A(m,n) &=& \sum\limits_i \sum\limits_j f(i,j) f(i-m,j-n)
\end{eqnarray*}
\end{itemize}

\subsection{Schablonenanpassung\index{Schablonenanpassung} (\textsl{Template Matching})}
\begin{itemize}
\item Eine einfache Form der Klassifikation.
\item Ziel: Ein Muster zu erkennen das einem abgespeicherten Beispiel ähnlich ist.
\item Maß der Übereinstimmung ist der Absolutbetrag zwischen Muster und Schablone (zentriert an $(m,n)$): $$M_{f,g}(m,n) = \sum\limits_i \sum\limits_j | f(i,j) g(i-m,j-n)|$$
\item Abstand $E(m,n)$ ist definiert als Quadrat der Kreuzkorrelation: $$E_{f,g} (m,n) = \left( \sum\limits_i \sum\limits_j f(i,j) g(i-m,j-n) \right)^2$$
\end{itemize}





\textit{}
% !TeX root = summary.tex

\newcommand{\myfbox}[1]{
\begin{tabular}{|l|}
\hline #1 \\ \hline
\end{tabular}
}

\chapter{Bildverarbeitung}

\section{Bildgenerierung}

\begin{itemize}
\item Heute werden oftmals Digitalkameras verwendet (meist CCD, aber auch CMOS)
\item Anschluss erfolgt über: Firewire (IEEE1394), USB, Camera Link, \dots
\item Kameras liefern direkt digitalisierte Bilddaten
\begin{itemize}
\item Format je nach Kamera und Modus unterschiedlich
\item Bei S/W-Kameras meist 8bit Graustufen
\item Bei Farbkameras entweder als bayer-Pattern oder meist bereits konvertiert als RGB24, YUV422, etc.
\end{itemize}
\item Kameras unterscheiden sich durch:
\begin{itemize}
\item Bildqualität (Qualität CCD-Chip, aber auch grundlegend: Auswahl Linse/Objektiv)
\item Graustufen- oder Farbkamera
\item Auflösung
\item Bei Farbkameras: Welche Farbkodierungen sind verfügbar? (8bit, 16bit, 24bit)
\item Wichtig je nach Anwendung: Welche maximale Framerate ist bei welchem Bildformat noch verfügbar? (z.B. 15/30/60/120/200 Hz)
\end{itemize}
\end{itemize}

\section{Bildrepräsentation}

\textbf{Monochrombild:} Diskrete Funktion
\begin{eqnarray*}
Img \, : \, [0 \dots n-1] \times [0 \dots m-1] &\to& [0 \dots q] \\ (u,v) &\mapsto& Img(u,v)
\end{eqnarray*}
Üblich: $q = 255$; $n = 640$, $m=480$ (VGA) oder $n = 768$, $m=576$ (PAL) \\[0,1cm]
\subsection{Farbbild}
\begin{itemize}
\item Viele verschiedene Farbmodelle für unterschiedliche Anwendungen
\item Klassifikation nach erreichbarem Farbraum
\item
\begin{itemize}
\item S/W, Grauwertstufen
\item RGB-Modell: speziell für Monitore (Phosphor-Kristalle), sehr üblich $$Img(u,v) \in \mathbb{R}^3 = (r,g,b)^T$$
\item HSI (Hue, Saturation, Intensity): speziell für Farbsegmentierung
\item CIE: physikalisch (Wellenlänge)
\item CMYK- Modell: Farbdrucker (subtraktive Farbmischung)
\item YIQ: Fernsehmodell
\end{itemize}
\end{itemize}

\subsubsection{RGB-Modell}\index{RGB-Modell}

\begin{eqnarray*}
Img \, : \, [0 \dots n-1] \times [0 \dots m-1] &\to& [0 \dots R] \times [0 \dots G] \times [0 \dots B] \\ (u,v) &\mapsto& Img(u,v) = (r,g,b)
\end{eqnarray*}
\begin{itemize}
\item additive Farbmischung
\item drei Farbwerte: Rot, Grün, Blau \\ oft: $256 \times 256 \times 256$ Nuancen \\ ($R=G=B=255$, 8Bit, "{}RGB24"{}) = 16,8 Mio. Farben
\item oft verwendet von Kamera-Treibern
\end{itemize}

\subsubsection{HSI-/HSV-Modell\index{HSI-/HSV-Modell}}

\begin{itemize}
\item Hue (Farbnuancen), Saturation (Sättigung), Intensity/Value (Helligkeit)
\item trennt Helligkeit vom Farbwert $\Rightarrow$ unempfindlich gegen Beleuchtungsänderungen
\item Umrechnung von RGB nach HSI (falls $R=G=B$, dann ist $H$ undefiniert; falls $R=G=B=0$, dann ist $S$ undefiniert)
\end{itemize}
\begin{eqnarray*}
c &=& \textrm{arcos } \frac{2R - G - B}{2 \sqrt{(R-G)^2 + (R-B)(G-B)}} \\
H &=& \left\{ \begin{array}{cl} c & \textrm{ falls } B < G \\ 360\degree - c & \textrm{ sonst} \end{array} \right. \\
S &=& 1 - \frac{3}{R+G+B} \min (R,G,B) \\
I &=& \frac{1}{3} (R + G + B)
\end{eqnarray*}

\subsubsection{Hinterlegung}

\begin{itemize}
\item Hinterlegung eines 8bit Graustufen-Bildes im Speicher
\begin{itemize}
\item Pixel werden zeilenweise, von oben links nach unten rechts, linear abgelegt (Achtung: z.B. bei Bitmaps von unten links nach oben rechts)
\item Graustufen-Kodierung: ein Byte pro pixel; 0 schwarz, 255 weiß, dazwischen Graustufen
\end{itemize}
\item Hinterlegung eines RGB24 Farbbildes im Speicher
\begin{itemize}
\item Pixel werden zeilenweise, wie beim Graustufen-Bild, abgelegt
\item Farbkodierung: drei Bytes pro Pixel; für jeden Kanal gilt: 0 minimale, 255 maximale Intensität, dazwischen Nuancen
\end{itemize}
\end{itemize}

\subsubsection{Grauwert-Transformation}

Transformation von RGB24 nach 8bit Graustufen:
\begin{itemize}
\item Eine Möglichkeit: $g = (R+G+B)/3$, aber: menschliches Auge ist am empfindlichsten gegenüber der Farbe Grün.
\item Üblicherweise wird deshalb verwendet: $$g = 0,299 \cdot R + 0,587 \cdot G + 0,114 \cdot B$$
\end{itemize}

\subsubsection*{Bayer-Pattern\index{Bayer-Pattern}}

Sehr hochwertige Kameras, wie z.B. zum Filmen verwendet, besitzen drei Chips pro Pixel. Die meisten Farbkameras haben einen Chip pro pixel, der gegenüber der Farbe Rot, Grün oder Blau empfindlich ist. Bei "{}Ein-Chip-Kameras"{} wird meist das Bayer-Pattern verwendet:
\begin{itemize}
\item Um nach RGB24 zu konvertieren, muss interpoliert werden.
\item Die Empfindlichkeit einer Ein-Chip-Kamera ist um den Faktor 3 niedriger als die einer reinen Graustufen-Kamera.
\end{itemize}

\subsubsection*{Lochkameramodell\index{Lochkameramodell}}

\mypic{8}{lochkamera2}

Projektion eines Szenenpunktes $P = (X,Y,Z)$ auf einen Bildpunkt $p = (u,v,w)$ mit Brennweite $f$:
$$\frac{-u}{f} = \frac{X}{Z} \quad , \quad \frac{-v}{f} = \frac{Y}{Z} \quad , \quad w = -f \qquad \Rightarrow \qquad X = - \frac{uZ}{f} \quad , \quad - \frac{vZ}{f}$$
$$p = \left( \begin{array}{c} u \\ v \\ w \end{array} \right) = \left( \begin{array}{c} u \\ v \\ -f \end{array} \right) = - \frac{f}{Z} \left( \begin{array}{c} X \\ Y \\ Z \end{array} \right) = - \frac{f}{Z} P$$
Bei der Projektion geht die $Z$-Komponente verloren! \\
Beispiel mit 2 Kameras:
\mypic{8}{lochkamera1}

\subsubsection*{Mattscheibenmodell\index{Mattscheibenmodell}}

Einziger Unterschied Mattscheibenmodell $\Leftrightarrow$ Lochkameramodell:
\begin{itemize}
\item Projektionszentrum $C$ liegt hinter der Bildebene
\item dadurch: keine Spiegelung (Minuszeichen entfallen)
\end{itemize}

\subsubsection*{Linsensysteme\index{Linsensysteme}}

Moderne Kameras benutzen Linsen. \\ Vorteil:
\begin{itemize}
\item mehr Lichteinfall
\end{itemize}
Nachteile:
\begin{itemize}
\item nur Teile der Szene können gleichzeitig fokussiert werden
\item Linsenkrümmung führt zu Bildverzerrungen
\end{itemize}
Bild eines Objekts in Tiefe $Z$ wird in Abstand von der Linse $Z'$ gebildet mit: $$\frac{1}{Z} + \frac{1}{Z'} = \frac{1}{f} \qquad f \, : \, \textrm{Brennweite}$$
\begin{itemize}
\item Bei Einstellung der Bildebene auf $Z_0'$ wird ein gewisser Objektbereich um $Z_0$ "{}ausreichend scharf"{} abgebildet.
\item Beim Menschen: Änderung der Linsenbrennweite.
\item Bei Kameras: Verschieben der Linse gegenüber Bildebene
\item Vereinfachung wegen $Z >> Z'$: \\ $\Rightarrow$ Perpektivgleichungen aus dem Lochkameramodell können weiterverwendet werden! $$\frac{1}{Z} + \frac{1}{Z'} \thickapprox \frac{1}{Z'} \quad \Rightarrow \quad \frac{1}{Z'} \thickapprox \frac{1}{f}$$
\end{itemize}

\section{Bildverarbeitung}

\subsubsection*{Konzepte}

\begin{itemize}
\item Homogene Punktoperationen
\item Histogrammauswertung
\item Filterung
\item Geometrische Operatoren
\end{itemize}
$\to$ Unterdrückung von Störungen, "{}Verschönern"{} von Bildern, Verformen von Bildern

\subsubsection*{Homogene Punktoperatoren\index{Homogene Punktoperatoren}}

Anwendung: $$Img'(u,v) = f(Img(u,v))$$
Unabhängig von der Position bzw. den Nachbarn des Pixels. Implementierung der Funktion $f$ oftmals als Look-Up-Table (Hardware).

\subsubsection*{Affine Punktoperatoren\index{Affine Punktoperatoren}}

Affine Punktoperatoren:
\begin{eqnarray*}
f \, : \, [0 \dots q] &\to& [0 \dots q] \\ x &\mapsto& ax+b
\end{eqnarray*}
Parameter $a$ und $b$ legen die Funktion fest. Anwendungen:
\begin{itemize}
\item Kontrasterhöhung: $b=0$, $a > 0$ \\ Kontrastverminderung: $b=0$, $a < 0$
\item Helligkeitserhöhung: $b > 0$, $a = 1$ \\ Helligkeitsverminderung: $b < 0$, $a = 1$
\item Invertierung: $b = q$, $a = -1$
\item Kombinationen
\end{itemize}

Spreizung\index{Spreizung}:
\begin{itemize}
\item Intensitäten im Bild werden linear auf gesamten Bereich ausgedehnt
\item Abbildungsvorschrift: \[Img'(x,y) = q*\frac{Img(x,y)-min}{max-min}\]
\item Als affine Punktoperation: \[Img'(x,y) =  \frac{q}{max-min}*Img(x,y)-\frac{q*min}{max-min}\]
\end{itemize}

\subsubsection*{Nicht-Affine Punktoperationen\index{Nicht-Affine Punktoperationen}}

Beliebige Abbildungsfunktion $$f \, : \, [0 \dots q] \to [0 \dots q]$$
Anwendung:
\begin{itemize}
\item Ausgleich von Sensor-Nichtlinearitäten
\item Gewichtung
\item Binarisierung
\end{itemize}

\subsubsection*{Histogramme\index{Histogramme}}

Histogrammfunktion: gibt die Häufigkeit eines Selektionsmerkmals an. Normalerweise: Grauwert
$$H_{Img}(x) = \# (u,v) \, : \, Img(u,v) = x \, , \, x \in [0 \dots q]$$

Histogrammdehnung\index{Histogrammdehung}:
\begin{itemize}
\item Verbesserung der Spreizung
\item Anstatt $min$ und $max$ werden Quantile verwendet
\item Akkumuliertes Histogramm berechnen: \[H_a(x):=\sum\limits_{k=0}^x H(k)\]
\item \(H_q(p_{min})\) und \(H_q(p_{max})\) (z.B. \(p_{min} = 0.1\) und \(p_{max} = 0.9\)) bestimmen mit: \[H_q(p) := \inf\{ x \in \{ 0, \dots ,q\}:H_a(x) \geq p*H_a(q) \}\]
\item Falls \(H_q(p_{min}) = H_q(p_{max})\): Bild ist homogen
\item Sonst:
\[ a:=\frac{q}{H_q(p_{max}) - H_q(p_{min})} \]
\[ b:=-\frac{q*H_q(p_{min})}{H_q(p_{max}) - H_q(p_{min})} \]
\[ Img' \gets \text{AffinePunktoperation}(Img,a,b) \]
\end{itemize}

Histogrammausgleich\index{Histogrammausgleich}:
\begin{itemize}
\item Histogrammausgleich: bessere Anpassung an das Sehvermögen des Menschen; kein Informationsgewinn \\
$H_n(0)$ \verb|:=| $H_{Img}(0)$ \\
\verb|for| $x$ \verb|:=| $1$ \verb|to| $q$ \verb|do| \\
\verb|  |$H_n(x)$ \verb|:=| $H_n(x-1) + H_{Img}(x)$ \\
\verb|endfor|
\item Automatische Kontrast- und Helligkeitsapassung durch Histogrammanalyse \\
$H_n(0)$ \verb|:=| $H_{Img}(0)$ \\
\verb|for| $x$ \verb|:=| $1$ \verb|to| $q$ \verb|do| \\
\verb|  |$H_n(x)$ \verb|:=| $H_n(x) \cdot \frac{q}{width \cdot heigth}$ \\
\verb|endfor|
\item Bei bereits berechnetem Histogramm $H_{Img}(x)$ kann der Histogrammausgleich durch den folgenden Algorithmus berechnet werden: \\
$H_n(0)$ \verb|:=| $H_{Img}(0)$ \\
\verb|for| $y$ \verb|:=| $0$ \verb|to| $height - 1$ \verb|do| \\
\verb|  for| $x$ \verb|:=| $0$ \verb|to| $width-1$ \verb|do| \\
\verb|    | $Img'(u,v)$ \verb|:=| $H_n(Img(u,v))$ \\
\verb|  endfor| \\
\verb|endfor|
\end{itemize}

\subsubsection*{Heute}

\begin{itemize}
\item Bildanalyse durch Frequenzanalyse
\item Filter in Frequenzbereich
\item Filter in Ortsbereich
\end{itemize}

\subsection{Bildanalyse durch Frequenzanalyse}

\begin{itemize}
\item (Grauwert-) Bilder lassen sich signaltheoretisch als Summe verschiedenfrequenter Signale betrachten.
\item Niedrige Frequenzen: Schwache Grauwertübergänge
\item Hohe Frequenzen: Scharfe Grauwertübergänge
\item Nützlich z.B. zum Finden gerader Linien
\end{itemize}
$\to$ Fourier-Analyse!

\subsubsection*{2-Dim Fouriertransformation}

Kontinuierliches 2-dimensionales Signal: $$F(u,v) = \int\limits_{x = - \infty}^{\infty} \int\limits_{y = - \infty}^{\infty} f(x,y) e^{-2i\pi (ux + vy)} dxdy$$
Diskretes 2-dimensionales Signal: $$F(u,v) = \sum\limits_{x = - \infty}^{\infty} \sum\limits_{y = - \infty}^{\infty} f[x,y] e^{-2i\pi (ux + vy)T}$$

\subsubsection*{Fouriertransformation in der Bildverarbeitung}

DFT bei Bild mit ImgSize $[0 \leq x \leq M][0 \leq y \leq N]$: $$F(u,v) = \sum\limits_{y=0}^{N-1} \left( \sum\limits_{x=0}^{M-1} f[x,y] e^{-2i\pi \frac{ux}{M}} \right) \cdot e^{-2i\pi \frac{vy}{N}}$$
DFT bei quadratischem Bild mit ImgSize $[0 \leq x \leq N][0 \leq x \leq N]$: $$\sum\limits_{y=0}^{N-1} \sum\limits_{x=0}^{N-1} f[x,y] e^{\frac{-2i\pi (ux + vy)}{N}}$$
Anschaulich: Durchführung der 1D-DFT auf jeder Zeile und Speicherung der Daten in Matrix (inndere Klammer). Durchführung der 1D-DFT auf jeder Spalte der ermittelten Matrix.

\begin{itemize}
\item Gewichtete Summation aller Bildpunkte
\item Zerlegung des Bildes in Sinus- und Cosinusfunktionen
\item Je weiter ein Punkt im Spektrum vom Bildmittelpunkt entfernt ist, desto höher ist seine darstellende Frequenz $u$ bzw. $v$.
\item D.h. im Bildinneren tiefe Frequenzen, in äußeren Bereichen hohe Frequenzen
\item Anwendung:
\begin{itemize}
\item Bildanalyse (z.B. Muster-, Geschwindigkeitserkennung)
\item Bildfilterung (z.B. Tiefpass)
\item Bildkompression (z.B. in jpeg Format)
\end{itemize}
\end{itemize}

\subsubsection*{Fouriertransformation}

\begin{itemize}
\item Die Variablen $u$ und $v$ heissen Frequenzvariablen
\item $F(u,v)$ ist komplexe Funktion
\item $F(u,v)$ ist darstellbar als 2 Bilder
\begin{itemize}
\item in Realteil und Imaginärteil $$F(u,v) = R(u,v) + I(u,v)$$
\item oder Betrag (auch Spektrum genannt, oft logarithmisch dargestellt) und Phase $$F(u,v) = |F(u,v)| \cdot e^{i \varphi (u,v)}$$
\end{itemize}
\item Quadrat des Spektrums heisst spektrale Dichte.
\end{itemize}

\subsection{Bildbearbeitung}

\begin{itemize}
\item Durch Filter im Ortsbereich oder Transferfunktionen im Frequenzbereich.
\item Ortsbereich:
\begin{itemize}
\item Manipulation von Grauwerten
\item anschaulich
\item häufig: Punktoperationen, Glättung
\end{itemize}
\item Frequenzbereich:
\begin{itemize}
\item Manipulation der Frequenzanteile
\item keine unmittelbare bildliche Vorstellung
\item häufig: starke Glättung, frequenzselektive Filter
\end{itemize}
\end{itemize}

\subsubsection*{Bildbearbeitung im Ortsbereich}

Faltung zweier Funktionen 1D kontinuierlich: $$h(x) = f(x) * g(x) = \int\limits_{a = - \infty}^{\infty} f(a) g(x-a) da$$
Faltung zweier Funktionen 1D zeitdiskret: $$h[x] = f[x] * g[x] = \sum\limits_{a = - \infty}^{\infty} f[a] g[x-a]$$
Faltung zweier Funktionen 2D kontinuierlich: $$h(x,y) = f(x,y) * g(x,y) = \int\limits_{a = -\infty}^{\infty} \int\limits_{b = -\infty}^{\infty} f(a,b) g(x-a,y-b) dadb$$
Faltung zweier Funktionen 2D zeitdiskret: $$h[x,y] = f[x,y] * g[x,y] = \sum\limits_{a = -\infty}^{\infty} \sum\limits_{b = -\infty}^{\infty} f[a, g[x-a,y-b]$$

\subsubsection*{Faltung}

\begin{itemize}
\item Bildmatrizen werden in den relevanten Randbereichen mit Nullen gefüllt.
\item Der neue Bildwert ist eine gewichtete Summe der Pixel die unter der gespiegelten Matrix liegen.
\item Als Gewichte dienen die Matrizenwerte.
\item Bildfilterung ist die Faltung einen Bildes mit einer Filtermatrix bzw. Maske.
\item Die Transferfunktion $H(u,v)$ ist die Fouriertransformierte der Filterfunktion $h(x,y)$.
\end{itemize}

\subsubsection*{Filteroperationen}

\begin{itemize}
\item Glättungsoperatoren (Rauschelimination): Mittelwertfilter, Gauß
\item Kantendetektoren: Prewitt, Sobel, Roberts, Laplace
\item kombinierte Filter: Laplacian of Gauß
\end{itemize}

\subsection{Filter}

\subsubsection*{Gauß\index{Gauß-Filter}}

Gaußfilter mit Radius $\sigma$ definiert durch die Gaußfunktion $$G(x,y) = \frac{1}{2 \pi \sigma^2} e^{-\frac{x^2 + y^2}{2\sigma^2}}$$

\subsubsection*{Mittelwrtfilter\index{Mittelwertfilter}}

Ziel: Störunterdrückung \\ Beispiel: Durchschnitt aus 8-Umgebung und Punkt. Größe beliebig wählbar.
$$m'(x,y) = \sum\limits_{j=-1}^1 \sum\limits_{i=-1}^1 m(i,j) \cdot p(x-j,y-i)$$

\subsubsection*{Anwendung der nachfolgenden Filter}
am Beispiel des Sobel-X Filters:
\mypic{12}{filterbsp}

\subsubsection*{Prewitt\index{Prewitt-Filter}}

\textbf{Prewitt-X Filter} $$P_x = \frac{\partial g(x,y)}{\partial x}$$ approximiert durch $$p_x = \left( \begin{array}{ccc} -1 & 0 & 1 \\ -1 & 0 & 1 \\ -1 & 0 & 1 \end{array} \right)$$
Kantendetektion: vertikal gut, horizontal schlecht \\
\textbf{Prewitt-Y Filter} $$P_y = \frac{\partial g(x,y)}{\partial y}$$ approximiert durch $$p_y = \left( \begin{array}{rrr} -1 & -1 & -1 \\ 0 & 0 & 0 \\ 1 & 1 & 1 \end{array} \right)$$
Kantendetektion: vertikal schlecht, horizontal gut \\
\textbf{Prewitt-Operator}
\begin{itemize}
\item Kombination der Prewitt-Filter zur Bestimmung des Grauwertgradientenbetrages $M$: $$M \thickapprox \sqrt{P_x^2 + P_y^2}$$
\item Danach: Schwellwertfilterung
\end{itemize}

\subsubsection*{Sobel\index{Sobel-Filter}}

\textbf{Sobel-X Filter} $$S_x = \frac{\partial g(x,y)}{\partial x}$$ approximiert durch $$s_x = \left( \begin{array}{ccc} -1 & 0 & 1 \\ -2 & 0 & 2 \\ -1 & 0 & 1 \end{array} \right)$$
Kantendetektion: vertikal gut, horizontal schlecht \\
\textbf{Sobel-Y Filter} $$S_y = \frac{\partial g(x,y)}{\partial y}$$ approximiert durch $$s_y = \left( \begin{array}{rrr} -1 & -2 & -1 \\ 0 & 0 & 0 \\ 1 & 2 & 1 \end{array} \right)$$
Kantendetektion: vertikal schlecht, horizontal gut \\
\textbf{Sobel-Operator}
\begin{itemize}
\item Kombination der Sobel-Filter zur Bestimmung des Grauwertgradienten-Betrages $M$: $$M \thickapprox \sqrt{S_x^2 + S_y^2}$$
\item Danach: Schwellwertfilterung
\end{itemize}

\subsubsection*{Roberts\index{Roberts-Filter}}

$$R(g(x,y)) = |R_x(g(x,y))| + |R_y(g(x,y))|$$ wobei $$R_x = \left( \begin{array}{rr} -1 & 0 \\ 0 & 1 \end{array} \right) \quad , \quad R_y = \left( \begin{array}{rr} 0 & -1 \\ 1 & 0 \end{array} \right)$$
Kantendetektion: diagonal gut

\subsubsection*{Laplace\index{Laplace-Filter}}

Laplace-Operator: $$\nabla^2 g(x,y) = \frac{\partial^2 g(x,y)}{\partial x^2} + \frac{\partial^2 g(x,y)}{\partial y^2}$$ wobei $$\nabla^2 \thickapprox \left( \begin{array}{rrr} 0 & 1 & 0 \\ 1 & -4 & 1 \\ 0 & 1 & 0 \end{array} \right)$$
Kantendetektion: Nulldurchgänge markieren Kanten, Subpixelgenauigkeit erreichbar \\ Näherung des Laplace-Operators: $$\nabla^2 \thickapprox \left( \begin{array}{rrr} 1 & 4 & 1 \\ 4 & -20 & 4 \\ 1 & 4 & 1 \end{array} \right)$$
Kantendetektion: Stärkere Kanten, aber mehr Störkanten

\subsubsection*{Laplacian of Gauß (LoG)\index{Laplacian of Gauß-Filter}}

Der Laplace-Operator ist gegen Rauschen sehr empfindlich. Wesentlich bessere Ergebnisse erhält man, wenn man das Bild zunächst mit einem Gauß-Filter glättet und danach den Laplace-Operator anwendet. $$LoG(g(x,y)) = \nabla^2 (G(x,y) * g(x,y))$$ Approximation (Faltung mit Matrix): $$\nabla^2 G(x,y) = \left( \begin{array}{rrrrr} 0 & 0 & -1 & 0 & 0 \\ 0 & -1 & -2 & -1 & 0 \\ -1 & -2 & 16 & -2 & -1 \\ 0 & -1 & -2 & -1 & 0 \\ 0 & 0 & -1 & 0 & 0 \end{array} \right)$$
Kantendetektion: Stärkere Kanten,weniger Rauschen

\subsubsection*{Canny-Kantendetektor\index{Canny-Kantendetektor}}
\begin{enumerate}
\item Rauschunterdrückung: Gauß-Filter
\item Kanten detektieren:
	\begin{itemize}
	\item Prewitt oder Sobel
	\item Zusammenrechnen mit \(B' = \sqrt{B_x^2 + B_y^2}\)
	\item Gradientenrichtung bestimmen: \(\Phi(x,y) = arctan\left(\frac{B_y(x,y)}{B_x(x,y)}\right)\)
	\item Runden auf 0°, 45°, 90° oder 135°
	\end{itemize}
\item Non-Maximum Surpression:
	\begin{itemize}
	\item Für jeden Pixel in Gradientenrichtung schauen, ob das Pixel davor oder dahinter einen höheren Wert hat. Falls ja, dann Pixel auf 0 setzen, falls nein, dann beibehalten.
	\end{itemize}
\item Hysterese-Schwellwerverfahren:
	\begin{itemize}
	\item Verwende zwei Schwellwerte \(T_1\) und \(T_2\) mit \(T_1 \leq T_2\)
	\item Markiere alle Pixel mit Werten größer \(T_2\) als Kantenpixel
	\item Setze alle Pixel mit Werten kleiner \(T_1\) auf 0
	\item Beginnend bei jedem Kantenpixel:
		\begin{itemize}
		\item Verfolge alle angrenzenden Kanten, solange Wert \(\geq T_1\)
		\item Markiere alle dazugehörigen Pixel als Kanten
		\end{itemize}
	\item Setze alle noch nicht als Kante markierten Pixel auf 0
	\end{itemize}
\end{enumerate}


\subsection{2D Bildverarbeitung}

\subsubsection*{Sensorische Erfassung}

Aufgaben der sensorischen Umwelterfassung:
\begin{itemize}
\item Wiedererkennung bekannter Sachverhalte: Objekte, Personen, Orte
\item Erlernen neuer Sachverhalte
\item Erkennung der eigenen Bewegung
\end{itemize}
Verfahren zur Lösung dieser Aufgaben:
\begin{itemize}
\item Sensorische Primitive, Segmentierung
\item Annahmen, Einschränkungen
\item Lernverfahren
\end{itemize}

\subsubsection*{Segmentierung\index{Segmentierung}}

Segmentierung ist die Aufteilung eines Bildes in aussagekräftige Bestandteile. Erlaubt:
\begin{itemize}
\item Aussagen über das Bild
\item Reduktion der Datenmenge
\item Verfolgung von Merkmalen über die Zeit / mehrere Sensoren
\end{itemize}
Beliebt sind: Kanten, Ecken, Textur, Farbe

\subsubsection*{Schwellwertfilterung\index{Schwellwertfilterung}}

Schwellwertfilterung zur Konvertierung eines Grauwertbildes in ein binäres Bild. Ziel: Trennung interessanter Objekte vom Hintergrund. $$Img'(u,v) = \left\{ \begin{array}{cl} q & \textrm{ falls } Img(u,v) \geq T \\ 0 & \textrm{ sonst} \end{array} \right.$$

\subsubsection*{Farbe}

Oft können Objekte über ihre Farbe segmentiert werden:
\begin{itemize}
\item menschliche Hautfarbe
\item einheitlich gefärbte Objekte
\end{itemize}
Problem:
\begin{itemize}
\item wechselnde Lichtbedingungen
\item Reflexionen, Schattenwürfe
\end{itemize}
Verfahren:
\begin{itemize}
\item Histogrammbasiert (z.B. in RGB, HSV bzw. RG, HS) \\ HS-Farbhistogramm:
\begin{itemize}
\item Weglassen des $I$-Kanals ergibt 2D-Histogramm
\item Training eines Klassifikators auf dem Histogramm
\end{itemize}
\item mit Hilfe der Mahalanobis-Distanz (z.B. in RGB): \\ gegeben: $x_i = (R,G,B)^T$ sind manuell positiv klassifizierte Pixel
\begin{eqnarray*}
C &=& \frac{1}{n-1} \sum\limits_{i=1}^n (x_i - \overline{x})(x_i - \overline{x})^T \qquad \textrm{Kovarianzmatrix} \\ p(x) &=& e^{- \frac{1}{2 \sigma^2} x^T C^{-1} x} \qquad \textrm{Berechnung der Farbwahrsch.}
\end{eqnarray*}
\item Klassifikation durch Verwendung von Neuronalen Netzen
\item durch Intervallschranken im HSI-Farbraum:
$$f(H,S,V) = H \geq H_{\min} \wedge H \leq H_{\max} \wedge S \geq S_{\min} \wedge S \leq S_{\max} \wedge V \geq V_{\min} \wedge V \leq V_{\max}$$
\end{itemize}

\subsubsection*{Morphologische Operatoren\index{Morphologische Operatoren}}

\begin{itemize}
\item Wähle Struckturelement, z.B. Quadrat mit 3x3 Pixel
\item Wandere mit Struckturelement über Bild und führe Operation durch
\item \textbf{Dilatation}\index{Dilatation} ist ein Operation auf Binärbildern. Alle Bereiche, die farbig (1 und nicht 0, da Binär) sind, werden ausgedehnt. Immer wenn unter dem mittleren Pixel des Struckturelements ein farbiges Pixel ist, werden alle Pixel unter dem Struckturelement als farbig markiert.
\item \textbf{Erosion}\index{Erosion} ist die Komplemetäroperation zu Dilatation. Nur falls alle Pixel unter dem Struckturelement farbig sind, wird das Element in der Mitte farbig gelassen und alle anderen auf 0 gesetzt. Falls nicht alle Pixel farbig sind, werden alle auf 0 gesetzt.
\item Öffnen-Operation\index{Öffnen-Operation}: Zuerst Erosion und anschließend Dilatation anwenden.
\item Schließen-Operation\index{Schließen-Operation}: Zuerst Dilatation und anschließend Erosion anwenden.

\end{itemize}


\subsubsection*{Bewegung}

Einfacher Ansatz: Differenzbilder
\begin{itemize}
\item Subtraktion aufeinander folgender Bilder einer Video-Sequenz: $$Img_t'(u,v) = |Img_t(u,v) - Img_{t-1}(u,v)|$$
\item Anschließend kann auf $Img_t'$ Schwellwertfilterung durchgeführt werden
\item Regionen, in denen sich etwas bewegt, erscheinen weiß; ruhige Regionen erscheinen schwarz
\item Bewegung in homogenen Regionen wird nicht erkannt (Kanten, Textur sind notwendig)
\item Richtung der Bewegung wird nicht erkannt
\end{itemize}
Differenzbilder werden auch für Hintergrundsubtraktion verwendet, dann wird $Img_{t-1}$ durch ein festes $Img_0$ ersetzt. Weitere Ansätze: Optical Flow und Erweiterungen

\subsubsection*{Region Growing\index{Region Growing}}

gegeben: Graustufen-Bild, gesucht: zusammenhängende Regionen \\ Algorithmus in Pseudocode:
\begin{enumerate}
\item wähle Saatpunkt $p_0 = (u_0,v_0)$
\item initialisiere Region $R = \{ p_0 \}$, wähle Schwelle $\varepsilon$
\item solange $\exists p \in R$, $q \not\in R$ mit $||p-q|| \leq 1$ und $|Img(p_0) - Img(q)| \leq \varepsilon$ mache $R = R \cup \{ q \}$
\end{enumerate}

\subsubsection*{Kanten}

Vom Menschen konstruierte Umgebungen:
\begin{itemize}
\item gut strukturiert
\item viele gerade Linien (Wände, Türen, Schränke)
\item einfache Segmentierung / 3D-Rekonstruktion
\item viele Informationen in einem einzigen Merkmal
\end{itemize}
Vorgehensweise: $$\textrm{Bildaufnahme} \quad \Rightarrow \quad \textrm{Filtern, Binarisieren} \quad \Rightarrow \quad \textrm{Pixel } \to \textrm{ Kantensegmente}$$

\subsubsection*{Pixel $\to$ Kanten}

\textbf{Iterative Endpoint Fit:}\index{Iterative Endpoint Fit} \\[0,1cm]
gegeben: Punkte $P$, Linien $L = \{ \}$, Distanzschwelle $d$
\begin{itemize}
\item Fine $x_1$, $x_2$ aus $P$ mit $||x_1 - x_2|| = \max$; \\ verbinde sie durch Linie $l_0 = \{x_1,x_2\}$; $L=L \cup \{l_0\}$
\item Entferne $x_1$, $x_2$ aus $P$
\item Für alle $l \in L$:
\begin{itemize}
\item Finde $x \in P$ mit $||l-x|| = \max$
\item Wenn $||l-x|| < d$:
\begin{itemize}
\item Ordne $x$ als Mitgliedpunkt $l$ zu
\item Entferne $x$ aus $P$
\end{itemize}
\item Sonst
\begin{itemize}
\item Brich $l$ in $l_1 = \{ x_1 , x\}$ und $l_2 = \{x , x_2 \}$ auf
\item Andere Mitgliedspunkte von $l$ wieder in $P$
\end{itemize}
\item $P$ leer $\Rightarrow$ Abbruch, sonst weiter
\end{itemize}
\item Lösche Linien mit weniger als $n$ Punkten
\end{itemize}

\textbf{Hough-Transformation:}\index{Hough-Transformation}
\begin{itemize}
\item Ziel: Erkennung gerader Linien im Bild
\item Ansatz: Stelle Linie durch Normalenvektor (Länge, Winkel) in Polarkoordinaten dar (Sinus-Kosinus-Kurve)
\item Kurven für kollineare Punkte schneiden sich in genau zwei Punkten $$r = x \cdot \cos(\theta) + y \cdot \sin(\theta)$$
\item Transformation in den Hough-Raum: Additives Eintragen aller Sinus-Kosinus-Kurven für alle Pixel in ein Histogramm
\item Finden der Maxima bzw. Cluster von "{}Treffern"{} im Hough-Raum
\item Brute-Force-Ansatz; für ein $n \times n$-Bild liegt die Laufzeit in $O(n^3)$
\end{itemize}
Unterschied zur Regressionsanalyse: Die Bestimmung der Regressionsgerade ist das geeignetere Verfahren, wenn schon klar ist, welche Pixel eine Gerade bilden sollen; dann ist die Regressionsgerade die optimale Gerade im Sinne der Summe der Fehlerquadrate. Die Regressionsgerade kann jedoch keine Segmentierung vornehmen. Dahingegen lassen sich mit der Hough-Transformation in beliebigen Bildern geradlinige Strukturen berechnen; die punkte dazu müssen jedoch verhältnismäßig exakt auf einer Linie liegen.

\subsubsection*{Punktmerkmale}

Kanten/Konturen/Farbe können nicht immer für die Segmentierung herangezogen werden. Texturierte Objekte lassen sich in der Regel nicht durch die bislang vorgestellten Verfahren segmentieren. Lösung: Verwendung von \textsl{lokalen} Punktmerkmalen (auch genannt: Textmerkmale):
\begin{itemize}
\item Harris Corner Detector
\item Shi-Tomasi Features
\item SIFT-Features
\item Maximally Stable Extremal Regions
\end{itemize}
Punktmerkmal: $(2n+1) \times (2n+1)$-Pixel-Block um Pixel $p$. Fast immer basierend auf Grauwertbildern. Gewünschte Eigenschaft: Wiedererkennbarkeit \\ $\Rightarrow$ hoher Gradient in mehrerer Richtungen \\[0,1cm]
\textbf{Harris Corner Detector}\index{Harris Corner Detector}: \\
Sind die Eigenwerte der Matrix $$A = \left( \begin{array}{cc} \left( \frac{\partial Img(x,y)}{\partial x} \right)^2 & \frac{\partial Img(x,y)}{\partial x} \frac{\partial Img(x,y)}{\partial y} \\ \frac{\partial Img(x,y)}{\partial x} \frac{\partial Img(x,y)}{\partial y} & \left( \frac{\partial Img(x,y)}{\partial y} \right)^2 \end{array} \right)$$ groß, dann verursacht eine kleine Bewegung in beliebiger Richtung eine große Grauwertänderung. \\ Finden von Ecken durch Suche nach lokalen Maxima in: $$R = det(A) - k \cdot trace(A)^2 \quad , \quad k \thickapprox 0,04$$
Häufiges Problem:
\begin{itemize}
\item Wiederfinden bzw. Zuordnung von Punktmerkmalen für:
\begin{itemize}
\item Objekterkennung auf der Basis von Punktmerkmalen
\item Stereo-Sehen bzw. "{}Structure from Motion"{} (Korrespondenzproblem)
\end{itemize}
\item Lösung des Korrespondenzproblems erfolgt für Punktmerkmale durch Korrelationsverfahren:
\begin{itemize}
\item \textbf{Sum of Squared Differences}\index{Sum of Squared Differences} (SSD) wird minimal bei guter Übereinstimmung: $$\sum\limits_{i = -n}^n \sum\limits_{j = -n}^n (Img_0(x-i,y-j) - Img_1(x-i,y-j))^2$$
\item \textbf{Sum of Absolute Differences}\index{Sum of Absolute Differences} (SAD) wird minimal bei guter Übereinstimmung: $$\sum\limits_{i = -n}^n \sum\limits_{j = -n}^n |Img_0(x-i,y-j) - Img_1(x-i,y-j)|$$
\item \textbf{(Zero Mean) Cross Correlation}\index{(Zero Mean) Cross Correlation} wird maximal bei guter Übereinstimmung: $$\sum\limits_{i = -n}^n \sum\limits_{j = -n}^n (Img_0(x-i,y-j)- \overline{Img_0})(Img_1(x-i,y-j)- \overline{Img_1})$$ wobei $$\overline{Img} \, : \, \textrm{Durchschnitt}$$
\item \textbf{Zero Mean Normalized Cross Correlation}\index{Zero Mean Normalized Cross Correlation} wird maximal bei guter Übereinstimmung: $$\frac{\sum\limits_{i=-n}^{n}\sum\limits_{j=-n}^{n}Img_1(u_1 + i, v_1 + j) - \overline{Img_1}(u_1,v_1,n)) \cdot (Img_2(u_2 + i, v_2 + j) - \overline{Img_2}(u_2,v_2,n))}{\sqrt{\sum\limits_{i=-n}^n \sum\limits_{j=-n}^n (Img_1(u_1 + i, v_1 + j) - \overline{Img_1}(u_1,v_1,n))^2 \sum\limits_{i=-n}^n \sum\limits_{j=-n}^n Img_2(u_1 + i, v_1 + j) - \overline{Img_2}(u_1,v_1,n))^2} }$$
wobei $$\overline{Img}(u,v,n) = \frac{1}{(2n+1)^2} \sum\limits_{i=-n}^n \sum\limits_{j=-n}^n Img(u+i,v+j)$$
\end{itemize}
\end{itemize}
Für die Objekterkennung geschieht das Wiederfinden oftmals unter Verwendung:
\begin{itemize}
\item der Hauptkomponentenanalyse oder engl. \textsl{Principal Component Analysis} (PCA) und Nearest-Neighbor- bzw. $k$-Nearest-Neighbor-Klassifikator
\item von Neuronalen Netzen
\item oder Kombinationen (zuerst PCA zur Kompression, dann SVM (\textsl{Support Vector Machine}) zur Klassifikation)
\end{itemize}

\subsection{Geometrische 2D-Transformationen}

\subsubsection*{Translation}\index{Translation}

Translation eines 2D-Vektors: $$\left( \begin{array}{c} x_0 \\ y_0 \end{array} \right) + \left( \begin{array}{c} x \\ y \end{array} \right) = \left( \begin{array}{c} x_0 + x \\ y_0 + y \end{array} \right)$$

\subsubsection*{Rotation}\index{Rotation}

\begin{itemize}
\item o.B.d.A. auf Einheitsvektor zurückführbar (Basistransformation)
\item Konvention: Rechtskoordinatensystem
\item Rotation von $(x_0,y_0)$ um Winkel $\beta$ mit Ergebnis $(x,y)$:
\end{itemize}
Aus Additionstheorem:
\begin{eqnarray*}
x &=& \cos(\alpha + \beta) = \cos(\alpha) \cos(\beta) - \sin(\alpha) \sin(\beta) \\ y &=& \sin(\alpha + \beta) = \sin(\beta) \cos(\alpha) + \cos(\beta) \sin(\alpha)
\end{eqnarray*}
und mit $(x_0,y_0) = (\cos(\alpha), \sin(\alpha))$:
$$\left( \begin{array}{c} x \\ y \end{array} \right) = \left( \begin{array}{rr} \cos(\beta) & -\sin(\beta) \\ \sin(\beta) & \cos(\beta) \end{array} \right) \left( \begin{array}{c} x_0 \\ y_0 \end{array} \right)$$

\subsubsection*{Homogene Koordinaten}

Homogene Koordinaten $$h = (h_0,h_1, \dots, h_i, h_{i+1})$$ eines Punktes $p$ im $R^i$ mit $$p = (p_0,\dots,p_i)$$ sind Zahlen, für die gilt: $$p_k = \frac{h_k}{h_{i+1}} \quad \forall 0 \leq k \leq i$$

\subsubsection*{Homogene 2D-Transformationen}

Transformationen definiert durch Rotation $R$ und Translation $t$.
$$\myvectwo{x}{y} = R \myvectwo{x_0}{y_0} + t = \myvecfour{r_{11}}{r_{12}}{r_{21}}{r_{22}} \myvectwo{x_0}{y_0} + \myvectwo{t_x}{t_y}$$
Darstellung mit Hilfe homogener Koordinaten und einer geschlossenen Transformationsmatrix:
$$\myvecthree{x}{y}{1} = \left( \begin{array}{cc|c} & R & t \\ \hline 0 & 0 & 1 \end{array} \right) \myvecthree{x_0}{y_0}{1} = \left( \begin{array}{cc|c} r_{11} & r_{12} & t_x \\ r_{21} & r_{22} & t_y \\ \hline 0 & 0 & 1 \end{array} \right) = A \myvecthree{x_0}{y_0}{1}$$

\subsubsection*{Partikel Filter und 2D-Tracking}

Für Tracking-Applikationen wird oftmals das Kalmanfilter, das Partikel Filter oder Kombinationen aus beiden verwendet. Vorteile des Partikel Filters gegenüber dem Kalmanfilter:
\begin{itemize}
\item Es kann automatisch mehrere, parallel existierende Hypothesen halten $\Rightarrow$ geringere Gefahr in lokalen Minima zu verharren.
\item Es kann beliebige Wahrscheinlichkeitsdichten modellieren und dadurch nichtlineare Bewegungen auf natürliche Weise verfolgen.
\end{itemize}
Nachteil: Rechenaufwand ist proportional zur Anzahl der notwendigen Partikel, welche (bei konstanter Auflösung) mit der Dimension des Konfigurationsraums exponentiell steigt. \\ Im Kern des Partikel Filters befinden sich ein Modell mit Konfigurationsraum $R^n$. \\ \textit{Eingabe:} Beobachtungen $z$; hier: Bilder einer Videosequenz \\ \textit{Ausgabe:} Schätzung der Konfiguration $s \in R^n$, die den aktuellen Beobachtungen $z$ entspricht \\ \textit{Zentrale Funktion:} Bewertungsfunktion $\myprob{z}{s}$, die die a-posteriori Wahrscheinlichkeit berechnet, dass $s$ die zu $z$ passende Konfiguration ist. \\
Das Partikel Filter modelliert die Wahrscheinlichkeitsdichtefunktion oder engl. \textsl{probability density function (pdf)} durch eine feste Anzahl von $N$ Partikeln. Die Wahrscheinlichkeitsdichtefunktion beschreibt die a-posteriori Wahrscheinlichkeiten für den Konfigurationsraum. Jedes Partikel ist ein Paar $(s_i, \pi_i)$, wobei $s_i$ eine Konfiguration ist und $\pi_i$ die dazugehörige a-posteriori Wahrscheinlichkeit mit $\sum_{i=1}^n \pi_i = 1$. Die aktuelle Schätzung des Partikel Filters erfolgt über das gewichtete Mittel über alle Partikel: $$\overline{s} = \sum\limits_{i=1}^{N} \pi_i \cdot s_i$$

\textbf{\textsl{Algorithmus}}

\begin{enumerate}
\item Initialisiere alle $N$ Partikel (z.B. mit Gleichverteilung)
\item Verarbeite neue Beobachtungen (z.B. Vorverarbeitung neuer Bilder)
\item Ziehe $N$ Partikel aus der letzten Generation, proportional zu ihrer Wahrscheinlichkeit $\pi_i$, und für jedes dieser Partikel:
\begin{itemize}
\item Berechne neue Konfiguration auf Basis der alten Konfiguration durch Addition normalverteilten Rauschens und evtl. durch Hinzunahme eines dynamischen Modells.
\item Berechne für diese neue Konfiguration die neue a-posteriori Wahrscheinlichkeit mit Hilfe der Bewertungsfunktion $\myprob{z}{s}$
\end{itemize}
\item Berechne aktuelle Schätzung $\overline{s}$ über alle Partikel
\item Fahre fort mit Schritt 2
\end{enumerate}

\textbf{\textsl{Einfaches Beispiel für eine Bewertungsfunktion}}

\begin{itemize}
\item Anwendung: \\ 2D-Tracking einer dichten, in etwa quadratischen Fläche von in etwa fester Größe in einem binarisierten Bild
\item Modell: \\ Quadrat fester Größe mit Kantenlänge $k$ mit Konfigurationsraum $R^2$ (Koordinaten $u,v$ im Bild)
\item Bewertungsfunktion: $$\myprob{z}{s} \propto e^{- \frac{1}{2 \sigma^2} \left( k^2 - \sum_{m \in M} g_m \right)}$$ wobei $M$ die Menge aller Pixel im binarisierten Bild $z$ (Beobachtungen) im durch $s$ (Konfiguration) definierten Quadrat beschreibt und $g_m$ deren Intensität aus $\{0,1\}$.
\end{itemize}

\subsection{Fragen zum Kapitel}
\begin{enumerate}
  %SS 12 Klausur
	\item Was ist der maximale Korrelationswert, den die Zero Mean Normalized Cross Correlation (ZNCC) liefern kann?
	%WS 12 Klausur
	\item Warum kann man mit Hilfe der Epipolargeometrie das Sterokorrespondenzproblem schneller lösen?
\end{enumerate}

% Fähigkeiten
% 1. Anwendung aller besprochener Filter
% 2. Quaternionen rechnen
% 3. Rechnen mit Rotationen und Translationen
% 4. Eine Spreizung vornehmen können








% !TeX root = summary.tex

\chapter{Klassifikation}

\begin{itemize}
	\item Einordnen in die Welt
	\begin{itemize}
		\item Gesellschaftlich definierte Konventionen: Buchstaben, menschliche Artifakte, Kredite
		\item Biologisch definierte Kategorien: Katze, Hund
	\end{itemize}
	\item Komplex
	\begin{itemize}
		\item Was definiert einen Stuhl? Eine Katze?
		\item Beziehungen zu Komplex, Regeln -> Lernen
		\item Nie 100\% richtig -> Wahrscheinlichkeit
	\end{itemize}
\end{itemize}

\begin{itemize}
	\item Eine einfache Form der Klassifikation ist Template Matching
	\item Ziel: Ein Muster zu erkennen das einem abgespeicherten Beispiel ähnlich ist.
	\item Maß der Übereinstimmung zwischen Muster und Schablone zentriert an (m,n) muss maximiert werden
	$M_{f,g}(m,n)=\sum_i\sum_j f(i,j)g(i-m,j-n)$
\end{itemize}

Pattern Recognition Overview
\begin{itemize}
	\item Static Patterns, no dependence on Time or Sequential Order
	\item Important Notions
	\begin{itemize}
		\item Supervised - Unsupervised
		\item Parametric - Non-Parametric
		\item Linear - Non-linear
	\end{itemize}
	\item Classical Methods
	\begin{itemize}
		\item Bayes Classifier
		\item k-nearest neighbor
	\end{itemize}
	\item Connectionist Methods
	\begin{itemize}
		\item Perceptron
		\item Multilayer Perceptrons
	\end{itemize}
\end{itemize}

\section{Schablonenanpassung\index{Schablonenanpassung} (\textsl{Template Matching})}
\begin{itemize}
\item Eine einfache Form der Klassifikation.
\item Ziel: Ein Muster zu erkennen das einem abgespeicherten Beispiel ähnlich ist.
\item Maß der Übereinstimmung ist der Absolutbetrag zwischen Muster und Schablone (zentriert an $(m,n)$): $$M_{f,g}(m,n) = \sum\limits_i \sum\limits_j | f(i,j) g(i-m,j-n)|$$
\item Abstand $E(m,n)$ ist definiert als Quadrat der Kreuzkorrelation: $$E_{f,g} (m,n) = \left( \sum\limits_i \sum\limits_j f(i,j) g(i-m,j-n) \right)^2$$
\end{itemize}

\section{Supervised -- Unsupervised Training}

\begin{description}
\item[Supervised Training] Die zu erkenndende Klasse ist für jede Auswahl in den Trainingsdaten bekannt. Benötigt a priori Wissen von nützlichen Eigenschaften und Kenntnis / Bezeichnung von jedem Trainingsmerkmal (Kosten!).
\item[Unsupervised Training] Die Klasse ist nicht bekannt und die Struktur muss automatisch herausgefunden werden.
\end{description}

\section{Parametrisch -- Nicht-parametrisch}

Parametrisch:
\begin{itemize}
\item grundlegende Aufteilungswahrscheinlichkeit annehmen
\item Parameter der Aufteilung abschätzen
\item Beispiel: "{}Gauss Klassifikation"{}
\end{itemize}
Nicht-parametrisch:
\begin{itemize}
\item keine Aufteilung annehmen
\item Fehlerwahrscheinlichkeit oder Fehlerkriterium direkt aus den Trainingsdaten berechnen
\item Beispiele: Parzen Fenster, $k$-nächster Nachbar, Perzeptron
\end{itemize}


\section{Bayes Entscheidungstheorie}

Bayes Regel\index{Bayes Regel}: $$P(\omega_j | x) = \frac{p(x | \omega_j) P(\omega_j)}{p(x)} \quad \textrm{wobei} \quad p(x) = \sum\limits_{j} p(x | \omega_j) P(\omega_j)$$
A priori Wahrscheinlichkeit: $$P(\omega_j)$$
A posteriori Wahrscheinlichkeit: $$P(\omega_j | x)$$
Klassenbedingte Wahrscheinlichkeitsdichte: $$p(x | \omega_j)$$

\section{Zwei Klassen Fall}

$$P(error | x) = \left\{ \begin{array}{cl} P(\omega_1 | x) & \textrm{wenn wir uns für } \omega_2 \textrm{ entscheiden} \\ P(\omega_2 | x) & \textrm{sonst} \end{array} \right.$$
Der Fehler ist minimiert, wenn wir uns entscheiden für:
\begin{itemize}
\item $\omega_1$ wenn $P(\omega_1 | x) > P(\omega_2 | x)$ \\ $\omega_2$ sonst
\item $\omega_1$ wenn $p(x | \omega_1)P(\omega_1) > p(x | \omega_2)P(\omega_2)$ \\ $\omega_2$ sonst
\end{itemize}
Mehr Klassen:
\begin{itemize}
\item $\omega_i$ wenn $P(\omega_i | x) > P(\omega_j | x)$ für alle $i \not= j$
\end{itemize}

\section{Klassifizierende Diskriminanzfunktionen}

$$g_i(x) \quad , \quad i = 1, \dots, c$$
Ordne $x$ der Klasse $\omega_i$ zu, wenn $g_i(x) > g_j(x)$ für alle $j \not= i$
\begin{eqnarray*}
g_i(x) &=& P(\omega_i | x) \\ &=& \frac{p(x | \omega_i) P(\omega_i)}{\sum\limits_{j=1}^c p(x | \omega_j)P(\omega_j)} \\ g_i(x) &=& p(x | \omega_i) P(\omega_i) \\
g_i(x) &=& \log(p(x | \omega_i)) + \log(P(\omega_i))
\end{eqnarray*}

\section{Classifier Design in Practice}
\begin{itemize}
	\item Need a priori probability $P(\omega_i)$ (not too bad)
	\item Need class conditional PDF $p(x/\omega_i)$
	\item Problems:
	\begin{itemize}
		\item limited training data
		\item limited computation
		\item class-labelling potentially costly and errorful
		\item classes may noat be known
		\item good features not known
	\end{itemize}
	\item Parametric Solution
	\begin{itemize}
		\item Assume that $p(x/\omega_i)$ has a particular prametric form
		\item Most common representative: multivariate normal density
	\end{itemize}
\end{itemize}

\section{Gauss Klassifizierer\index{Gauss Klassifizierer}}

Eindimensionale Normaldichte: $$p(x) = \frac{1}{\sqrt{2\pi} \sigma} e^{- \frac{1}{2} \left( \frac{\myvector{x} - \myvector{\mu}}{\sigma} \right)^2} \sim N(\myvector{\mu},\sigma^2)$$
Mehrdimensionale Dichte: $$p(x) = \frac{1}{(2\pi)^{d/2}|\Sigma|^{1/2}}e^{- \frac{1}{2} ( \myvector{x} - \myvector{\mu})^t \Sigma^{-1} (\myvector{x} - \myvector{\mu}) } \sim N(\myvector{\mu},\Sigma)$$
$$g_i(x) = - \frac{1}{2} (x - \mu_i)^t \sum_i^{-1} (x - \mu_i) - \frac{d}{2} \log(2\pi) - \frac{1}{2} \log|\Sigma_i| + \log P(\omega_i)$$
Für jede Klasse $i$ muss folgendes aus den Trainingsdaten berechnet werden:
\begin{itemize}
\item Kovarianz-Matrix $\Sigma_i$ \item Mittelwertsvektor $\myvector{\mu_i}$
\end{itemize}
\textbf{Schätzung der Parameter}
\begin{itemize}
\item MLE, Maximum Likelihood Estimation
\item Für den mehrdimensionalen Fall:
$$\myvector{\mu} = \frac{1}{N} \sum\limits_{k=1}^N \myvector{x_k}$$ $$\Sigma = \frac{1}{N} \sum\limits_{k=1}^N (\myvector{x_k} - \myvector{\mu})(\myvector{x_k} - \myvector{\mu})^T$$
\end{itemize}

\section{Probleme beim Klassifikationsentwurf}

Merkmale:
\begin{itemize}
\item Welche und wie viele Merkmale sollten gewählt werden?
\item Beliebige Merkmale?
\item Je mehr desto besser?
\item Wenn zusätzliche Merkmale nicht nützlich sind, sollen sie dann automatisch ignoriert werden?
\end{itemize}
\textbf{Der Unsegen der Dimensionalität}
\begin{itemize}
\item Allgemein gilt: das Hinzufügen von Eigenschaften verschlechtert die Performance!
\item Grund: Trainingsdaten vs. Anzahl der Parameter; beschränkte Trainingsdaten
\item Lösung: Eigenschaften sorgfältig wählen; Dimension verringern; Principle Component Analysis
\end{itemize}

\section{Principal Component Analysis (PCA)}
\begin{itemize}
	\item Assumption: Single dimensions are correlated
	\item Aim: Reduce number of dimensions with minimum loss of information
	\item Remove dimensions with low variance
\end{itemize}

\section{Risiko}

\begin{itemize}
\item Es kann zu Entscheidungsverweigerungen kommen in mehrdeutigen Fälle ($\to$ Bandbreite)
\item Bewerte die Kosten für jede Entscheidung (etwas Kostenaufwendiger als anders) $$\Omega = \{ \omega_1, \dots, \omega_s \} \,\, s \textrm{ Zustände der Eigenschaften}$$ $$A = \{ \alpha_1, \dots, \alpha_a \} a \textrm{ mögliche Aktionen}$$
\end{itemize}

\textbf{Verlustfunktion}
$\lambda(\alpha_i | \omega_j)$: Verlust der Aktion $\alpha_i$ beim gegebenen Zustand $\omega_j$ $$P(\omega_j | \myvector{x}) = \frac{P(\myvector{x} | \omega_j) P(\omega_j)}{P(\myvector{x})}$$
Angenommener Verlust von Aktion $\alpha_i$: $$R(\alpha_i | \myvector{x}) = \sum\limits_{j=1}^s \lambda(\alpha_i | \omega_j) P(\omega_j | \myvector{x}) \qquad \textrm{(bedingtes Risiko)}$$
Minimierung des angenommenen Verlusts indem man die Aktion $\alpha_i$ wählt, die das bedingte Risiko minimiert.
\subsubsection*{Zwei Kategorien Fall}
$$\lambda(\alpha_i | \omega_j) \triangleq \lambda_{ij}$$
\begin{eqnarray*}
R(\alpha_1 | \myvector{x}) &=& \lambda_{11} P(\omega | \myvector{x}) + \lambda_{12} P(\omega_2 | \myvector{x}) \\
R(\alpha_2 | \myvector{x}) &=& \lambda_{21} P(\omega | \myvector{x}) + \lambda_{22} P(\omega_2 | \myvector{x}) \\
\end{eqnarray*}

\begin{itemize}
\item Wähle $\omega_1$, wenn $R(\alpha_1 | \myvector{x}) < R(\alpha_2 | \myvector{x})$
\item Wähle $\omega_1$, wenn $(\lambda_{21} - \lambda_{11} P(\omega_1 | \myvector{x}) > (\lambda_{12} - \lambda_{22} P(\omega_2 | \myvector{x})$
\item Wähle $\omega_1$, wenn $(\lambda_{21} - \lambda_{11} P(\myvector{x} | \omega_1)P(\omega_1) > (\lambda_{12} - \lambda_{22} P(\myvector{x} | \omega_2) P(\omega_2)$
\item Wähle $\omega_1$, wenn $\frac{p(\myvector{x} | \omega_1)}{p(\myvector{x} | \omega_2)} > \frac{\lambda_{12} - \lambda_{22}}{\lambda_{21} - \lambda_{11}} \cdot \frac{P(\omega_2)}{P(\omega_1)}$
\end{itemize}


\section{Minimum Error Rate Classification}
\begin{itemize}
	\item Decision rule to minimize error rate
	\item Define zero-one loss function
	\item $\lambda(\alpha_i|\omega_j)=\left\{
	\begin{array}{c}
		0 : i=j \\
		1 : i\neq j
	\end{array}\right\} i,j=1...c
	$
	\item $R(\alpha_i|\vec{x})=\sum^{c}_{j=1}\lambda(\alpha_i|\omega_j)p(\omega_j|\vec{x})$
	\item $=\sum_{j\neq i}p(\omega_j|\vec{x})=1-p(\omega_i|\vec{x})$
	\item To minimize risk and the average probability of error, select i that maximizes posterior $p(\omega_i|\vec{x})$
	\item Decide $\omega_i$ if $p(\omega_i|\vec{x})>p(\omega_j|\vec{x})$ for all $j\neq i$
	\item Normal distribution does not model this situation well
	\item other densities may be mathematically intractable -> non-parametric techniques
\end{itemize}

\section{Parzen Fenster}

Es werden keine Annahmen über die Verteilung gemacht, stattdessen wird $p(x)$ direkt aus den Daten geschätzt.
\begin{itemize}
\item wähle ein Fenster mit dem Volumen $V$
\item zähle die Anzahl Samples, die in das Fenster fallen
\item $p(x) \thickapprox \frac{k/n}{V}$, $k = $ Anzahl, $n = $ Anzahl Samples
\end{itemize}
Probleme:
\begin{itemize}
\item Volumen zu groß \\ $\Rightarrow$ Auflösung geht verloren
\item Volumen zu klein \\ $\Rightarrow$ unbeständig, schlechte Abschätzung
\end{itemize}
Setze $$V_n = \frac{1}{\sqrt{n}}$$

\section{$k$-nächster Nachbar}

Volumen als Funktion der Daten. Verwende die $k$ nächsten Nachbarn für die Abschätzung. Setze $$k = \sqrt{n}$$ Um Sample $x$ zu klassifizieren:
\begin{itemize}
\item finde $k$ nächste Nachbarn von $x$
\item bestimme die am häufigsten vorkommende Klasse in diesen $k$ Samples
\item ordne $x$ dieser Klasse zu
\end{itemize}
Probleme: Für eine endliche Anzahl von Samples $n$ sollte $k$ möglichst
\begin{itemize}
\item groß sein für eine gute Abschätzung
\item klein sein, um zu garantieren, dass alle $k$ NAchbarn nah beieinander sind
\end{itemize}
Trainingsdatenbanken müssen groß sein.

\section{Entscheidungsfunktion $g(x)$}

\begin{eqnarray*}
g(\myvector{x}) > 0 &\Rightarrow& \textrm{Klasse A} \\ g(\myvector{x}) < 0 &\Rightarrow& \textrm{nicht Klasse A} \\ g(\myvector{x}) = 0 &\Rightarrow& \textrm{keine Entscheidung}
\end{eqnarray*}
$$g(\myvector{x}) = \sum\limits_{i=1}^n w_1x_1 + w_0 = \myvector{w}^T \myvector{x} + w_0$$
$\myvector{x} = (x_1,\dots,x_n)^T$ : Eigenschaftsvektor, $\myvector{w} = (w_1,\dots,w_n)^T$ : Gewichtungsvektor, $w_0$ : Schwellenwert


\section{Lineare Diskriminantenfunktionen}

\begin{itemize}
\item Keine Annahme über die Verteilung (Nicht-parametrisch)
\item Lineare Entscheidungsflächen
\item Start durch überwachtes Training (Klassen der Trainingsdaten gegeben)
\item Diskriminantenfunktion: $$g(x) = w_0 + \sum\limits_{i=1}^n w_ix_i = \sum\limits_{i=0}^n w_ix_i, \quad x_0 = 1$$
\item $g(x)$ ergibt die Distanz von der Entscheidungsfläche
\item Zwei Kategorien Fall:
\begin{eqnarray*}
g_1(x) > 0 &\Rightarrow& \textrm{Klasse 1} \\ g_1(x) < 0 &\Rightarrow& \textrm{Klasse 2}
\end{eqnarray*}
\end{itemize}

Hyperebe $H$: $g(\myvector{x}) = \sum\limits_{i=1}^n w_1x_1 + w_0 = \myvector{w}^T \myvector{x} + w_0 = 0$
$$\Rightarrow \myvector{x} = q \frac{\myvector{w}}{|| \myvector{w} ||} + r \frac{\myvector{w}}{|| \myvector{w} ||} + \myvector{x}_p$$
Vektor $q \frac{\myvector{w}}{|| \myvector{w} ||}$ entspricht: $g \left(q \frac{\myvector{w}}{|| \myvector{w} ||} \right) = 0 = q ||\myvector{w}|| + w_0$ $$\Rightarrow q = - \frac{w_0}{|| \myvector{w} ||}$$
Und mit $g(\myvector{x}) = \myvector{w}^T q \frac{\myvector{w}}{|| \myvector{w} ||} + \myvector{w}^T r \frac{\myvector{w}}{|| \myvector{w} ||} + \myvector{w} \myvector{x}_p + w_0 = -w_0 + r ||\myvector{w}|| + w_0$ erhalten wir $$r = \frac{g(\myvector{x})}{|| \myvector{w} ||}$$


\section{Fisher-lineare Diskriminante}

\begin{itemize}
\item Dimensionsreduktion
\item Projeziert eine Menge von mehrdimensionalen Punkten auf eine Line $y = \myvector{w} \myvector{x}$
\item Die Fisher Diskriminaten ist eine Funktion, die folgendes Kriterium maximiert $$g(x) = \frac{| \tilde{m}_1 - \tilde{m_2}|}{\tilde{s}_1 + \tilde{s}_2}$$
wobei $\tilde{m}_i = \frac{1}{n} \sum\limits_{y \in Y_i} y$ : Mittel für projezierte Muster, $\tilde{s}_i^2 = \sum\limits_{y \in Y_i} (y - \tilde{m}_i)^2$ : Streuung für projezierte Muster
\item Fisher's lineare Diskriminante:
\begin{eqnarray*}
\myvector{w} &=& s_w^{-1} (\myvector{m}_1 - \myvector{m}_2) \\ s_w &=& s_1 + s_2 \\ s_i &=& \sum\limits_{x \in X_i} (\myvector{x} - \myvector{m}_i)(\myvector{x} - \myvector{m}_i)^T
\end{eqnarray*}
\end{itemize}







% !TeX root = summary.tex

\chapter{Spracherkennung}

\subsection{Spracherkennungssystem}

\mypic{10}{speech1}

%Erkennungsprozess:

%\begin{center}
%\includegraphics[width=10cm]{pics/erkennung.eps}
%\end{center}

\subsection{Erkennung}

\begin{description}
\item[gegeben:] akustische Daten $A = a_1,a_2,\dots,a_k$
\item[Ziel:] Wortsequenz $W = w_1,w_2,\dots,w_n$ finden, so dass $\myprob{W}{A}$ miximiert wird.
\end{description}

Wiederholung: \textbf{Bayes Regel}\index{Bayes Regel} $$\myprob{W}{A} = \frac{\myprob{A}{W} \cdot P(W)}{P(A)}$$ wobei $\myprob{A}{W}$ : akustisches Modell (HMM), $P(W)$ : Sprachmodell, $P(A)$ : Konstante für einen kompletten Satz

\subsection{Hidden Markov Modelle\index{Hidden Markov Modelle} (HMM)}

Ein Hidden Markov Modell ist ein stochastisches Modell, das sich durch zwei Zufallsprozesse beschreiben lässt. \\
Der erste Zufallsprozess entspricht dabei einer Markow-Kette, die durch Zustände und Übergangs- wahrscheinlichkeiten gekennzeichnet ist. Die Zustände der Kette sind von außen jedoch nicht direkt sichtbar (sie sind verborgen). Stattdessen erzeugt ein zweiter Zufallsprozess zu jedem Zeitpunkt beobachtbare Ausgangssymbole gemäß einer zustandsabhängigen Wahrscheinlichkeitsverteilung. Die Aufgabe besteht häufig darin, aus der Sequenz der Ausgabesymbole auf die Sequenz der verborgenen Zustände zu schließen.

\subsubsection*{Elemente}

\begin{center}
\begin{tabular}{ll}
Menge an Zuständen: & $S = \{S_0, S_1, \dots, S_N\}$ \\
Übergangswahrscheinlichkeiten: & $P(q_t = S_i \, | \, q_{t-1} = S_j \} = a_{ij}$ \\
Ausgabewahrscheinlichkeitsverteilungen: & $P(y_t = O_k \, | \, q_t = S_j) = b_j(k)$ \\
(bei Zustand $j$ für Symbol $k$)
\end{tabular}
\end{center}

\subsubsection*{HMM Probleme und Lösungen}

\begin{description}
\item[Evaluation:] \textsl{Problem:} Bei einem gegebenen Modell soll die Wahrscheinlichkeit einer speziellen Ausgabesequenz bestimmt werden. \\ \textsl{Lösung:} \textbf{Forward-Algorithmus}\index{Forward-Algorithmus} und \textbf{Viterbi-Algorithmus}
\item[Dekodierung:] \textsl{Problem:} Es soll die wahrscheinlichste Sequenz der Zustände bestimmt werden, die eine vorgegebene Ausgabesequenz erzeugt hat. \\ \textsl{Lösung:} \textbf{Viterbi-Algorithmus}\index{Viterbi-Algorithmus}
\item[Training:] \textsl{Problem:} Gegeben ist nur die Ausgabesequenz. Es sollen die Parameter des HMM bestimmt werden, die am wahrscheinlichsten die Ausgabesequenz erzeugen. \\ \textsl{Lösung:} \textbf{Forward-Backward-Algorithmus}\index{Forward-Backward-Algorithmus}
\end{description}

\subsection{Evaluation}

Wahrscheinlichkeit einer Ausgabesequenz $O = O_1O_2 \dots O_T$ bei einem gegebenen Hidden Markov Modell $\lambda$ ist
\begin{eqnarray*}
\myprob{O}{\lambda} &=& \sum\limits_{\forall Q} \myprob{O \, , \, Q}{\lambda} \\ &=& \sum\limits_{\forall q_0, \dots, q_T} a_{q_0q_1} b_{q_2}(O_2) \cdots a_{q_{T-1}q_T}b_{q_T}(O_T)
\end{eqnarray*}
wobei $Q = q_0q_1\dots q_T$ eine Folge von Zuständen ist \\
\textbf{Nicht praktisch, da die Zahl der Wege in} $O(N^T)$ \textbf{liegt.} ($N$: Anzahl der Zustände im Modell, $T$: Anzahl der Ausgabesequenzen)

\subsubsection*{Der Forward Algorithmus}

$$\alpha_{t}(j) = P(O_1O_2 \dots O_t, \, q_t = S_j \, | \, \lambda)$$
rekursive Berechnung von $\alpha$:
\begin{eqnarray*}
\alpha_0(j) &=& \left\{ \begin{array}{cl} 1 & \textrm{ wenn } j \textrm{ Startzustand} \\ 0 & \textrm{ sonst} \end{array} \right. \\
\alpha_t(j) &=& \left( \sum\limits_{i=0}^N \alpha_{t-1}(i) a_{ij} \right) b_j (O_t) \quad t > 0
\end{eqnarray*}
($\myprob{O}{\lambda} = \alpha_T(S_N)$, Berechnung liegt in $O(N^2T)$)

\subsubsection*{Forward Trellis}

\mypictwo{8}{hmm_automat}{10}{forward_trellis}

\subsubsection*{Der Backward Algorithmus}

$$\beta_{t}(i) = P(O_{t+1} O_{t+2} \dots O_{T} \, , \, q_t = S_i \, | \, \lambda )$$
rekursive Berechnung von $\beta$:
\begin{eqnarray*}
\beta_0(i) &=& \left\{ \begin{array}{cl} 1 & \textrm{ wenn } i \textrm{ Endzustand} \\ 0 & \textrm{ sonst} \end{array} \right. \\
\beta_t(i) &=& \sum\limits_{j=0}^N a_{ij} b_j(O_{t+1}) \beta_{t+1} (j) \quad t < T
\end{eqnarray*}
($\myprob{O}{\lambda} = \beta_0(S_0) = \alpha_T(S_N)$, Berechnung liegt in $O(N^2T)$)

\subsubsection*{Backward Trellis}

\mypictwo{8}{hmm_automat}{10}{backward_trellis}

\subsection{Dekodierung}

\subsubsection*{Der Viterbi Algorithmus}

\begin{itemize}
\item Finde die Zustandsfolge $Q$, die $\myprob{O \, , \, Q}{\lambda}$ maximiert.
\item Verläuft ähnlich wie der Forward Algorithmus, jedoch wird das Maximum anstatt der Summe berechnet.
\end{itemize}
$$VP_t(i) = \max_{q_0, \dots, q_{t-1}} \myprob{O_1O_2 \dots O_{t} \, , \, q_t = i}{\lambda}$$
rekursive Berechnung:
\begin{eqnarray*}
VP_t(j) &=& \max_{i=0, \dots, N} VP_{t-1}(i) a_{ij}b_j(O_t) \quad t > 0 \\
P(O \, , \, Q \, | \, \lambda) &=& VP_T(S_N)
\end{eqnarray*}
Speicher jedes Maximum für die Ablaufverfolgung am Ende.

\subsubsection*{Viterbi Trellis}

\mypictwo{8}{hmm_automat}{10}{viterbi_trellis}

\subsection{Training}

\begin{itemize}
\item Trainiere Parameter vom Hidden Markov Modell
\begin{itemize}
\item $\lambda$ einstellen, um $\myprob{O}{\lambda}$ zu maximieren
\item kein effizienter Algorithmus für das globale Optimum
\item ein effizienter iterativer Algorithmus findet das lokale Optimum
\end{itemize}
\item Viterbi-Training
\begin{itemize}
\item berechne den Viterbi-Weg mittels aktuellem Modell
\item bewerte die Parameter neu, indem die Bezeichnung benutzt werden, die der Viterbi Algorithmus bestimmt hat
\end{itemize}
\item Baum-Welch\index{Baum-Welch} (Forward-Backward)
\begin{itemize}
\item berechne Wahrscheinlichkeiten mittels aktuellem Modell
\item Filtere $\lambda \to \lambda$ basierend auf den berechneten Weten
\item benutze $\alpha$ und $\beta$ vom Forward-Backward
\end{itemize}
\end{itemize}

\subsubsection*{Der Forward-Backward Algorithmus}

Wahrscheinlichkeit beim Übergang von $S_i$ nach $S_j$ zur Zeit $t$ bei gegebenem $O$:
\begin{eqnarray*}
\xi_t(i,j) &=& \myprob{q_t = S_i \, , \, q_{t+1} = S_j}{\lambda} \\ &=& \frac{\alpha_t(i) a_{ij} b_j(O_{t+1}) \beta_{t+1}(j)}{\myprob{O}{\lambda}}
\end{eqnarray*}

\subsubsection*{Baum-Welch Neubewertung}

\begin{eqnarray*}
\overline{a}_{ij} &=& \frac{\textrm{angenommene Anzahl an Übergängen von } S_i \textrm{ nach } S_j}{\textrm{angenommene Anzahl an Übergängen von } S_i} \\
&=& \frac{\sum_{t=1}^T \xi_t(i,j)}{\sum_{t=1}^T \sum_{j=0}^N \xi_t(i,j)} \\
\overline{b}_j(k) &=& \frac{\textrm{angenommene Dauer im Zustand } j \textrm{ mit Symbol } k}{\textrm{angenommene Dauer im Zustand } j} \\
&=& \frac{\sum_{t: O_t = k} \sum_{i=0}^N \xi_t(i,j)}{\sum_{t=1}^T \sum_{i=0}^N \xi_t(i,j)}
\end{eqnarray*}

\subsubsection*{Konvergenz des Forward-Backward Algorithmus}

\begin{enumerate}
\item initialisiere $\lambda = (A,B)$
\item berechne $\alpha$, $\beta$ und $\xi$
\item bewerte $\overline{\lambda} = (\overline{A},\overline{B})$ aus $\xi$
\item ersetze $\lambda$ durch $\overline{\lambda}$
\item wenn nicht konvergiert, gehe zu 2
\end{enumerate}
Es kann gezeigt werden, dass $\myprob{O}{\overline{\lambda}} > \myprob{O}{\lambda}$ gilt, wenn nicht $\lambda = \overline{\lambda}$ gilt.

\subsection{Sprachmodelle}

\subsubsection*{Bassierend auf die Grammatik}

\begin{itemize}
\item Bestimme Grammatik für mögliche Satzmuster
\item Vorteile: Lange Historie / Kontext; es wird keine große Textdatenbank benötigt
\item Problem: Grammatik zu schreiben ist sehr aufwendig; unflexibel: nur erstellte Muster können erkannt werden
\end{itemize}

\subsubsection*{N-Gram\index{N-Gram}}

\begin{itemize}
\item nächstes Wort wird anhand der Historie bestimmt
\item die Historie ist approximiert durch die letzten 2 oder 3 (allgemein $n$) Wörter
\item alles vor Wort $w_{i-n+1}$ wird in eine Äquivalenzklasse plaziert
\item schliesslich ist die Wahrscheinlichkeit für das nächste Wort gegeben durch
\begin{itemize}
\item Trigram: $\myprob{w_i}{w_{i-1} \, , \, w_{i-2}}$
\item Bigram: $\myprob{w_i}{w_{i-1}}$
\item Unigram: $P(w_i)$
\end{itemize}
\item Vorteile:
\begin{itemize}
\item trainierbar auf großen Textdatenbanken
\item "{}milde"{} Vorhersage
\item kann direkt kombiniert werden mit einem Akustikmodell
\end{itemize}
\item Problem: benötigt eine große Textdatenbank für jedes Gebiet
\end{itemize}

\subsubsection*{Objektive Bewertung der Qualität von Sprachmodellen}

Genau ein Sprachmodell ist besser als eine Alternative, wenn die Wahrscheinlichkeit $\hat{P}(w_1,w_2,\dots ,w_n)$ mit dem es den Großteil eines Tests $W$ erzeugen würde, größer ist. Aber:
$$\hat{P}(w_1,w_2,\dots ,w_n) = \prod\limits_{i=1}^n Q(w_i \, | \, \Psi(w_1,\dots,w_{i-1}))$$
ein gutes Qualitätsmaß ist das LOGPROB:
$$\hat{H}(W) = \frac{1}{n} \sum\limits_{i=1}^n \log_2 Q(w_i \, | \, \Psi(w_1,\dots,w_{i-1}))$$
Wenn Wörter einheitlich durch Zufall aus einem Vokabular der Größe $V$ von "{}Sprachmechanismen"{} erzeugt wurden, dann gilt
$$Q(w_i \, | \, \Psi(w_1,\dots,w_{i-1})) = \frac{1}{V}$$ und $$2^{\hat{H}(W)} = 2^{\log V} = V$$
Also definieren wir die Perplexität\index{Perplexität} eines Sprachmodells folgendermaßen: $$PP(W) = 2^{\hat{H}(W)}$$
und interpretieren sie als "{}Abzweigungsfaktor"{} der Sprache, wenn $\Psi$ gegeben ist.

\subsubsection*{Erfasste Erkennungsperformance}

Wortfehlerrate\index{Wortfehlerrate}

$$WER = \frac{\# Ins + \# Del + \# Sub}{N}$$




% !TeX root = summary.tex

\section{Maschinelles lernen}

\subsection{Unsupervised Learning}

\begin{itemize}
\item Datensammlung und Bezeichnung kostenaufwendig und zeitverschwenderisch.
\item Die Charaktermerkmale der Muster können sich ändern im Verlauf der Zeit.
\item Vielleicht hat man keine Einsicht in die Struktur der Daten.
\end{itemize}
$\Rightarrow$ Die Klassen sind \underline{nicht} bekannt.

\subsection{Mischdichten}

\begin{itemize}
\item Beispiele kommen von $c$ Klassen
\item A priori Wahrscheinlichkeit $P(\omega_j)$
\item Annahme: Die Formen der Klassenbedingten PDF (probability density function -- Wahrscheinlichkeitsdichtefunktion) $P(X | \omega_j , \theta_j)$ sind bekannt
\item Unbekannter Parametervektor $\theta_1, \dots, \theta_c$
\end{itemize}
Problem: Haarige Mathematik \\ Vereinfachung / Approximation \\ Betrachte nur Mittel $\to$ Isodaten
\begin{itemize}
\item Wähle initiale $\mu_1, \dots, \mu_c$
\item Klassifiziere $n$ Muster zu dem Mittel, das am nähesten liegt.
\item Berechne die Mittel neu aus den Mustern der Klasse
\item Mittel hat sich geändert? Gehe zu Schritt 2, sonst: stopp
\end{itemize}
Isodaten, Probleme:
\begin{itemize}
\item Die Wahl der initialen Mittel $\mu$.
\item Wissen über die Anzahl der Klassen.
\item Annahme über die Verteilung.
\item Was bedeutet "{}nah"{}?
\end{itemize}

\subsection{Clustering}

\begin{itemize}
\item Ähnlichkeit
\item Kriteriumsfunktion
\item Muster in der selben Klasse sollten die Kriteriumsfunktion noch extremer machen, dies erfasst die Cluserqualität.
\item Beispiel: Summe des Fehlerkriteriums: $$J = \sum\limits_{i=1}^c \sum\limits_{\max} || x - m_i ||^2$$
\end{itemize}

\subsection{Hierarchisches Clustering}

\begin{itemize}
\item $c$ muss nicht bestimmt werden
\item Mittel müssen nicht erraten werden
\item[(1)] Initialisiere $c := n$
\item[(2)] Finde die nähesten Paare von eindeutigen Clustern $X_i$ und $X_j$
\item[(3)] Vereinige sie und dekrementiere $c$
\item[(4)] Wenn $c < C_{stop}$: stoppen, sonst: gehe zu Schritt (2)
\end{itemize}

\section{Neuronale Netze}

\subsection{Wieso neuronale Netze?}

\begin{itemize}
\item Massiver Parallelismus
\item Massive Randbedingungsgenugtuung für "{}krank"{}-definierte Eingaben
\item Einfache Recheneinheiten
\item Viele Prozesseinheiten, viele Verbindungen
\item Einheitlichkeit ($\to$ Sensorverschmelzung)
\item Nichtlineare Klassifizierer / Abbilder ($\to$ gute Performance)
\item Lernfähig / anpassbar
\end{itemize}

\subsection{Entscheidungsfunktion $g(x)$}

\begin{eqnarray*}
g(\myvector{x}) > 0 &\Rightarrow& \textrm{Klasse A} \\ g(\myvector{x}) < 0 &\Rightarrow& \textrm{nicht Klasse A} \\ g(\myvector{x}) = 0 &\Rightarrow& \textrm{keine Entscheidung}
\end{eqnarray*}
$$g(\myvector{x}) = \sum\limits_{i=1}^n w_1x_1 + w_0 = \myvector{w}^T \myvector{x} + w_0$$
$\myvector{x} = (x_1,\dots,x_n)^T$ : Eigenschaftsvektor, $\myvector{w} = (w_1,\dots,w_n)^T$ : Gewichtungsvektor, $w_0$ : Schwellenwert

\subsection{Lineare Diskriminantenfunktion}

Hyperebe $H$: $g(\myvector{x}) = \sum\limits_{i=1}^n w_1x_1 + w_0 = \myvector{w}^T \myvector{x} + w_0 = 0$
$$\Rightarrow \myvector{x} = q \frac{\myvector{w}}{|| \myvector{w} ||} + r \frac{\myvector{w}}{|| \myvector{w} ||} + \myvector{x}_p$$
Vektor $q \frac{\myvector{w}}{|| \myvector{w} ||}$ entspricht: $g \left(q \frac{\myvector{w}}{|| \myvector{w} ||} \right) = 0 = q ||\myvector{w}|| + w_0$ $$\Rightarrow q = - \frac{w_0}{|| \myvector{w} ||}$$
Und mit $g(\myvector{x}) = \myvector{w}^T q \frac{\myvector{w}}{|| \myvector{w} ||} + \myvector{w}^T r \frac{\myvector{w}}{|| \myvector{w} ||} + \myvector{w} \myvector{x}_p + w_0 = -w_0 + r ||\myvector{w}|| + w_0$ erhalten wir $$r = \frac{g(\myvector{x})}{|| \myvector{w} ||}$$

\subsection{Fisher-lineare Diskriminante}

\begin{itemize}
\item Dimensionsreduktion
\item Projeziert eine Menge von mehrdimensionalen Punkten auf eine Line $y = \myvector{w} \myvector{x}$
\item Die Fisher Diskriminaten ist eine Funktion, die folgendes Kriterium maximiert $$g(x) = \frac{| \tilde{m}_1 - \tilde{m_2}|}{\tilde{s}_1 + \tilde{s}_2}$$
wobei $\tilde{m}_i = \frac{1}{n} \sum\limits_{y \in Y_i} y$ : Mittel für projezierte Muster, $\tilde{s}_i^2 = \sum\limits_{y \in Y_i} (y - \tilde{m}_i)^2$ : Streuung für projezierte Muster
\item Fisher's lineare Diskriminante:
\begin{eqnarray*}
\myvector{w} &=& s_w^{-1} (\myvector{m}_1 - \myvector{m}_2) \\ s_w &=& s_1 + s_2 \\ s_i &=& \sum\limits_{x \in X_i} (\myvector{x} - \myvector{m}_i)(\myvector{x} - \myvector{m}_i)^T
\end{eqnarray*}
\end{itemize}

\subsection{Das Perzeptron}

\mypic{10}{perzeptron}

\subsubsection*{Lineare Diskriminantenfunktionen}

\begin{itemize}
\item Keine Annahme über die Verteilung (Nicht-parametrisch)
\item Lineare Entscheidungsflächen
\item Start durch überwachtes Training (Klassen der Trainingsdaten gegeben)
\item Diskriminantenfunktion: $$g(x) = w_0 + \sum\limits_{i=1}^n w_ix_i = \sum\limits_{i=0}^n w_ix_i, \quad x_0 = 1$$
\item $g(x)$ ergibt die Distanz von der Entscheidungsfläche
\item Zwei Kategorien Fall:
\begin{eqnarray*}
g_1(x) > 0 &\Rightarrow& \textrm{Klasse 1} \\ g_1(x) < 0 &\Rightarrow& \textrm{Klasse 2}
\end{eqnarray*}
\end{itemize}

\subsubsection*{Perzeptron}

$$g(x) = \sum\limits_{i=0}^n w_i x_i \quad x_0 = 1 \qquad \to \qquad \textrm{finde } w$$
Alle Vektoren $x_i$ sind korrekt benannt
\begin{eqnarray*}
\myvector{w}\myvector{x}_i > 0 && \textrm{ wenn } x_i \textrm{ zu } \omega_1 \textrm{ gehört} \\ \myvector{w}\myvector{x}_i < 0 && \textrm{ wenn } x_i \textrm{ zu } \omega_2 \textrm{ gehört}
\end{eqnarray*}
Oder setze alle Muster, die zu $\omega_2$ gehören auf ihr negatives ($-\myvector{x}_i$). Dann sind alle Vektoren korrekt klassifiziert, wenn $\myvector{w} \myvector{x}_i > 0$ für alle $i$.
\subsubsection*{Kriteriumsfunktion vom Perzeptron}
$$J_P(\myvector{w}) = \sum\limits_{x \in X} (- \myvector{w} \myvector{x})$$
$X$ ist die Menge der falsch klassifizierten Merkmale. Da $\myvector{w} \myvector{x}$ die Negation für falsch klassifizierte Merkmale ist, ist $J_P(\myvector{w})$ positiv. Sobald $J_P = 0$ gilt, ist ein Lösungsvektor gefunden. \\ $J_P$ ist proportional zu der Summe der Distanzen der falsch klassifizierten Mustern zur Entscheidungsgrenze.
$$\nabla J_P = \sum\limits_{x \in X} (- \myvector{x}) \qquad \myvector{w}_{k+1} = \myvector{w}_k + \zeta_k \sum\limits_{x \in X} \myvector{x}$$

\subsubsection*{Lernen des Perzeptron}

Fragen:
\begin{itemize}
\item Wie soll man die Lernrate setzen?
\item Wie soll man die initialen Gewichte setzen?
\end{itemize}
Probleme:
\begin{itemize}
\item Untrennbare Daten
\item Trennbare Daten, aber welche Entscheidungsfläche
\item Nicht linear trennbar
\end{itemize}

\subsubsection*{Variationen}

Entspannungsprozedur: $$J_q(\myvector{w}) = \sum\limits_{x \in X} (\myvector{w}^t \myvector{y})^2$$
Der Gradient ist kontinuierlich $\to$ gleichmäßige Fläche. Manchmal geht es gegen Null!
$$J_q(\myvector{w}) = \frac{1}{2} \sum\limits_{x \in X} \frac{(\myvector{x}^t\myvector{w} - b)^2}{||x||^2} \textrm{ begrenzt durch } b$$

\subsection{Generalisierung}

\begin{itemize}
\item Leistung auf Trainingsdaten interessiert uns nicht, sondern ungesehene real Welt\dots
\item Wie gut funktioniert mein System in unvorhergesehenen Situationen?
\end{itemize}

\subsubsection*{Generalisierungsfähigkeit}

Fähigkeit, das aus den Trainingsdaten Erlernte auf neue Daten (Testdaten) anzuwenden.

\subsubsection*{Drei Gründe für schlechte Generalisierung}
\begin{enumerate}
\item Overfitting / Overtraining (zu lange trainiert)
\item Zu viele Parameter (weights) bzw. zu wenig Trainingsmaterial
\item falsche Netzwerkstruktur
\end{enumerate}
\subsubsection*{Methoden zur Verbesserung der Generalisierung}

\begin{enumerate}
\item Training am Besten Punkt abbrechen
\begin{itemize}
\item Aufteilung der Daten in: Trainings-, Crossvalidation- und Test-Menge
\item Wichtig: Testdaten von Trainingsdaten trennen! Wiederholtes Testen ist leichtes Trainieren (Tuning) $\to$ Crossvalidation Set
\end{itemize}
\item Reduzierung der Komplexität des Netzwerkes durch Regularisierung: Weight Decay, Weight Elimination, Optimal Brain Damage, Optimal Brain Surgeon
\item Schrittweises Vergrößern eines zu kleinen Netzes (konstruktiv): Cascade Correlation, Meiosis Netzwerke, ASO (Automativ Structure Optimization)
\end{enumerate}















% !TeX root = summary.tex

\chapter{3D-Bildverarbeitung}



\section{Geometrische 3D-Transformationen}

Grundlage von Sensorik und Aktorik: Beschreibung von Objektposen (Position, Rotation)
\begin{itemize}
\item Pose des messenden Systems im Raum
\item Pose mehrerer Sensoren zueinander
\item Pose sensorisch erfasster Objekte relativ zum Sensor
\item Pose von Aktoren (Greifer, Lötlampen etc.) im Raum und relativ zum manipulierten Objekt
\item Gelenkwinkelstellungen im Roboterarm
\end{itemize}
Anforderungen:
\begin{itemize}
\item geschlossene Ausdrücke
\item Invertierbarkeit
\item Interpolation
\end{itemize}
Zwei Systeme haben sich durchgesetzt:
\begin{itemize}
\item Homogene Geometrie (für Translationen und Rotationen)
\item Quaternionendarstellung (nur für Rotationen)
\end{itemize}

\subsection{Translation}\index{Translation}

Translation eines 3D-Vektors: $$\myvecthree{x_0}{y_0}{z_0} + \myvecthree{x}{y}{z} = \myvecthree{x_0 + x}{y_0 + y}{z_0 + z}$$

\subsection{Rotation}\index{Rotation}

\begin{itemize}
\item o.B.d.A. auf Einheitsvektor zurückführbar (Basistransformation)
\item Konvention: Rechtskoordinatensystem
\item Rotation von $(x_0,y_0,z_0)$ um Winkel $\beta$ mit Ergebnis $(x,y,z_0)$
\end{itemize}
Aus Additionstheorem:
\begin{eqnarray*}
x &=& \cos(\alpha + \beta) = \cos(\alpha) \cos(\beta) - \sin(\alpha) \sin(\beta) \\ y &=& \cos(\alpha + \beta) = \sin(\beta) \cos(\alpha) + \cos(\beta) \sin(\alpha)
\end{eqnarray*}
und mit $(x_0,y_0) = (\cos(\alpha), \sin(\alpha))$:
$$\left( \begin{array}{c} x \\ y \end{array} \right) = \left( \begin{array}{rr} \cos(\beta) & -\sin(\beta) \\ \sin(\beta) & \cos(\beta) \end{array} \right) \left( \begin{array}{c} x_0 \\ y_0 \end{array} \right)$$
$z_0$ invariant, da Rotation um $z$!

\subsubsection{Rotationsmatrix}\index{Rotationsmatrix}

Rotationsmatrix allgemein: $$\myvecthree{x}{y}{z} = R \myvecthree{x_0}{y_0}{z_0}$$
Rotation um $x$: $$R_x(\theta) = \myvecnine{1}{0}{0}{0}{\cos(\theta)}{- \sin(\theta)}{0}{\sin(\theta)}{\cos(\theta)}$$
Rotation um $y$: $$R_y(\theta) = \myvecnine{\cos(\theta)}{0}{\sin(\theta)}{0}{1}{0}{- \sin(\theta)}{0}{\cos(\theta)}$$
Rotation um $z$: $$R_z(\theta) = \myvecnine{\cos(\theta)}{- \sin(\theta)}{0}{\sin(\theta)}{\cos(\theta)}{0}{0}{0}{1}$$

\textbf{\textsl{Eigenschaften von Rotationsmatrizen:}}
\begin{itemize}
\item regulär, invertierbar, Determinante $=1$
\item jede beliebige Rotation im Raum kann durch drei Variablen beschrieben werden (Eulers Theorem)
\item Einzelrotationen können als eine Matrix dargestelt werden: $$R_r(\gamma) R_q(\beta) R_p(\alpha) \myvecthree{x}{y}{z} \quad \textrm{mit} \quad p,q,r \in \{x,y,z\} = R_{pqr}(\alpha, \beta, \gamma) \myvecthree{x}{y}{z}$$
\item damit reicht Angabe von $\alpha$, $\beta$, $\gamma$ zur Beschreibung der Rotation
\item das macht natürlich nur Sinn, wenn eine Konvention für die Zuordnung $p$, $q$, $r$ zu den Achsen $x$, $y$, $z$ definiert wurde
\end{itemize}
Zwei grundlegend unterschiedliche Rotationstypen:
\begin{itemize}
\item Rotation um mitgedrehte Achsen (Euler-Winkel\index{Euler-Winkel})
\item Rotation um raumfeste Achsen (Roll Pitch Yaw\index{Roll Pitch Yaw})
\end{itemize}
Konventionen zur Erstellung von Rotationsmatrizen:
\begin{itemize}
\item Standard-Beispiel für Euler-Winkel: \\ Zuerst um die $x$-Achse, dann um die mitgedrehte $y$-Achse, dann um die (zweimal) mitgedrehte $z$-Achse $$R_{X'Y'Z'}(\alpha , \beta , \gamma) = R_X(\alpha) R_Y(\beta) R_Z(\gamma)$$
\item Standard-Beispiel für raumfeste Achsen: \\ Zuerst um die raumfeste $z$-Achse, dann um die raumfeste $y$-Achse, dann um die raumfeste $x$-Achse. $$R_{ZYX}(\gamma, \beta, \alpha)= R_Z(\alpha) R_Y(\beta) R_X(\gamma)$$
\end{itemize}

\subsection{Homogene 3D-Transformation}

Transformation definiert durch Rotation $R$ und Translation $t$:
$$\myvecthree{x}{y}{z} = R \myvecthree{x_0}{y_0}{z_0} + t = \myvecnine{r_{11}}{r_{12}}{r_{13}}{r_{21}}{r_{22}}{r_{23}}{r_{31}}{r_{32}}{r_{33}} \myvecthree{x_0}{y_0}{z_0} + \myvecthree{t_x}{t_y}{t_z}$$
Darstellung mit Hilfe homogener Koordinaten und einer geschlossenen Transformationsmatrix:
$$\myvecqfour{x}{y}{z}{1} = \left( \begin{array}{ccc|c} &&& \\ & R && t \\ &&& \\ \hline 0 & 0 & 0 & 1 \end{array} \right) \myvecqfour{x_0}{y_0}{z_0}{1} = \left( \begin{array}{ccc|c} r_{11} & r_{12} & r_{13} & t_x \\ r_{21} & r_{22} & r_{23} & t_y \\ r_{31} & r_{32} & r_{33} & t_z \\ \hline 0 & 0 & 0 & 1 \end{array} \right) \myvecqfour{x_0}{y_0}{z_0}{1} = A \myvecqfour{x_0}{y_0}{z_0}{1}$$

\textbf{\textsl{Probleme mit Rotationsmatrizen:}}
\begin{itemize}
\item hoch redundant
\item rechenaufwendig
\item Interpolation schwierig
\item Euler-Winkel: Singularitäten
\end{itemize}

\subsection{Quaternionen\index{Quaternionen}}

Erweiterung der komplexen zahlen ins vierdimensionale. Definition: \\[0,1cm]
Ein Quaternion $\textbf{q}$ ist eine Zahl
\begin{eqnarray*}
\quaternion &=& (q_x,q_y,q_z,q_w)(i,j,k,1)^T \\ &=& (\quaternion_v,q_w)(i,j,k,1)^T \\ &=& iq_x + jq_y + kq_z + q_w
\end{eqnarray*}
mit
\begin{eqnarray*}
i^2 &=& j^2 = k^2 = -1 \\ ij &=& -ji = k \\ jk &=& -kj = i \\ ki &=& -ik = j
\end{eqnarray*}
$q_w$ ist der Realteil, $\quaternion_v = (q_x,q_y,q_z)$ der Imaginärteil des Quaternions. Man schreibt einfach $(q_x,q_y,q_z,q_w)$ oder $(\quaternion_v,q_w)$. \\[0,1cm]

\subsubsection{Rechenregeln für Quaternionen}

\begin{itemize}
	\item Addition: \\ 
	\begin{center}
	$\mathbf{q}+\mathbf{r}=(q_w,\mathbf{q}_v)+(r_w,\mathbf{r}_v)=(q_w+r_w,\mathbf{q}_v+\mathbf{r}_v)$
	\end{center}
	\item Multiplikation: \\
	\begin{center}
	$\mathbf{q} \mathbf{r}=(q_w r_w-\mathbf{q}_v \cdot \mathbf{r}_v,\mathbf{q}_v \times \mathbf{r}_v + q_w \mathbf{r}_v + r_w \mathbf{q}_v)$
	\end{center}
	assoziativ, aber nicht kommutativ
	\item Konjugierter Quaternion:
		\begin{center}
	$\overline{\mathbf{q}}=(q_w,-\mathbf{q}_v)$ für $\mathbf{q}=(q_w,\mathbf{q}_v)$
	\end{center}
	\item Norm: 
	\begin{center}
	$N(\quaternion) = \sqrt{\quaternion \overline{\quaternion}} = \sqrt{\overline{\quaternion}\quaternion} = \sqrt{q_x^2 + q_y^2 + q_z^2 + q_w^2}$
	\end{center}
	Quaternionen $\quaternion$ mit $N(\quaternion) = 1$ heißen Einheitsquaternionen\index{Einheitsquaternionen}
	\item  Multiplikative Identität: 
	\begin{center}
	$I = (0,1)$
	\end{center}
	\item Multiplikative Inverse:
	\begin{center}
	$\quaternion^{-1} = \frac{\overline{\quaternion}}{N^2(\quaternion)}$
	\end{center}
\end{itemize}


 
\subsubsection{Rotation mit Quaternionen}
\begin{itemize}
\item Einheitsquaternion $\quaternion$ ist definiert durch Rotationsachse $u$ mit $|u| = 1$ und Winkel $\theta$: $$\quaternion = \left( \cos \frac{\theta}{2} \, , \, u \sin \frac{\theta}{2} \right)$$
\item Quaternion \textbf{a} ist definiert durch zu rotierenden Vektor $v$: $$\textbf{a} = (v \, , \, 0)$$
\item Das Produkt $\textbf{qa}\overline{\textbf{q}}$ rotiert $v$ um die Achse $u$ mit dem Winkel $\theta$.
\end{itemize}


\subsubsection{Interpolation zwischen zwei Quaternionen}
\begin{itemize}
\item Sphärische Lineare Interpolation (SLERP)
\item Berechnet für $t \in [0,1]$ die kürzeste Verbindung auf der vierdimensionalen Einheitssphäre zwischen $q$ und $r$.
\item Analytisch: $$SLERP(q,r,t) = q(rq^{-1})^t$$
\item Numerisch: $$SLERP(q,r,t) = q \frac{\sin((1-t)\theta)}{\sin(\theta)} + r \frac{\sin(t \theta)}{\sin(\theta)}$$ mit Winkel $\theta$ zwischen $r$ und $q$.
\end{itemize}

\subsubsection{Umrechnung}
Quaternion $\to$ Rotationsmatrix: $$\quaternion = (q_x,q_y,q_z,q_w) \Rightarrow M_q = \myvecnine{1-2(q_y^2 + q_z^2)}{2(q_xq_y - q_wq_z)}{2(q_xq_z + q_wq_y)}{2(q_xq_y + q_wq_z)}{1-2(q_x^2 + q_z^2)}{2(q_yq_z - q_wq_x)}{2(q_xq_z - q_wq_y)}{2(q_yq_z + q_wq_x)}{1-2(q_x^2 + q_y^2)}$$
Rotationsmatrix $\to$ Quaternion:
\begin{eqnarray*}
q_w &=& \frac{1}{2} \sqrt{1 + \sum\limits_{i=1}^3 m_{ii}} \\ q_x &=& \frac{(m_{32} - m_{23})}{4q_w} \\ q_y &=& \frac{(m_{13} - m_{31})}{4q_w} \\ q_z &=& \frac{(m_{21} - m_{12})}{4q_w}
\end{eqnarray*}

\subsubsection{Vor- und Nachteile}
Vorteile:
\begin{itemize}
\item Rotation direkt um gewünschte Drehachse
\item Interpolation möglich
\item weniger Rechenaufwand
\item keine Redundanz $\Rightarrow$ numerisch stabiler, weniger Gefahr für Singularitäten
\end{itemize}
Nachteil:
\begin{itemize}
\item nur Rotation berechenbar $\Rightarrow$ Kombination mit Matrizen nötig $\Rightarrow$ Rechenaufwand für Umwandlungen
\end{itemize}

\section{Erweitertes Kameramodell}

Lochkameramodell vereinfacht die realen Verhältnisse stark. Deshalb werden in der Praxis Erweiterungen des Lochkameramodells verwendet. Zunächst einige Definitionen:
\begin{description}
\item[Optische Achse:] Gerade durch das Projektionszentrum, senkrecht zur Bildebene
\item[Bildhauptpunkt $C(c_x \, , \, c_y)$:] Schnittpunkt der optischen Achse mit der Bildebene
\end{description}
Koordinatensysteme:
\begin{description}
\item[Bildkoordinatensystem:\index{Bildkoordinatensystem}] 2D-Koordinatensystem, Einheit [Pixel], Vereinbarung für die Vorlesung (gilt für die meisten Kameratreiber): Ursprung in der linken oberen Ecke des Bildes, $u$-Achse zeigt nach rechts, $v$-Achse zeigt nach unten.
\item[Kamerakoordinatensystem:\index{Kamerakoordinatensystem}] 3D-Koordinatensystem, Einheit [mm], Ursprung liegt im Projektionszentrum, Achsen parallel zu den Achsen des Bildkoordinatensystems, d.h. $x$-Achse nach rechts, $y$-Achse nach unten und die $z$-Achse gemäß der Dreifingerregel für ein rechtshändiges Koordinatensystem nach vorne.
\item[Weltkoordinatensystem:\index{Weltkoordinatensystem}] 3D-Koordinatensystem, Einheit [mm], Basiskoordinatensystem, das beliebig im Raum liegen kann.
\end{description}
Begriffe:
\begin{description}
\item[Intrinsische Kameraparameter:] Brennweite, Bildhauptpunkt, Parameter für die Beschreibung radialer/tangentialer Linsenverzerrung; definieren die nicht (eindeutig) umkehrbare Abbildung vom Kamerakoordinatensystem in das Bildkoordinatensystem.
\item[Extrinsische Kameraparameter:] Definieren die Transformation vom Kamerakoordinatensystem in das Weltkoordinatensystem, im Allgemeinen durch eine Rotation $R$ und eine Translation $t$.
\end{description}
Vereinfachungen des Lochkameramodells:
\begin{itemize}
\item Ursprung des Bildkoordinatensystems ist identisch mit dem Bildhauptpunkt
\item Pixel werden als quadratisch angenommen
\item keinerlei Modellierung der Linsenverzerrung
\item es existiert kein Weltkoordinatensystem bzw. es ist identisch mit dem Kamerakoordinatensystem, d.h. es werden keine extrinsischen Kameraparameter modelliert
\end{itemize}
Brennweite:
\begin{itemize}
\item In der Praxis wird die Umrechnung von [mm] nach [Pixel] in den/die Parameter für Brennweite mit aufgenommen.
\item Da Pixel nicht mehr als quadratisch sondern als rechteckig angenommen werden, gibt es deshalb für jede Richtung einen Parameter, also: $f_x$, $f_y$.
\item Die Parameter $f_x$, $f_y$ sind dann das Produkt aus der tatsächlichen Brennweite mit Einheit [mm] und dem jeweiligen Umrechnungsfaktor mit Einheit [Pixel/mm].
\item Die Einheit für die Parameter $f_x$, $f_y$ ist somit [Pixel].
\end{itemize}
Die Abbildung vom Kamerakoordinatensystem in das Bildkoordinatensystem, ausschließlich mit den intrinsischen Parametern, ist dann definiert durch: $$\myvectwo{u}{v} = \myvectwo{c_x}{c_y} + \frac{1}{Z} \cdot \myvectwo{f_x \cdot X}{f_y \cdot Y}$$ oder als Matrixmultiplikation mit Kalibriermatrix $K$ auf homogenen Koordinaten: $$\myvecthree{u \cdot w}{v \cdot w}{w} = K \cdot \myvecthree{X}{Y}{Z} \qquad K = \myvecnine{f_x}{0}{c_x}{0}{f_y}{c_y}{0}{0}{1}$$
Extrinsische Kamerakalibrierung:
\begin{itemize}
\item Ist definiert durch eine Koordinatentransformation bestehend aus Rotation und Translation.
\item Koordinatentransformation vom Weltkoordinatensystem in das Kamerakoordinatensystem: $$x_c = Rx_w + t$$
Koordinatentransformation vom Kamerakoordinatensystem in das Weltkoordinatensystem: $$x_w = R^T x_c - R^T t$$
\item $3 \times 4$ Gesamt-Projektionsmatrix\index{Projektionsmatrix} $P$ (intrinsisch und extrinsisch) auf homogenen Koordinaten: $$\myvecthree{u \cdot w}{v \cdot w}{w} = P \cdot \myvecqfour{X}{Y}{Z}{1} \qquad P = (K \, R \, | \, K \, t)$$
\end{itemize}

\section{Kamerakalibrierung\index{Kamerakalibrierung}}

Die Kalibrierung einer Kamera bedeutet die Bestimmung ihrer Parameter bezüglich eines gewählten Kameramodells. Die Bestimmung der intrinsischen Parameter ist unabhängig vom Aufbau; solange Zoom und Fokus der Kamera gleich bleiben, verändern sich diese Parameter nicht. Die Bestimmung der extrinsischen Parameter ist abhängig von der Wahl des Weltkoordinatensystems und ändert sich je nach Aufbau. \\
Ist die Kamera kalibriert, dann liegt die Abbildungsfunktion $f$ vor, die einen Punkt vom Weltkoordinatensystem eindeutig in das Bildkoordinatensystem abbildet: $$f \, : \, R^3 \to R^2$$
$f$ ist definiert durch die Projektionsmatrix $P$ und anschließender Transformation der homogenen Koordinaten durch Division durch $w$. Die Inverse Abbildung bildet einen Punkt im Bildkoordinatensystem auf eine Gerade im Weltkoordinatensystem ab, die durch das Projektionszentrum verläuft. \\[0,1cm]
Verfahren zur Kamerakalibrierung:
\begin{itemize}
\item Direkte Lineare Transformation (DLT)
\item Erweiterungen der DLT, welche Linsenverzerrung modellieren
\end{itemize}
gesucht: $3 \times 4$-Matrix, hat also 12 Unbekannte; Verfahren Testfeldkalibrierung:\index{Testfeldkalibrierung}
\begin{itemize}
\item Bestimmung einer Menge von Punktkorrespondenzen: 3D-Punkt in einem gewählten Weltkoordinatensystem und 2D-Punkt im Bildkoordinatensystem
\item 3D-Punkte sind durch Verwendung eines geeigneten Kalibrierobjekts oder -musters a-priori bekannt
\item 2D-Punkte werden durch Methoden der Bildverarbeitung berechnet
\end{itemize}
benötigt: 6 bekannte Objektpunkte, da jede Punktkorrespondenz zwei Gleichungen liefert \\
Bedingung: 3D-Punkte dürfen nicht koplanar liegen, d.h. sie müssen einen dreidimensionalen Raum aufspannen \\
Möglichkeiten:
\begin{itemize}
\item Verwendung eines 2D-Musters, das in mindestens zwei verschiedenen Tiefen präsentiert wird.
\item Verwendung eines geeigneten 3D-Kalibrierobjekts.
\end{itemize}

\subsection{Direkte Lineare Transformation}

Ein Standard-Verfahren für die Berechnung der Projektionsmatrix $P$ ist die Direkte Lineare Transformation (DLT)
$$\myvecthree{u \cdot w}{v \cdot w}{w} = P \cdot \myvecqfour{X}{Y}{Z}{1} \qquad P = (K \, R \, | \, K \, t) = \left( \begin{array}{cccc} p_1 & p_2 & p_3 & p_4 \\ p_5 & p_6 & p_7 & p_8 \\ p_9 & p_{10} & p_{11} & p_{12} \end{array} \right)$$
\begin{eqnarray*}
\Rightarrow u &=& \frac{p_1X + p_2Y + p_3Z + p_4}{p_9X + p_{10}Y + p_{11}Z + p_{12}} \\
v &=& \frac{p_5X + p_6Y + p_7Z + p_8}{p_9X + p_{10}Y + p_{11}Z + p_{12}}
\end{eqnarray*}
o.B.d.A. kann ein Parameter normiert werden. Üblicherweise wird $p_{12} = 1$ gewählt.
\begin{eqnarray*}
p_1X + p_2Y + p_3Z + p_4 &=& up_9X + up_{10}Y + up_{11}Z + u \\
p_5X + p_6Y + p_7Z + p_8 &=& vp_9X + vp_{10}Y + vp_{11}Z + v
\end{eqnarray*}
Formuliert als überbestimmtes LGS $Ax = b$ mit $n \geq 6$ Punktkorrespondenzen, das beispielsweise mit Hilfe der Normalengleichung gelöst werden kann:
$$A = \left( \begin{array}{ccccccccccc} X_1 & Y_1 & Z_1 & 1 & 0 & 0 & 0 & 0 & -u_1X_1 & -u_1Y_1 & -u_1Z_1 \\ 0 & 0 & 0 & 0 & X_1 & Y_1 & Z_1 & 1 & -v_1X_1 & -v_1Y_1 & -v_1Z_1 \\ \vdots & \vdots & \vdots & \vdots & \vdots & \vdots & \vdots & \vdots & \vdots & \vdots & \vdots \\ X_n & Y_n & Z_n & 1 & 0 & 0 & 0 & 0 & -u_nX_n & -u_nY_n & -u_nZ_n \\ 0 & 0 & 0 & 0 & X_n & Y_n & Z_n & 1 & -v_nX_n & -v_nY_n & -v_nZ_n \end{array} \right) \quad x = \myvecthree{p_1}{\vdots}{p_{11}} \quad b = \left( \begin{array}{c} u_1 \\ v_1 \\ \vdots \\ u_n \\ v_n \end{array} \right)$$

\section{Stereokonstruktion}

Gegeben:
\begin{itemize}
\item zwei Kameras (durch ihre Zentren $C$ und $C'$) mit Projektionsmatrizen $P$ und $P'$
\item zwei Abbilder $x$ und $x'$ des Punktes $X$
\item dann kann $X$ rekonstruiert werden
\end{itemize}
Triangulation zwischen linker und rechter Kamera möglich durch Kenntnis der Kameraparameter. Eine Möglichkeit zur Berechnung von 3D-Punkten aus Bildpunkt-Korrespndenzen $x$, $x'$:
\begin{itemize}
\item Aufstellen der beiden Geraden $g$, $g'$ der möglichen Punkte zu $x$, $x'$ im Weltkoordinatensystem mit Hilfe der Projektionsmatrizen $P$, $P'$:
\begin{eqnarray*}
g \, : \, x &=& a + r \cdot u \\ g' \, : \, x &=& b + s \cdot v
\end{eqnarray*}
\item Berechnung des optimalen "{}Schnittpunktes"{} $S$ durch Lösung des überbestimmten LGS $Ax = c$ mit: $$A = \left( \begin{array}{cc} u_1 & -v_1 \\ u_2 & -v_2 \\ u_3 & -v_3 \end{array} \right) \quad , \quad x = \myvectwo{r'}{s'} \quad , \quad c = b-a \qquad s = \frac{a + r' \cdot u + b + s' \cdot v}{2}$$
\end{itemize}


\subsection{Epipolargeometrie\index{Epipolargeometrie}}

\mypic{8}{kameramodell}

Zusammenhang zwischen zwei Kameras ist gegeben durch die Epipolargeometrie. Die Schnittpunkte $e$ und $e'$ der Geraden durch die Projektionszentren mit den Bildebenen nennt man Epipole\index{Epipol}.
\begin{description}
\item[Epipolarebene $\pi(X)$:] Ebene, die durch $C$, $C'$ und Szenenpunkt $X$ aufgespannt wird.
\item[Epipolarlinie $l'(x)$:] Schnittgerade von $\pi(X)$ mit Bildebene.
\end{description}
Alle Punkte $X$, die auf $x$ in Kamera 1 abgebildet werden, werden auf einen Punkt der Linie $l'(x)$ in Kamera 2 abgebildet. Alle Epipolarlinien eines Kamerasystems schneiden sich in den Epipolen $e$ und $e'$. \\
\textsl{Nutzen:} Einschränkung des Korrespondenzproblems von zwei Dimensionen auf eine Dimension, da nach entsprechenden Merkmalen nur noch entlang der Epipolarlinie gesucht werden muss:
\begin{itemize}
\item höhere Robustheit (weniger falsche Korrespondenzen)
\item höhere Effizienz
\end{itemize}

\subsection{Fundamentalmatrix\index{Fundamentalmatrix}}

Mathematische Beschreibung der Epipolargeometrie erfolgt durch die Fundamentalmatrix. Eigenschaften der Fundamentalmatrix $F$:
\begin{itemize}
\item $3 \times 3$-Matrix
\item Rang 2
\item für alle Korrespondenzen $x$, $x'$ gilt: $$x'^TFx = 0$$ $x$ und $x'$ sind Bildpunkte in homogenen Koordinaten mit $w = 1$
\end{itemize}
Mit der Fundamentalmatrix lassen sich die Epipolarlinien berechnen: $$l = F^Tx' \qquad \textrm{und} \qquad l' = Fx$$ Für die Epipole gilt: $$Fe = 0 \qquad \textrm{und} \qquad F^Te' = 0$$ Hinweis: $l$ (bzw. $l'$) definieren eine 2D-Gerade wie folgt: \\ $lx = 0$ für alle Bildpunkte $x$ (in homogenen Koordinaten mit $w = 1$), die auf dieser Geraden liegen. \\
Die Fundamentalmatrix lässt sich auf mehrere Arten berechnen:
\begin{itemize}
\item über Bildpunkt-Korrespondenzen in der linken und rechten Kamera
\item bei bekannter intrinsischer und extrinsischer Kalibrierung der Kameras direkt über die Kalibriermatrizen $K$, $K'$ und der Essentialmatrix $E$, die durch die extrinsischen Parameter definiert ist
\end{itemize}
\textbf{\textsl{Berechnung über Bildpunkt-Korrespondenzen:}}
$$x'^T Fx = 0 \quad , \quad x' = (x',y',z') \quad , \quad x = (x,y,z)$$
\begin{eqnarray*}
\Rightarrow \qquad \qquad x'x f_{11} + x'y f_{12} + x'f_{13} && \\ + \,\, y'x'f_{21} + y'yf_{22} + y'f_{23} && \\ + \,\, xf_{31} + yf_{32} + f_{33} &=& 0
\end{eqnarray*}
Für $n \geq 7$ Korrespondenzen $x$, $x'$:
$$\underbrace{\left( \begin{array}{ccccccccc} x_1'x_1 & x_1'y_1 & x_1' & y_1'x_1 & y_1'y_1 & y_1' & x_1 & y_1 & 1 \\ \vdots & \vdots & \vdots & \vdots & \vdots & \vdots & \vdots & \vdots & \vdots \\ x_n'x_n & x_n'y_n & x_n' & y_n'x_n & y_n'y_n & y_n' & x_n & y_n & 1 \end{array} \right)}_{A} \underbrace{\myvecqfour{f_{11}}{f_{12}}{\vdots}{f_{33}}}_{f} = 0$$
$Af=0$ lösen z.B. mit Singulärwertzerlegung (SVD) \\
\textbf{\textsl{Berechnung über Essentialmatrix:}} \\
Essentialmatrix\index{Essentialmatrix} lässt sich durch die extrinsischen Parameter berechnen: \\
gegeben:
\begin{itemize}
\item Kamera 1 mit $(I \, | \, 0)$ als Transformation (Identität)
\item Kamera 2 mit $(R \, | \, t)$ als Transformation
\end{itemize}
Essentialmatrix $E$ lässt sich berechnen zu: $$E = [t]_xR = \myvecnineright{0}{-t_3}{t_2}{t_3}{0}{-t_1}{-t_2}{t_1}{0}$$ Für die Epipole gilt: $$e = -KR^Tt \quad \textrm{und} \quad e' = K't$$
Hat man die Essentialmatrix (z.B.über die extrinsischen Parameter) berechnet und die intrinsischen Parameter, d.h. Kalibriermatrizen $K$, $K'$, so lässt sich die Fundamentalmatrix berechnen zu: $$F = K'^{-T}EK^{-1}$$
Hat man umgekehrt die Fundamentalmatrix bestimmt (z.B. über Bildpunkt-Korrespondenzen) und die intrinsischen Parameter, d.h. die Kalibriermatrizen $K$, $K'$, so lässt sich die Essentialmatrix berechnen zu: $$E = K'^TFK$$

\subsubsection*{Stereo-Sehen}

Weiter Eigenschaften der Fundamentalmatrix:
\begin{itemize}
\item Mit ihr lassen sich die Eingabebilder rektifizieren.
\begin{itemize}
\item Nach Rektifizierung verlaufen alle Epipolarlinien horizontal mit derselben $v$-Koordinate wie der Bildpunkt im anderen Kamerabild.
\item Nach Korrespondenzen muss nur noch horizontal (in eine Richtung) gesucht werden.
\end{itemize}
\item Mit Hilfe der Essentialmatrix lassen sich die Projektionsmatrizen bis auf Skalierung genau rekonstruieren, mit Hilfe der Fundametalmatrix bis auf Skalierung und Projektion genau.
\end{itemize}
Rektifizierte Bilder haben den Vorteil, dass sich optimierte Korrelations-Algorithmen für die Lösung des Korrespondenzproblems verwenden lassen. $\Rightarrow$ Laufzeit unabhängig von der Fenstergröße \\
Nachteile:
\begin{itemize}
\item Interpolation notwendig für die Berechnung der rektifizierten Bilder $\Rightarrow$ Qualitätsverlust
\item Bilder je nach Aufbau stark verzerrt
\end{itemize}
Nach Lösung des Korrespondenzproblems können
\begin{itemize}
\item Punktwolken berechnet werden durch Triangulation, wie zuvor erläutert
\item Tiefenbilder erzeugt werden durch Eintrag der Disparitäten (Differenz der $u$-Koordinaten für gefundene Korrespondenzen in den rektifizierten Bildern) in ein Graustufenbild: \\ $\Rightarrow$ Je höher der Grauwert, desto näher befindet sich der entsprechende 3D-Punkt zur Kamera
\end{itemize}



\section{Fragen zum Kapitel}
\subsection{Geometrische 3D-Transformationen}
\begin{enumerate}
% Selbst ausgedacht:
\item Welche zwei Vorgehensweisen haben sich für die geometrische 3D-Transformation durchgesetzt?
% Antwort: Homogene Geometrie und Quaternionen
\item Nennen Sie drei Anforderungen an geometrische 3D-Transformationen:
% Antwort: Geschlossene Ausdrücke, Invertierbarkeit, Interpolation
\item Wie lautet die Rotationsmatrix für eine Rotation um die $x$-Achse? % Wie SS13 Klausur Aufgabe 2.2a)
\item Wie lautet die Rotationsmatrix für eine Rotation um die $y$-Achse? % WIe WS12 Klausur Aufgabe 5
\item Wie lautet die Rotationsmatrix für eine Rotation um die $z$-Achse?
\item Der Vektor $a$ soll um $180°$ um die $x$-Achse gedreht werden. Stellen Sie das entsprechende Quaternion auf. 
\item Wie lautet die Formel für die Umrechnung von Quaternion zu Rotationsmatrix?
\item Wie lautet die Formel für die Umrechnung von Rotationsmatrix zu Quaternion?
% Antwort: q=(0,(1,0,0)
\item Wie lautet die Formel zur numerischen Berechnung der SLERP zwischen $q$ und $r$?
\item Berechnen Sie SLERP(q,r,0) und SLERP(q,r,1).
% Antwort: SLERP(q,r,0)=q und SLERP(q,r,1)=1
	\item Gegeben sei das Quaternion $q_1=(s,(x,y,z))=(2,(4,-3,0))$. Berechnen Sie das multiplikativ inverse Quaternion $q^{-1}_{1}$. % Übungsblatt 6 von 2012
	% Lösung $q^{-1}_{1}=(\frac{2}{29},(-\frac{4}{29},\frac{3}{29},0))$
	\item Rotieren Sie den Punkt $\vec{x}=(0,0,3)$ mit dem Quaternion $q_2=(\frac{\sqrt{2}}{2},(\frac{\sqrt{2}}{2},0,0))$. % Übungsblatt 6 von 2012
	% Lösung $\vec{x}'=(0,-3,0)$
	\item Wie lautet die Formel zur analytischen Berechnung der SLERP zwischen $q$ und $r$, die in der Vorlesung vorgestellt wurde? % Klausur WS12 - Aufgabe 4 b)
\end{enumerate}


\subsection{Erweitertes Kameramodell}
\begin{enumerate}
% Selbst ausgedacht
	\item Wenn man das in der Vorlesung behandelte Erweiterte Kameramodell mit dem Lochkameramodell vergleicht, welche vier Vereinfachungen werden beim Lochkameramodell gemacht?
\end{enumerate}

\subsection{Fähigkeitencheck für die Klausur}
\begin{itemize}
	\item Berechnung von Rotationsmatrix und inverser Rotationsmatrix % Wie bei Übungsblatt 5
	\item Umgang mit Stereokameramodell % Wie bei Übungsblatt 5
	\item Umgang mit Epipolargeometrie % Wie bei Übungsblatt 5
	\item Umgang mit Quaternionen % Wie bei Übungsblatt 6
\end{itemize}

































% !TeX root = summary.tex

\chapter{Visuelle Wahrnehmung des Menschen}
Klausurrelevant Folien 1-24
\subsection{Bewegungserfassung I}
Problemstellung beim Human Motion Capture (HMC):
\begin{itemize}
	\item Eingabe: Sequenz von Bildern bzw. Bildpaaren oder Bildtupeln
	\item Ausgabe: Geschätzte Konfiguration (Gelenkwinkel) für jedes Frame bezüglich eines zuvor definierten Menschmodells
	\item Schwierigkeit: Hohe Dimensionalität des Suchraumes
\end{itemize}
\subsubsection{Menschmodell}
Menschmodell für HMC setzt sich zusammen aus
\begin{itemize}
	\item Kinematischem Modell
	\item Geometrischen Modell
	\begin{itemize}
		\item Meist aus Festköpern
		\item Optional: deformierbares Hautmodell
	\end{itemize}
\end{itemize}
Aus Gründen der Rechenzeit werden vereinfachte Modelle verwendet.
\subsubsection{Kinematisches Modell}
\begin{itemize}
	\item Definiert die Anzahl und Art der Gelenke
	\item Definiert die Segmentlängen zwischen der Gelenken
	\item Für die Erfassung wird die Schulter meist durch ein einzelnes Kugelgelenk modelliert
	\end{itemize}
\subsubsection{Geometrisches Modell}
\begin{itemize}
	\item Definiert die 3D-Form der einzelnen Segmente
	\item Übliche 3D-Primitive:
	
	\begin{itemize}
		\item Zylinder
		\item Kegelausschnitte (Kreis- oder Ellipsenförmiger Querschnitt)
	\end{itemize}
\end{itemize}
\subsubsection{Berechnung}
Berechnung der projizierten Kontur $\overline{P_1 P_2}$ und $\overline{P_3 P_4}$ eines Kegelausschnitts mit kreisförmigem Querschnitt.
Gegeben:
\begin{itemize}
	\item Porjektionszentrum=Ursprung $Z$
	\item Fußpunkt $c$
	\item Richtung $n$
	\item Länge $L$
	\item Radien $r$,$R$
\end{itemize}
Berechnung
\begin{itemize}
	\item $u=\frac{n\times c}{\left|n\times c\right|}$
	\item $c_t=c+L \cdot \frac{n}{\left|n\right|}$
	\item $p_{1,3}=c \pm R \cdot u$
	\item $p_{2,4}=c_t \pm r \cdot u$
\end{itemize}
\subsubsection{Bildbasiert mit Partikelfilter}
Den Kern bildet eine Wahrscheinlichkeitsfunktion, welche bewertet, wie gut eine gegeben Konfiguration des Menschmodells zu den aktuellen Beobachtungen (Bilddaten) passt.

Hinweise (engl. Cues) für die Bewertung, die aus den Bilddaten gewonnen werden können sind:
\begin{itemize}
	\item Region Cue \cite{Deutscher.2000}
	\item Kanten Cue \cite{Deutscher.2000}
	\item Distanz Cue [Azad]
\end{itemize}
\subsubsection{Region Cue}
\begin{itemize}
	\item Benötigt Segmentierung der Person vom Hintergrund
	\item Bewertet den Abgleich des Segmentierungsergebnisses mit der Projektion der Körpersegmente
	\item Hierzu werden Punkte innerhalb der projizierten Kontur überprüft
	\item Bewertungsfunktion:
\end{itemize}
\subsection{Iterative Closet Point (ICP)}
\subsection{Bewegungserfassung II}













% !TeX root = summary.tex

\section{Wissen und Planung}

\subsection{Einführung}

\subsubsection*{Was ist Wissen?}
Gespeicherte
\begin{itemize}
\item Beschreibungen, Modelle, Aktionsfolgen
\item Vorschriften zur Reaktion auf Ereignisse
\item Motorische, kognitive Fähigkeiten
\end{itemize}
konkreter:
\begin{itemize}
\item Computerprogramme
\item Regeln der Aussagenlogik
\item Gewichte von Neuronalen Netzen, \dots
\end{itemize}

\subsection{Grundlagen}

% Bild Folie 8
\mypic{6}{wb1}

Eine Wissensdatenbank\index{Wissensdatenbank} besteht aus Sätzen, die in einer Wissensrepräsentationssprache abgefasst sind. Jeder Satz stellt eine Annahme über die Welt dar:
\begin{center}
\begin{tabular}{ll}
$\alpha$: & "{}\textbf{Ident1} IST-EIN Fahrrad."{} \\ $\beta$: & "{}\textbf{Ident2} IST-EIN Mensch."{} \\ $\gamma$: & "{}\textbf{Ident3} BESITZT Ident1."{} oder \\ $\alpha$: & "{}$r \in R^{20}$."{}
\end{tabular}
\end{center}

% Bild Folie 9
\mypic{8}{wb2}

\begin{description}
\item[Erkläre:] Füge der Datenbank Wissen hinzu.
\item[Befrage:] Frage Wissen aus Datenbank ab.
\end{description}
\begin{center}
\begin{tabular}{|c|}
\hline
Erkläre(DB, "{}Ident1 BESITZT Ident2."{}) \\ $\to$ OK \\ Befrage(DB, "{}Was BESITZT Ident2?"{}) \\ $\to$ Ident1 \\ \hline
\end{tabular}
\end{center}

% Bild Folie 10
\mypic{8}{wb3}

\begin{description}
\item[Deduktion\index{Deduktion}:] Leite neue Sätze von bekannten ab. Injektive Abbildung Deduktion: $$DB \to DB$$ Minimale Deduktion: Identität.
\end{description}

\subsection{Logik allgemein}

\begin{center}
Logik \quad : \quad (Symbolmenge, Modellmenge, Syntax, Semantik, Folgerungsop. $\models$)
\end{center}

\subsubsection*{Symbol-, Modellmenge}

\begin{itemize}
\item Elementarwerte für Aussagen: wahr, falsch
\item Symbolmenge $S$ enthält alle Symbole, über die Aussagen gemacht werden können.
\item Modell: Menge $\{(s_i,w_i)\}$ von Symbolen \\ $s_i \in S$ mit zugeordneten Werten, $w_i \in Def(s_i)$
\item Modellmenge: Menge aller möglichen Modelle: $$\forall i \forall w \in Def(s_i) \exists M \, : \, (s_i,w_i) \in M = (s_i,w)$$
\end{itemize}
\textbf{\textsl{Beispiel: Aussagenlogik}}
\begin{eqnarray*}
S &=& \{a,b\} \\ M &=& \{ \{ (a,wahr),(b,wahr) \} , \\ && \,\,\, \{ (a,wahr),(b,falsch)\} , \\ && \,\,\, \{ (a,falsch),(b,wahr) \} , \\ && \,\,\, \{ (a,falsch) , (b,falsch) \} \}
\end{eqnarray*}
Schreibweise: lateinische Buchstaben für Symbole, griechische Buchstaben für Sätze

\subsubsection*{Syntax}

\begin{description}
\item[Syntax\index{Syntax}:] Legt fest, welche Sätze wohlgeformt sind.
\end{description}
Nur wohlgeformte Sätze sind gültig!
\begin{center}
\begin{tabular}{|r @{$\, \Rightarrow \,$}l @{\quad} l|}
\hline
$(a \vee b)$ & $c$ & wohlgeformt \\
$()a \vee$ & $bc$ & nicht wohlgeformt \\ \hline
\end{tabular}
\end{center}

\subsubsection*{Semantik}

\begin{description}
\item[Semantik\index{Semantik}:] Bestimmt den Wahrheitswert eines Satzes in Bezug auf ein Modell.
\end{description}
\begin{eqnarray*}
\textrm{Satz } \alpha &=& (a \wedge b) \Rightarrow c \\ \textrm{Modell } M &=& \{ (a,wahr) , (b,wahr) , (c,wahr) \} \\ &\Rightarrow& \alpha / M \,\, wahr \, , \, M \textrm{ erfüllt } \alpha \\
\textrm{Modell } M &=& \{ (a,wahr) , (b,wahr) , (c,falsch) \} \\ &\Rightarrow& \alpha / M \,\, falsch
\end{eqnarray*}

\subsubsection*{Folgerung $\models$}

$$\alpha \models \beta \, : \, \textrm{"{}} \beta \textrm{ folgt aus } \alpha \textrm{"{}}$$
$\alpha \models \beta$ genau dann, wenn für alle Modelle, in denen $\alpha$ wahr ist, $\beta$ ebenfalls wahr ist.
$$\alpha \, : \, x+y=4 \qquad \beta_1 \, : \, y = 4-x \qquad \beta_2 \, : \, \frac{x}{y} = 1 \quad \Rightarrow \quad \alpha \models \beta_1 \qquad \alpha \not\models \beta_2$$

Zurück zur Wissensdatenbank $WB$: \\
Kann als $\wedge$-verknüpfte Sequenz von Sätzen aufgefasst werden:
$$WB \, : \, \left\{ \begin{array}{r @{\, : \,} r @{\, = \,} l} \alpha_1 & x+y & 4 \\ \alpha_2 & x/y & 1 \end{array} \right\}$$
$\Rightarrow$ $WB = \alpha_1 \wedge \alpha_2$ $wahr$ $\forall M \, : \, x=2, \, y=2$ \\
$\Rightarrow$ $M$ erfüllt $WB$ $\forall M \, : \, x=2, \, y=2$ \\
$$WB \models \beta \quad \textrm{ gdw } \quad \left( \bigwedge\limits_{\alpha_i \in WB} \alpha_i \right) \Rightarrow \beta = wahr$$

\subsubsection*{Modellprüfung}

Algorithmus zur Überprüfung einer $WB \models \alpha$-Relation:
\begin{enumerate}
\item Finde Menge aller Modelle $M = \{M_i\}$ über $S$
\item Eliminiere alle $M_i$ mit $WB/M = falsch$
\item $WB \models \alpha$ genau dann, wenn $$\forall M_i \in M \, : \, \alpha / M_i = wahr$$
\end{enumerate}

\subsubsection*{Deduktion}

Wenn ein Algorithmus $i$ existiert, der den Satz $\alpha$ aus der Wissensbasis $WB$ ableiten kann, schreiben wir $$WB \vdash_i \alpha$$
("{}$\alpha$ wird durch $i$ aus $WB$ abgeleitet"{} oder "{}$i$ leitet $\alpha$ aus $WB$ ab"{}) \\[0,1cm]
Eigenschaften von Deduktionsalgorithmen:
\begin{itemize}
\item Algorithmus $i$ "{}korrekt"{} genau dann, wenn er \underline{nur} Sätze aus $WB$ ableitet, die aus $WB$ folgen: $$\forall \alpha \, : \, WB \vdash_i \alpha \quad \Rightarrow \quad WB \models \alpha$$
\item Algorithmus $i$ "{}vollständig"{} genau dann, wenn er \underline{alle} Sätze aus $WB$ ableitet, die aus $WB$ folgen: $$\forall \alpha \, : \, WB \models \alpha \quad \Rightarrow \quad WB \vdash_i \alpha$$
\end{itemize}

\subsubsection*{Zusammenfassung}

Eine Logik:
\begin{itemize}
\item bestimmt den Wahrheitsgehalt von Sätzen in Bezug auf Modelle
\item besteht aus Symbolmenge, Modellmenge, Syntax, Semantik, Folgerung
\end{itemize}
Eine Wissensbasis:
\begin{itemize}
\item besteht aus einer Menge von Sätzen
\item passt zu einem Modell $M$ (oder auch nicht)
\end{itemize}
Ein Deduktions-Algorithmus
\begin{itemize}
\item leitet Sätze aus einer Wissensbasis ab
\item kann korrekt und / oder vollständig sein
\end{itemize}

\subsection{Aussagenlogik}

\begin{itemize}
\item Aussagenlogik als Beispiel einer sehr einfachen Logik
\item sehr alt (antikes Griechenland)
\item bildet die Basis, von der die allgemeine Logik abgeleitet wurde
\item eignet sich sehr "{}natürlich"{} zur Illustration bestimmter Konzepte
\end{itemize}

\subsubsection*{Syntax}

\begin{eqnarray*}
Satz &\to& Atom \, | \, Komplex \\ Atom &\to& True \, | \, False \, | \, Symbol \\ Symbol &\to& P \, | \, Q \, | \, R \, \dots \\
Komplex &\to& \neg Satz \, | \, (Satz \wedge Satz) \, | \, (Satz \vee Satz) \, | \, (Satz \Rightarrow Satz) \, | \, (Satz \Leftrightarrow Satz)
\end{eqnarray*}

\subsubsection*{Semantik}
Wahrheitsgehalt im Modell $M$:
\begin{enumerate}
\item $True = wahr$, $False = falsch$ $\forall M$
\item Wahrheitsgehalt von Symbolen wird in $M$ spezifiziert
\item Wahrheitsgehalt aller anderen Sätze rekursiv:
\begin{enumerate}
\item $\forall$ Sätze $\alpha$ $\forall$ Modelle $M$ : $\neg \alpha$ $falsch$ gdw $\alpha \in M$ $wahr$
\item $\forall \alpha,\beta \, \forall M \, : \, \alpha \wedge \beta \,\, wahr$ gdw $\alpha \in M \,\, wahr$ und $\beta \in M \,\, wahr$
\item etc.
\end{enumerate}
(Alternative: Wahrheitstabelle)
\end{enumerate}

\subsubsection*{Muster}

Allgemeingültige Deduktionsregeln aus der klassischen Literatur:
\begin{itemize}
\item \textbf{Modus Ponens\index{Modus Ponens}:} \\ Wenn $\alpha \Rightarrow \beta$ und $\alpha$ gegeben sind, kann $\beta$ inferiert werden. $$\frac{\alpha \Rightarrow \beta \, , \, \alpha}{\beta}$$
\item \textbf{Und-Elimination\index{Und-Elimination}:} $$\frac{\alpha \wedge \beta}{\alpha} \qquad \frac{\alpha \wedge \beta}{\beta}$$ Aus einer Konjunktion kann jedes der Elemente inferiert werden. \\ Korrektheitsbeweis aus Wahrheitstabelle!
\end{itemize}

\subsection{Resolution\index{Resolution}}

Basis für Deduktionsalgorithmen \\
\textbf{Einheits-Resolutionsregel} (aus Modus Ponens): \\ Seien $p_1,\dots,p_k$ und $q$ Literale mit $q = \neg p_i$. Dann gilt: $$\frac{p_1 \vee \dots \vee p_k \, , \, q}{p_1 \vee \dots \vee p_{i-1} \vee p_{i+1} \vee \dots \vee p_k}$$
\textbf{Allgemeine Form} der Resolutionsregel: \\ $p_1,\dots,p_k$ und $q_1,\dots,q_n$ Literale mit $p_i = \neg q_j$. Dann gilt: $$\frac{p_1 \vee \dots \vee p_k \, , \, q_1 \vee \dots \vee q_n}{p_1 \vee \dots \vee p_{i-1} \vee p_{i+1} \vee \dots \vee p_k \vee q_1 \vee \dots \vee q_{j-1} \vee q_{j+1} \vee \dots \vee q_n}$$
Jeder vollständige Suchalgorithmus, kombiniert mit der Resolutionsregel,
\begin{itemize}
\item kann jede Schlussfolgerung ableiten, die aus jeder Wissensbasis der Aussagenlogik folgt.
\item bildet die Basis für Familien von vollständigen Deduktionsalgorithmen.
\item beweist $WB \models \alpha$ durch Widerlegung von $(WB \wedge \neg \alpha)$.
\end{itemize}
Voraussetzung: Sätze müssen in konjunktiver Form vorliegen.
\begin{description}
\item[Klausel\index{Klausel}:] Eine Disjunktion von Literalen $$(l_1 \vee \dots \vee l_n)$$
\item[Konjunktive Form (KF):] Jede aussagenlogische Formel kann als Konjunktion von Klauseln ausgedrückt werden. Dann liegt sie in konjunktiver Form vor: $$(l_{1,1} \vee \dots \vee l_{1,k_1}) \wedge \dots \wedge (l_{n,1} \vee \dots \vee l_{n,k_n})$$
\end{description}

\subsubsection*{Resolutionsalgorithmus}

Zeige $WB \models \alpha$ durch Widerlegung von $(WB \wedge \neg \alpha)$
\begin{center}
\begin{tabular}{l}
$klauseln$ $:=$ Menge Klauseln in KF $(WB \wedge \neg \alpha)$ \\
$neu$ $:=$ $\{\}$ \\
\verb|loop:| \\
\verb|  for all | $C_m$, $C_n$ \verb| in | $klauseln$: \\
\verb|    |$res :=$ \verb| resolution(|$C_m$, $C_n$\verb|)| \\
\verb|    if | $\emptyset$ \verb| = | $res$: \verb| return | $true$ \\
\verb|    |$neu := neu \cup res$ \\
\verb|  if | $neu \subseteq klauseln$: \verb| return | $false$ \\
\verb|  |$klauseln := klauseln \cup neu$
\end{tabular}
\end{center}

Beispiel:
% Bild Folie 34 (Block)
\mypic{10}{bsp_resolution}

Der Resolutionsalgorithmus ist vollständig aber auch sehr afwendig -- $O(n^2)$. Realistische Vereinfachungen:
\begin{itemize}
\item Horn-Klauseln (Prolog), Vorwärts-/Rückwärts-Verkettung
\item Davis-Putnam-Logemann-Loveland (DPLL)
\end{itemize}

\subsection{Effiziente Logik-Algorithmen}

\begin{description}
\item[Horn-Klausel\index{Horn-Klausel}:] Disjunktion von Literalen, von denen höchstens eins positiv ist: $$(\neg a \vee \neg b \vee c)$$ lasst sich schreiben als: $$\underbrace{(a \wedge b)}_{Koerper} \Rightarrow \underbrace{c}_{Kopf}$$
\end{description}
Spezialtypen von Horn-Klauseln:
\begin{itemize}
\item keine negativen Literale: Fakt, Axiom
\item genau ein positives Literal: Definition
\item kein positives Literal: Integritätseinschränkung
\end{itemize}
Eigenschaften von Horn-Klausel-basierten Wissensbasen:
\begin{itemize}
\item einfach visualisierbar: Und-Oder-Graph
\item Inferenz durch Vorwärts- und Rückwärtsverkettung, sehr natürliche und leicht verständliche Algorithmen
\item Folgerungsentscheidung kann in linearer Zeit geschehen!
\end{itemize}
\begin{description}
\item[Und-Oder-Graph\index{Und-Oder-Graph}:] einfache Visualisierung eines Horn-Klausel-Systems.
\end{description}

% Bild Folie 39
\mypic{7}{und_oder_graph}

Vorwärtsverkettung:
\begin{itemize}
\item gehe von Fakten aus
\item arbeite dich vorwärts durch den Baum
\item Ergebnis: alle durchführbaren Folgerungsentscheidungen
\end{itemize}
Rückwärtsverkettung:
\begin{itemize}
\item gehe von Anfrage aus
\item arbeite dich rückwärts durch den Baum
\item stoppe bei bekannten Fakten (Ground Truth, Axiome)
\item Ergebnis: Wahrheitsgehalt der Anfrage
\end{itemize}
\begin{description}
\item[Davis-Putnam-Logemann-Loveland\index{Davis-Putnam-Logemann-Loveland} (DPLL):] Rekursiver Modellprüfungs-Algorithmus mit folgenden Verbesserungen:
\begin{itemize}
\item früher Abbruch: benutzt $(A \vee B) \wedge (A \vee C) = true$, sobald $A = true$
\item Einheitsklauseln (Klauseln mit nur einem Literal): Konjunktionen mit Einheitsklauseln sind nur wahr, wenn die Einheitsklauseln wahr sind \\ $\Rightarrow$ weitere Einschränkung des Suchraums
\item reine Symbole: in $(A \vee \neg B)$, $(\neg B \vee \neg C)$, $(C \vee A)$ sind $A$ und $\neg B$ rein und $C$ unrein \\ Suche nach reinen Symbolen schränkt Raum möglicher Modelle stark ein!
\end{itemize}
\end{description}
Algorithmus:
\begin{center}
\begin{tabular}{l}
\verb|funktion| $DPLL(k,s,m)$: \\
\verb|  # | $k$ Klauselmenge, $s$ Symbolmenge, $m$ Modell \\
\verb|  if| alle Klauseln wahr in $m$: \verb|return | $true$ \\
\verb|  if| eine Klausel falsch in $m$: \verb|return | $false$ \\
\verb|  |$P, \, wert :=$ \verb| FINDE_EINHEITSKLAUSEL(|$k,s,m$\verb|)| \\
\verb|  if | $P$ nicht leer: \\
\verb|    return | $DPLL(k,s-P,$\verb|SETZE_IN_MODELL(|$P,wert,m$\verb|)|$)$ \\
\verb|  |$P, \, wert :=$ \verb| FINDE_REINES_SYMBOL(|$k,m$\verb|)| \\
\verb|  if | $P$ nicht leer: \\
\verb|    return | $DPLL(k,s-P,$\verb|SETZE_IN_MODELL(|$P,wert,m$\verb|)|$)$ \\
\verb|  |$P :=$ \verb| ERSTES(|$s$\verb|); | $rest :=$ \verb| REST(|$s$\verb|)| \\
\verb|  return |$DPLL(k, rest,$\verb|SETZE_IN_MODELL(|$P,true,m$\verb|)|$)$ \verb| or| \\
\verb|         |$DPLL(k, rest,$\verb|SETZE_IN_MODELL(|$P,false,m$\verb|)|$)$
\end{tabular}
\end{center}
DPLL zur Überprüfung von Erfüllbarkeit:
\begin{center}
\begin{tabular}{l}
\verb|function| $DPLL_ERFUELLBAR(s):$ \\
\verb|  #| $s:$ Satz der Aussagenlogik \\
\verb|  | $klauseln :=$ Klauselmenge der KNF von $s$ \\
\verb|  | $symbole :=$ Menge der Symbole aus $s$ \\
\verb|  return| $DPLL(klauseln, symbole, \{\})$
\end{tabular}
\end{center}

\subsection{Semantisches Planen}

\subsubsection*{Repräsentation von Plänen}

Wie kann man Probleme so formulieren, dass
\begin{itemize}
\item ein Lösungsplan einfach zu erstellen ist?
\item die Existenz einer Lösung bewiesen/widerlegt werden kann?
\end{itemize}
$\to$ Einschränkungsregeln für formelle Spezifikation von Zuständen, Zielen, Aktionen. Beispiele solcher Regelsysteme: STRIPS, ADL

\subsubsection*{STRIPS\index{STRIPS}}

\begin{itemize}
\item \textbf{ST}anford \textbf{R}esearch \textbf{I}nstitute \textbf{P}roblem \textbf{S}olver
\item "{}Urvater"{} vieler Planungssysteme
\item sehr einfach aufgebaut
\item teilweise eingeschränkt
\end{itemize}
Repräsentation von Zuständen:
\begin{itemize}
\item Konjunktion positiver aussagenlogischer Literale $$Blau \wedge Rund$$
\item Literale erster Ordnung ($L1$) $$IstTasse(T1) \wedge IstUntertasse(U1) \wedge StehtAuf(T1,U1)$$
\item $L1$ müssen funktions- und variablenfrei sein!
\end{itemize}
\textbf{\textsl{Closed-World Assumption}}
\begin{itemize}
\item geschlossene Welt: was nicht im Zustand vorkommt, ist falsch!
\item keine Negationen im Zustand und Vorbedingungen, aber in Nachbedingungen benötigt:
$$b \quad \Rightarrow \quad \begin{array}{l} a \textrm{ darf nicht im Weltmodell vorkommen} \\ b \textrm{ muss im Weltmodell vorkommen} \end{array}$$
\item Negationen in Nachbedingungen:
$$\neg c \wedge d \quad \Rightarrow \quad \begin{array}{l} \textrm{entferne } c \textrm{ aus dem Weltmodell} \\ \textrm{füge } d \textrm{ dem Weltmodell hinzu} \end{array}$$
\end{itemize}
Repräsentation von Zielen:
\begin{itemize}
\item Ziel: teilweise spezifizierter Zustand
\item angegeben als Konjunktion von positiven Literalen
$$StehtAuf(T1,U1) \wedge StehtAuf(U1,Tisch)$$
\item ein Zustand $S$ erfüllt ein Ziel $Z$, wenn er alle Literale in $Z$ enthält
$$StehtAuf(T1,U1) \wedge StehtAuf(U1,Tisch) \wedge StehtAuf(Teller,Tisch)$$
erfüllt $$StehtAuf(T1,U1) \wedge StehtAuf(U1,Tisch)$$
\end{itemize}
Aktionen werden angegeben durch
\begin{itemize}
\item Aktionsname
\item Parameter
\item Vorbedingungen
\item Effekte
\end{itemize}
$$A = (N_A,P_A,V_A,E_A)$$
\begin{center}
\begin{tabular}{l}
$(StelleAuf,$ \\ $(Obj1,Obj2),$ \\ $IstUntersatz(Obj2) \wedge Auf(Obj1,Tisch) \wedge Auf(Obj2,Tisch),$ \\ $\neg Auf(Obj1,Tisch \wedge Auf(Obj1,Obj2))$
\end{tabular}
\end{center}
Alternative Schreibweise:
\begin{center}
\begin{tabular}{lll}
\textbf{Action} ( & \multicolumn{2}{l}{$StelleAuf(Obj1, Obj2),$} \\
& \textbf{Vorbed:} & $IstUntersatz(Obj2) \wedge$ \\ && $Auf(Obj1, Tisch) \wedge$ \\ && $Auf(Obj2, Tisch),$ \\
& \textbf{Effekt:} & $\neg Auf(Obj1, Tisch) \wedge$ \\ && $Auf(Obj, Obj2)$ \qquad \quad )
\end{tabular}
\end{center}
Semantik:
\begin{description}
\item[Anwendbarkeit:] Eine Aktion ist anwendbar auf allen Modellen $M$, die ein Modell für die Vorbedingung $V_A$ sind.
\item[Ergebnis:] Das Ergebnis $M'$ der Ausführung einer Aktion $A$ auf einem Modell $M$ erhält man durch
\begin{itemize}
\item entfernen aller negativen Literale des Effekts $E_A$ aus $M$
\item hinzufügen aller positiven Literale aus $E_A$ zu $M$.
\end{itemize}
\end{description}
Einschränkunen von STRIPS:
\begin{itemize}
\item Literale müssen funktionsfrei sein
\begin{itemize}
\item endliche Grammatik
\item jede Aktion kann als endliche aussagenlogische Konjunktiondargestellt werden
\item Lösbarkeitsbeweis einfach
\end{itemize}
\item Nur positive Literale $l$, "{}geschlossene Welt"{}
\begin{itemize}
\item einfachere Planung
\item aber: Falsch-Sachverhalte nur implizit
\end{itemize}
\item keine Quantisierung
\end{itemize}
$\Rightarrow$ STRIPS-Aussagen oft lang und unübersichtlich

\subsubsection*{ADL\index{ADL}}

\begin{itemize}
\item \textbf{A}ction \textbf{D}escription \textbf{L}anguage: Weiterentwicklung von STRIPS
\item "{}offene Welt"{}: Was nicht im Zustand vorkommt, ist unbekannt.
\item negative Literale und Disjunktionen erlaubt
\item Effekt $\neg P \wedge Q$:
\begin{itemize}
\item füge $\neg P$ und $Q$ hinzu
\item lösche $P$ und $\neg Q$
\end{itemize}
\item Gleichheits-Prädikat "{}="{} eingebaut
\begin{itemize}
\item $StelleAuf(T1,T1)$ nicht mehr möglich
\end{itemize}
\end{itemize}
Typüberprüfung
\begin{itemize}
\item Typüberprüfung benötigt für Ausführbarkeit bestimmter Aktionen
\item in STRIPS nur explizit (als Prädikat): $IstUntersetzer(x)$, $IstTasse(y)$
\item häufig weggelassen (Schreibarbeit!), aber eigentlich nötig
\item in ADL implizit möglich:
\begin{center}
\begin{tabular}{lll}
\textbf{Action} ( & \multicolumn{2}{l}{$StelleAuf(Obj1 \, : \, Untersatz, \,\, Obj2 \, : \, Manipulierbar),$} \\
& \textbf{Vorbed:} & $Auf(Obj1, Tisch) \wedge$ \\ && $Auf(Obj2, Tisch),$ \\
& \textbf{Effekt:} & $\neg Auf(Obj1, Tisch) \wedge$ \\ && $Auf(Obj, Obj2)$ \qquad \quad )
\end{tabular}
\end{center}
\end{itemize}
Suche im Zustandsraum ist einfach:
\begin{itemize}
\item keine Funktionssymbole \\ $\Rightarrow$ endlicher Zustandsraum \\ $\Rightarrow$ Standard-Algorithmen zur Baumsuche (z.B. A*)
\item Transformation Vorbedingungen $\Leftrightarrow$ Effekte bijektiv \\ $\Rightarrow$ Suche auch rückwärts möglich
\end{itemize}
Aber:
\begin{itemize}
\item Zustandsraum oft sehr groß
\item naive Suche sehr ineffizient
\item benötigt gute Heuristik oder Segmentierung in Teilprobleme
\end{itemize}
Heuristiken für Zustandsraumsuche
\begin{itemize}
\item versuche, die Suchtiefe einzuschränken
\item Def. \textbf{Heuristikfunktion}\index{Heuristikfunktion}: $$h = H(M_0,M_1)$$ angenommene maximale Anzahl von Aktionen, um von $M_0$ zu $M_1$ zu kommen
\item beschränke Suche auf Teilbaum der Tiefe $h$
\item wenn nicht gefunden: suche je eine Ebene tiefer
\item Problem: finde möglichst gute Heuristikfunktionen
\end{itemize}
\subsubsection*{Vorranggraph\index{Vorranggraph}}
Vor- und Nachbedingungen können zur Problemdekomposition verwendet werden. Parralelisierung möglich? $\to$ Vorranggraph \\
Vorranggraph ermöglicht:
\begin{itemize}
\item Parallelisierung der Planung
\item Parallelisierung der Ausführung
\item bei dynamischen Effekten (Hindernisse, etc.)
\begin{itemize}
\item Blockierung des momentanen Teilplans
\item zuerst Ausführung anderer Teilpläne
\item dann Überprüfung, ob blockierter Teilplan jetzt ausführbar (Hindernis jettz weg?)
\item ggf. Neuplanung nur ab der letzten Gabelung notwendig
\end{itemize}
\end{itemize}

\subsection{Umweltmodell\index{Umweltmodell}}

Das Umweltmodell eines kognitiven Systems bildet die reale Umwelt auf eine innere Repräsentation ab. Logisch-semantische Beziehungen genügen für ein System mit Aktorik nicht; zusätzlich:
\begin{itemize}
\item geometrisches Modell: Ausdehung und Lage von Objekten
\item topologisches Modell: (teil-geometrische) Beziehungen zwischen Objekten
\end{itemize}


\begin{description}
\item[Geometrisches Modell:] Anwendungen:
\begin{itemize}
\item Bahnplanung feinkörnig
\item aktives Messen
\item Objekterkennung (Basisobjekte)
\item fahren von $x_s,y_s,z_s$ nach $x_e,y_e,z_e$
\end{itemize}
\item[Topologisches Modell:] Anwendungen:
\begin{itemize}
\item Planung (Manipulationsplanung, mittlere Körnigkeit, Bewegungsplanung)
\item "{}fahre von Raum1 nach Raum5"{}
\end{itemize}
Über topologische Modelle:
\begin{itemize}
\item Anordnung von Objekten und Umwelt relativ zueinander gespeichert
\item abgeleitet aus geometrischen Modellen: \\ Pfade z.B. aus Voronoi.Diagrammen, Quadtree, Potentialfeldern
\item grobe Aktionsplanung aus topologischen Modellen
\item kurzfristige Verwendung anderer Modelle bei unerwarteten Hindernissen (geometrischer Bodenplan, ad-hoc erzeugte Kantenmodelle aus Laserscan, Kamera)
\end{itemize}
\item[Semantisches Modell:] Anwendung: Planung auf Aufgabeneben $\to$ "{}fahre durch alle Büros"{} \\
Über semantische Modelle:
\begin{itemize}
\item besondere Bedeutung im Rahmen der Mensch-Maschine-Kommunikation
\item belegte Fläche als Objekt klassifiziert (z.B Schrank, Stuhl)
\item eingeordnet in topologische Beschreibung
\item Objekteigenschaften verwertbar: Objektzustand (Tür offen, Tasse leer); äußere Erscheinung des Objekts (Form, Farbe) als Diskriminator und Hilfe für die Sensorik; geometrisches Objektmodell für Greifplanung etc.
\end{itemize}
zugehörig zu semantischen Umweltbeschreibungen:
\begin{itemize}
\item Funktion eines Objektes
\item Landmarken
\item geometrische Objekte
\item topologische Umweltdarstellungen
\item Positionen
\end{itemize}
\end{description}

Umweltrepräsentation der Information:
\begin{itemize}
\item Pfade \item Freiraum \item Objekte \item gemischte Modelle
\end{itemize}

\subsubsection*{Pfade}

Normal: 2-dimensionaler Raum
\begin{itemize}
\item Polygonale Beschreibung der Objekte
\item Pfad: lineare oder nichtlineare Verbindung zweier Punkte im Operationsraum
\item Speicherung der Umwelt und Planungsinformation
\item Repräsentation als ungerichteter Graph
\end{itemize}
Sichtbarkeitsgraphen:
\begin{itemize}
\item einfache Umweltdarstellung, partielle Modellierung
\item kollisionsfreie Bewegung nur auf gespeicherten Pfaden
\item Sichtbarkeitsgraphen zur automatischen Generierung von Pfaden
\end{itemize}
Erstellung eines Sichtbarkeitsgraphen:
\begin{center}
\begin{tabular}{l}
\verb|for i = 1..#Obj| \\
\verb|  for j = 1..#Objekte[i]| \\
\verb|    for k = i..#Obj| \\
\verb|      for l = 1..#Objekte[k]| \\
\verb|        if (Sichtbarkeitstest(objekte[i][j], objekte[k],[l] == ok)| \\
\verb|          neuer Pfad(i,j,k,l)|
\end{tabular}
\end{center}
\begin{description}
\item[Voronoi-Diagramme\index{Voronoi-Diagramme}:] Fahrwege liegen auf maximalem Abstand zu den Hindernissen. Das Voronoi-Diagramm ist der duale Graph der Delaunay-Triangulation.
\end{description}

\subsubsection*{Freiraum}

\begin{itemize}
\item Projektion der realen Welt in 2D/3D-Grundrissdarstellung
\item kollisionsfrei befahrbare Freiräume in geeignete Bereiche zerlegt
\item vorhandene Objekte und Hindernisse nicht berücksichtigt
\end{itemize}
Freiraumgraph:
\begin{itemize}
\item Knoten: Freiraumbereiche
\item Kanten: Verbindung der Gebiete
\end{itemize}
Vorteile:
\begin{itemize}
\item geringere Komplexität der Wegplanung
\item reduzierter Zeit- und Rechenaufwand
\item Fahrsicherheitsüberlegungen deutlich vereinfacht
\item Verbesserung der Anpassungsfähigkeit der Algorithmen an verschiedene Umwelten
\end{itemize}
Darstellungsformen der Freiflächen:
\begin{itemize}
\item Kacheln (Quader)
\begin{itemize}
\item Darstellung von besetzten Räumen durch orthogonale begrenzende Linien (Näherungen)
\item Verlängerung der Linien bis zum nächsten besetzten Raum
\item bildet Mosaik von besetzten und freien Kacheln/Quadern
\item freie Kacheln $\to$ Graph, durch den Bewegung möglich ist
\end{itemize}
Quadtree-/Octtree-Aufteilung:
\begin{itemize}
\item hierarchische Aufteilung in freie und belegte Zellen
\item Aufteilung teilbelegter Zellen in 4 (2D) bzw. 8 (3D) Unterzellen
\item Resultat: Baumstruktur
\item Pfadplanung: Aufstieg von Start- und Zielknoten bis zum ersten gemeinsamen Knoten; Liste besuchter Knoten: Pfad
\end{itemize}
\item konvexe Polygone (Polyeder)
\end{itemize}

\subsubsection*{Objekte}

\begin{itemize}
\item Darstellung der Objekte der realen Umwelt (Türen, Wände, Hindernisse)
\item 3-dim. Darstellung der Umgebung aus Sensorwahrnehmungen
\item Projektion auf $x$-$y$-Ebene für Navigation bodengebundener Roboter ausreichend
\end{itemize}
verschiedene Darstellungen:
\begin{itemize}
\item Kantenmodelle: \\
Ermittlung von markanten Punkten. Verbinden durch geeignete Kanten auf Oberfläche des Objekts
\item Oberflächenmodelle
\begin{itemize}
\item Nachbildung der Objektoberflächen
\item Darstellung ebener Flächenelemente mit Polygonen
\item gekrümmte Flächenelemente:
\begin{itemize}
\item mathematische Grundflächen (Zylinder, Kegel, Torusflächen)
\item Bezier-Flächen (Erweiterung des Ansatzes der Bezierkurven): \\
gegeben ist ein Gitter von Führungspunkten $P_{ij}$ mit $0 \leq i \leq N$ und $0 \leq j \leq M$. Damit ist die Fläche beschrieben durch $$F(u,v) = \sum\limits_{i=0}^N \sum\limits_{j=0}^M P_{ij} \cdot B_{i,N}(u) \cdot B_{j,M}(v)$$ mit
\begin{eqnarray*}
B_{i,N}(u) &=& (1-u) B_{i,N-1}(u) + uB_{i+1,N-1}(u) \quad , \quad B_{i,0} = 1 \\
B_{j,M}(v) &=& (1-v) B_{j,M-1}(v) + uB_{j+1,M-1}(v) \quad , \quad B_{j,0} = 1
\end{eqnarray*}
Die $B_{i,N}$ bzw. $B_{j,M}$ heißen auch Bernsteinpolynome.
\item näherungsweise durch Freiformflächen (Patches)
\end{itemize}
\end{itemize}
\item Volumenmodelle:
\begin{itemize}
\item Unterscheidung von Raumpunkten hinsichtlich ihrer Lage zum Objekt (innen-/außenliegend)
\item Repräsentationsmöglichkeiten: Begrenzungsflächenmethode, Grundkörperdarstellung, Zellenzerlegung (Octtree), Volumenapproximation, einhüllende Quader, Geradensegmente
\end{itemize}
\end{itemize}

\subsubsection*{Modelle}

Aufgliederung des benötigten Wissens:
\begin{itemize}
\item Ausführung von Aufgaben
\begin{itemize}
\item semantisches Aufgabenmodell
\item Umweltmodell vorher, nachher
\end{itemize}
\item Bewegung über Grund in 2D, Bewegung des Arms in 3D
\begin{itemize}
\item statische Umweltkarte (geometrisch und topologisch)
\item dynamische Hinderniserfassung in 3D
\item Freiraum-, Hindernismodell
\end{itemize}
\item Manipulation von Objekten
\begin{itemize}
\item Objektpositionen
\item geometrische und topologische Objektmodelle
\end{itemize}
\end{itemize}

\subsection{Geometrisches Planen}

\subsubsection*{Grundlagen der Bahnplanung\index{Bahnplanung}}

Bewegung eines Roboters:
\begin{itemize}
\item Zustandsänderungen über der Zeit (Trajektorie)
\item relativ zu stationärem Koordinatensystem (kartesischer Raum, Gelenkwinkelraum)
\item häufig Gütekriterien, Neben-, Randbedingungen
\end{itemize}
Bekannt:
\begin{itemize}
\item $S_{start}$: Zustand zum Startzeitpunkt
\item $S_{ziel}$: Zustand zum Zielzeitpunkt
\end{itemize}
Gesucht:
\begin{itemize}
\item $S_I$: Zwischenzustände (Stützpunkte)
\item glatte, stetige Trajektorie
\end{itemize}
Bahnplanungsverfahren nach Zustandsraum:
\begin{itemize}
\item Gelenkwinkelzustandsraum (Konfigurationsraum)
\item 3-dim. euklidischer Raum
\item Sensorzustandsraum, Objektzustandsraum, \dots
\end{itemize}
Bahnplanungsverfahren nach Art des Roboters:
\begin{itemize}
\item Bahnplanung für Manipulatoren
\item Bahnplanung für mobile Roboter
\item Bahnplanung für Laufmaschinen und antropomorphe Systeme
\item Greif- und Montageplanung
\end{itemize}

\subsubsection*{Bahnplanung im Gelenkwinkelraum}

\begin{itemize}
\item Trajektorie als Funktion der Gelenkwinkel
\item Ausführung solcher Trajektoren durch
\begin{itemize}
\item Steuerung der Achsen unabhängig voneinander (Punkt zu Punkt) oder
\item achsinterpolierte Steuerung (Bewegung aller Achsen beginnt und endet zum gleichen Zeitpunkt) erfolgen
\end{itemize}
\item Bahnverlauf muss im kartesischen Raum nicht definiert sein
\item Vorteile: einfach; keine Singularitäten
\end{itemize}

\subsubsection*{Bahnplanung im kartesischen Raum}

\begin{itemize}
\item Trajektorie angegeben als Funktion der Endeffektorposition
\item Funktionen z.B.: lineare Bahnen, Polynombahnen, Splines
\item Vorteile:
\begin{itemize}
\item Verlauf der Trajektorie explizit in 3D
\item einfach nachvollziehbar, visualisierbar
\end{itemize}
\item Nachteile:
\begin{itemize}
\item für jeden Punkt muss Gelenkwinkelrücktransformation berechnet werden
\item Trajektorie nicht immer ausführbar (Arbeitsraumbegrenzung, Singularitäten des Roboters)
\end{itemize}
\end{itemize}

\subsubsection*{Bahnplanungs-Schema}

gegeben: Robotermodell (Geometrie, Kinematik), Umweltmodell

\begin{enumerate}
\item Berechnung des Konfigurationsraums $K$
\item Berechnung des Hindernisraums $H$
\item Berechnung des Freiraums $F = K \backslash H$
\item Zerlegung des Freiraums in Unterräume
\item Bahnplanung im Unterraum
\item Integration der lokalen Lösungen in Gasamtlösung
\end{enumerate}

\subsubsection*{Konfiguration}

\begin{description}
\item[Konfiguration $k_R$:] beschreibt den Zustand eines Roboters $R$
\begin{itemize}
\item im euklidischen Raum durch Lage und Orientierung
\item im Gelenkwinkelraum durch die Werte der Gelenke
\end{itemize}
\item[Konfigurationsraum $K_R$:] Raum aller möglichen Konfigurationen von $R$
\item[Weg $w$ von $k_{start}$ bis $k_{ziel}$:] stetige Abbildung $$w \, : \, [0,1] \to K \quad \textrm{mit} \quad w(0) = k_{start} \, , \, w(1) = k_{ziel}$$
\item[Arbeitsraum-Hindernis $h_{A,O}$:] Raum, der von einem Objekt $O$ im Arbeitsraum $A$ eingenommen wird
\item[Konfigurationsraum-Hindernis $h_{K,O}$:] Raum, der von einem Objekt $O$ im Konfigurationsraum $K$ eingenommen wird
\item[Hindernisraum $H$:] \quad $$H = \bigcup\limits_{O} h_{K,O}$$
\end{description}

\subsubsection*{Freiraum}

\begin{itemize}
\item Freiraum $F_R \, : \, F_R = K_R \backslash H$
\item Aufwand für Freiraumberechnung: $O(m^n)$ mit
\begin{itemize}
\item $n$: Anzahl der Freiheitsgrade des Roboters
\item $m$: Anzahl der Hindernisse
\end{itemize}
\item Deshalb oft approximative Verfahren zur Vereinfachung des Freiraums
\begin{itemize}
\item Sichtgraphen
\item Quadtree, Octtree
\end{itemize}
\end{itemize}

\subsection{Bahnplanung in 2D}

Einfacher Algorithmus: "{}Strassenkarten"{} \\
gegeben: 2-dim. Weltmodell, Start und Ziel \\
gesucht: güstigste Verbindung von Start zu Ziel \\
Lösung:
\begin{enumerate}
\item konstruiere Netz von Wegen $W$ in $F_R$
\item bilde $k_{start}$ und $k_{ziel}$ auf $W$ ab: $(W(k_{start}), W(k_{ziel})$
\item suche Weg $w$, der $W(k_{start})$ mit $W(k_{ziel})$ verbindet
\end{enumerate}

\subsubsection*{Wegkonstruktion}

Wegkonstruktion mit
\begin{itemize}
\item Retraktionverfahren (z.B. Voronoi-Diagramm)
\item Sichtgraphen
\item Zellzerlegungsmethoden
\end{itemize}
Suche im Wegnetz mit z.B.
\begin{itemize}
\item A*-Algorithmus (Baumsuche)
\item euklidischer Abstand
\item Potentialfeld
\end{itemize}

\begin{description}
\item[Retraktion\index{Retraktion}:] Sei $X$ eine Menge und $Y \subset X$. Eine surjektive Abbildung $$p \, : \, X \to Y$$ heisst Retraktion genau dann, wenn $p$ stetig ist und $p(y) = y$ für alle $y \in Y$ gilt. \\
D.h. die Abbildung der Menge $X$ auf ihre Teilmenge $Y$, wobei die Menge $Y$ auf sich selbst abgebildet wird. Für die Bahnplanung gilt:
\begin{itemize}
\item $Y$ ist ein Netz von eindimensionalen Kurven (Wegenetz).
\item Retraktionsmethoden unterscheiden sich in Wahl von $p$.
\end{itemize}
\end{description}

\subsubsection*{Sichtgraphen}

Konstruktion:
\begin{itemize}
\item Verbinde jedes Paar von Eckpunkten auf dem Rand von $F_R$ durch gerades Liniensegment, wenn das Segment kein Hindernis schneidet.
\item Verbinde $k_{start}$ mit $k_{ziel}$ analog dazu.
\end{itemize}
Anmerkungen:
\begin{enumerate}
\item Wege sind nur "{}halbfrei"{} (nicht kollisionsfrei), da Hinderniskanten auch Wegsegmente sein können. \\ Abhilfe: Erweiterung der Hindernisse
\item Wenn ein Weg gefunden ist, ist es auch der kürzeste Weg.
\item Methode ist exakt, wenn Roboter nur 2 translatorische Freiheitsgrade hat und sowohl Roboter als auch Hindernisse durch Polygone dargestellt werden können.
\item Methoden auch im $R^3$ anwendbar, jedoch sind die gefundenen Wege i.A. keine kürzesten Wege mehr.
\end{enumerate}

\subsubsection*{Kürzester Pfad im Graphen}

vom Startknoten ausgehend:
\begin{itemize}
\item wähle Nachbarknoten $N_k$ so, dass Evaluationsfunktion $f(N_k)$ minimal
\item suche von $N_k$ ausgehend weiter
\item wenn $f(N_k)$ nicht mehr kleiner wird, mache weiter oben im Baum weiter
\item Problem: $f(N_k)$ darf nicht vom Teilbaum an $N_k$ abhängen! \\ (Würden wir den kennen, bräuchten wir nicht suchen.)
\item Lösung: verwende Heuristik $h(N_k)$
\item beliebt für die Heuristikfunktion in 2D: euklidischer Abstand $$h(N_k) = ||N_k - N_{ziel}||$$
\item rein heuristikbasierte Suche ist nicht optimal und führt nur unter Einschränkungen zum Ziel
\item A*-Algorithmus\index{A*-Algorithmus}: \\ mit $g(N_k)$ Kosten zum Erreichen von $N_k$
\item beweisbar optimal $$f(N_k) = g(N_k) + h(N_k)$$
\end{itemize}














% !TeX root = summary.tex

\chapter{Wiederholungsfragen}
%#######################################################################################################
%#######################################################################################################
\section{Signalverarbeitung}
\begin{enumerate}
  % SS12 Klausur: Aufgabe 4
	\item Wenn im Frequenzbereich eine Faltung durchgeführt wird, welche Operation tritt dann im Zeitbereich auf?
	% Antwort: Multiplikation
  % SS12 Klausur: Aufgabe 4
	\item Welche Eigenschaft muss ein Signal haben, damit eine Fourierreihenzerlegung für das Signal durchgeführt werden kann?
	% Antwort: Das Signal muss kontinuierlich und periodisch sein.

	% SS12 Klausur: Aufgabe 4
	\item In der Spracherkennung wird häufig das Spektrum eines kurzen Intervalls berechnet. Leider wird dadurch eine Annahme verletzt und es tritt ein Effekt auf. Was ist die Annahme und welcher Effekt tritt auf?
	% Antwort: Es wird angenommen das sich das Signal aus dem Intervall periodisch fortsetzt. Im Spektrum tretten Frequenzen auf die im Signal nicht auftreten. Dies wird als der Leck-Effekt bezeichnet.
	% SS12 Klausur: Aufgabe 5
	\item Was besagt das Abtasttheorem?
	% Antwort: Habe ein kontinuierliches Signal mit $f_{max}$ die größte auftretende Frequenz, so muss die Abtastfrequenz mindestens das doppelte von $f_{max}$ sein um das Signal zu rekonstruieren.
	\item Welches Problem entsteht wenn es nicht beachtet wird?
	% Antwort: Aliasing. Das Spektrum ist verfälscht und das Signal kann nicht rekonstruiert werden.
	\item Mit welchen zwei Methoden kann diesem Problem entgegengewirkt werden? Welche Nachteile haben diese Methoden?
	% Antwort: 1. Erhöhung Abtastfrequenz (Nachteil: Erhöhtes Daten aufkommen). 2. Tiffpassfilterung des Signals (Nachteil: Verlust von Information). 
	\item Skizzieren Sie den Unterschied zwischen einer Rechteck-Fensterfunktion und einer Fensterfunktion die den Leck-Effekt vermindert.
	% WS12 Klausur: Aufgabe 5
	%\item Was ist ein Spektrum? Was ist ein Spektogramm?
\end{enumerate}


\subsection{Fähigkeitencheck für die Klausur}
\begin{itemize}
	\item Bestimmung der Faltung grafisch % Wie in Übungsblatt 1 
	\item Bestimmung der Faltung rechnerisch % Wie in Übungsblatt 1
	\item Digitalisierung von Signalen, Abtastung, Tritt Aliasing auf? % Wie in Übungsblatt 1
	\item Filtern mit Filter (Fourietransformation) % Wie in Übungsblatt 1
	\item Berechnung von Samplingrate, Grenzfrequenz, Frequenzauflösung, Zeitauflösung für DFT % Wie in Übungsblatt 1
\end{itemize}
%#######################################################################################################
%#######################################################################################################
\section{Bildverarbeitung}
\begin{enumerate}
  % Zur Lösung der WS14 Klausur - Aufgabe 1.1
	\item Geben Sie die Formeln für die Umrechnung von RGB nach HSI an
	\begin{itemize}
		\item falls $R=G=B$, dann ist $H$ undefiniert
		\item falls $R=G=B=0$, dann ist $S$ undefiniert
		\item $c = arccos  \frac{2R - G - B}{2 \sqrt{(R-G)^2 + (R-B)(G-B)}}$
		\item $H = \left\{ \begin{array}{cl} c & \textrm{ falls } B < G \\ 360\degree - c & \textrm{ sonst} \end{array} \right.$
		\item $S = 1 - \frac{3}{R+G+B} \min (R,G,B)$
		\item $I = \frac{1}{3} (R + G + B)$
	\end{itemize}
	\item Gebe die besonderen Werte für den arccos an.
		\begin{center}
  \begin{tabular}{ l | l | l | l | l | l | l | l | l}
    \hline
    $-1$ & $-\frac{\sqrt{3}}{2}$ & $-\frac{\sqrt{2}}{2}$ & $-\frac{1}{2}$ & $0$ & $\frac{1}{2}$ & $\frac{\sqrt{2}}{2}$ & $\frac{\sqrt{3}}{2}$ & $1$ \\ \hline
    $\pi$ & $\frac{5\pi}{6}$ & $\frac{3\pi}{4}$ & $\frac{2\pi}{3}$ & $\frac{\pi}{2}$ &  $\frac{\pi}{3}$ &  $\frac{\pi}{4}$ &  $\frac{\pi}{6}$ & $0$ \\
		\hline
  \end{tabular}
\end{center}
	% Zur Lösung der WS14 Klausur - Aufgabe 2 a)
	\item Wie berechnet man das Grauwerthistogramm?
	\begin{equation}
	H(x)= \left\{ \begin{array}{cl} Anzahl x=Wert \\ 0 \textrm{ sonst} \end{array} \right.
	\end{equation}
	% Zur Lösung der WS14 Klausur - Aufgabe 2 b)
	\item Wie berechnet man beliebige Quantile?
	
	\begin{itemize}
		\item AnzahlWerte*Quantil
		\item Schauen welcher Wert ist in dieses Intervall von links nach rechts eingeschlossen
	\end{itemize}
	
	% Zur Lösung der WS14 Klausur - Aufgabe 2 c) - Wie bei Übungsblatt 2
	\item Wie berechnet man eine Histogrammdehnung? Welche Auswirkung hat eine Histogrammdehnung?
	
	\begin{itemize}
		\item $f(x)=0$ für linkes Quantil
		\item $f(x)=255$ für rechtes Quantil
		\item $f(x) = \left\{ \begin{array}{cl} 0 & \textrm{ falls } x < \textrm{linkes Quantil} \\ 255 & \textrm{falls} x>\textrm{rechtes Quantil} \\ \frac{255}{rQ-lQ}x-\frac{255 \cdot lQ}{rQ-lQ} & \textrm{ sonst }\end{array} \right.$
	\end{itemize}

  Verbesserung der Spreizung. Anstatt der minimalen bzw. maximalen Intensität werden Quantile verwendet, um tatsächliche Maxima im Histogramm zu erkennen. Ist eine affine Punktoperation.
  
	% Zur Lösung der WS14 Klausur - Aufgabe 2 d) - Wie bei Übungsblatt 2
  \item Wie berechnet man ein Histogrammausgleich? Welche Auswirkung hat ein Histogrammausgleich? \\
	Beim Histogrammausgleich werden Intensitäten erhöht die im Histogramm oft vorkommen. Dadurch werden stark vertretene Grauwerte besser sichtbar. Ist eine homogene Punktoperation aber keine affine Punktoperation. Kann in manchen Fällen auch zu einer Verminderung des Kontrast führen (siehe Übung).

  % Zur Lösung der WS14 Klausur - Aufgabe 2 e)
	\item Wie führt man Region-Growing durch? \\
	
	\begin{itemize}
		\item Wähle einen Startpunkt
		\item Initialisiere eine Liste mit diesem Punkt
		\item Wähle eine Schwelle $\epsilon$
		\item Nehme aus der Liste einen Punkt für den noch nicht geschaut wurde (also nie für eine Punkt mehr als einmal schauen)
		\item Ist der Unterschied der Grauwerte eines der direkten 4 Nachbarn kleiner als die Schwelle $\epsilon$ dann nehme diesen Punkt in die Liste auf.
		\item Mache dies solange bis die Liste nicht mehr wächst.
	\end{itemize}
	
  %SS 12 Klausur
	\item Was ist der maximale Korrelationswert, den die Zero Mean Normalized Cross Correlation (ZNCC) liefern kann?
	%WS 12 Klausur
	\item Warum kann man mit Hilfe der Epipolargeometrie das Sterokorrespondenzproblem schneller lösen?
	%Selbst ausgedacht
	\item Bennen Sie alle affinen Punktoperationen die in der Vorlesung behandelt wurden!
	\item Was versteht man unter Non-Maximum Surpression?
\end{enumerate}

\subsection{Fähigkeitencheck für die Klausur}
\begin{itemize}
	\item Eine Sprzeiung durchführen % Wie bei Übungsblatt 2
	\item Berechnung eines Histogramms, akummuliertes Histogramm % Wie bei Übungsblatt 2
	\item Beweis das Filtermatrix eine Approximation des Gauß-Filters ist % Wie bei Übungsblatt 2
	\item Anwendung des Morphologischen Schließen und Öffnen Operators % Wie bei Übungsblatt 5
	\item Anwendung der Hough-Transformation % Wie bei Übungsblatt 5
	\item Berechnung von Korrelation und Autokorrelation % Wie bei Übungsblatt 5
\end{itemize}
% Fähigkeiten
% 1. Anwendung aller besprochener Filter
% 2. Quaternionen rechnen
% 3. Rechnen mit Rotationen und Translationen
% 4. Rechnungen mit Kameramodell wie in Übungsblatt 2
%#######################################################################################################
%#######################################################################################################
\section{Klassifikation}
\begin{enumerate}
% Selbst ausgedacht
	\item Erklären Sie den Unterschied zwischen Supervised - Unsupervised
	\item Erklären Sie den Unterschied zwischen Parametrisch - Nicht-parametrisch
\end{enumerate}
\section{Spracherkennung}
% Aus SS12 Klausur - Aufgabe 4
\begin{enumerate}
	\item Welche Rolle spielt das Sprachmodell in der Automatischen Spracherkennung? Beschreiben Sie die zwei verschiedenen Ansätze, die in der Vorlesung behandelt wurden und nennen Sie je einen Vor- und einen Nachteil.
% Aus WS12 Klausur - Aufgabe 4	
	\item Benennen Sie alle wichtigen und in der Vorlesung dargestellten Komponenten eines modernen Spracherkenners (grobes Blockschaltbild). Kennzeichnen Sie auch in welcher Form die Datenströme vor und nach den jeweiligen Schritten vorliegen.
	\item Wofür werden HMMs in der Spracherkennung eingesetzt? Welche Annahme wird in der Praxis für die Länge der Markov Ketten getroffen, wenn Sprache mit HMMs modelliert wird?
	\item Stimmhafte Phoneme können in unterschiedlichen Tonhöhen produziert werden. Aus anatomischer Sicht ist dies für ein und dasselbe Phonem möglich, da die drei prinzipiellen Komponenten der Sprachproduktion unabhängig voneinander gesteuert werden können. Wie heißen diese, was ist deren Aufgabe und warum genau können Phoneme in unterschiedlichen Tonhöhen produziert werden?
	\item Was ist ein Formant? Was ist das Vokaldreieck?
\end{enumerate}
%#######################################################################################################
%#######################################################################################################
\section{Maschinelles Lernen}
%#######################################################################################################
%#######################################################################################################
\section{3D-Bildverarbeitung}
\subsection{Geometrische 3D-Transformationen}
\begin{enumerate}
% Selbst ausgedacht:
\item Welche zwei Vorgehensweisen haben sich für die geometrische 3D-Transformation durchgesetzt? \\
\begin{itemize}
	\item Homogene Geometrie
	\item Quaternionen
\end{itemize}
\item Nennen Sie drei Anforderungen an geometrische 3D-Transformationen:
\begin{itemize}
	\item Geschlossene Ausdrücke
	\item Invertierbarkeit
	\item Interpolation
\end{itemize}
\item Wie lautet die Rotationsmatrix für eine Rotation um die $x$-Achse? % Wie SS13 Klausur Aufgabe 2.2a)
\item Wie lautet die Rotationsmatrix für eine Rotation um die $y$-Achse? % WIe WS12 Klausur Aufgabe 5
\item Wie lautet die Rotationsmatrix für eine Rotation um die $z$-Achse?
\item Der Vektor $a$ soll um $180°$ um die $x$-Achse gedreht werden. Stellen Sie das entsprechende Quaternion auf. 
\item Wie lautet die Formel für die Umrechnung von Quaternion zu Rotationsmatrix?
\item Wie lautet die Formel für die Umrechnung von Rotationsmatrix zu Quaternion?
% Antwort: q=(0,(1,0,0)
\item Wie lautet die Formel zur numerischen Berechnung der SLERP zwischen $q$ und $r$?
\item Berechnen Sie SLERP(q,r,0) und SLERP(q,r,1).
% Antwort: SLERP(q,r,0)=q und SLERP(q,r,1)=1
	\item Gegeben sei das Quaternion $q_1=(s,(x,y,z))=(2,(4,-3,0))$. Berechnen Sie das multiplikativ inverse Quaternion $q^{-1}_{1}$. % Übungsblatt 6 von 2012
	% Lösung $q^{-1}_{1}=(\frac{2}{29},(-\frac{4}{29},\frac{3}{29},0))$
	\item Rotieren Sie den Punkt $\vec{x}=(0,0,3)$ mit dem Quaternion $q_2=(\frac{\sqrt{2}}{2},(\frac{\sqrt{2}}{2},0,0))$. % Übungsblatt 6 von 2012
	% Lösung $\vec{x}'=(0,-3,0)$
	\item Wie lautet die Formel zur analytischen Berechnung der SLERP zwischen $q$ und $r$, die in der Vorlesung vorgestellt wurde? % Klausur WS12 - Aufgabe 4 b)
\end{enumerate}


\subsection{Erweitertes Kameramodell}
\begin{enumerate}
% Selbst ausgedacht
	\item Wenn man das in der Vorlesung behandelte Erweiterte Kameramodell mit dem Lochkameramodell vergleicht, welche vier Vereinfachungen werden beim Lochkameramodell gemacht?
% WS12 Klausur - Aufgabe 2 1.
  \item Wie lautet die Formel für die Kalibriermatrix $K$? \\
	\begin{equation}
	K = \myvecnine{f_x}{0}{c_x}{0}{f_y}{c_y}{0}{0}{1}
	\end{equation}
% Selbst ausgedacht
	\item Wie kommt man auf die Kalibriermatrix $K$? \\
	\begin{equation}
		\myvecthree{u \cdot w}{v \cdot w}{w} = K \cdot \myvecthree{X}{Y}{Z}
  \end{equation}
% WS12 Klausur - Aufgabe 2 2.
  \item Wie kommt man von einem Punkt in Weltkoordinaten $A_w$ in die Bildkoordinaten $A_B$
	\begin{equation}
		A_B=K \cdot T \cdot W \cdot A_w
  \end{equation}
	$T$ optional je nach dem ob linke oder rechte Kamera.
% WS12 Klausur - Aufgabe 2 3.
  \item Formel für die Berechnung der Linie $L$ die in einem bestimmten Bildpunkt $B$ resultiert
	\begin{equation}
		L=\left(
		\begin{array}{c}
		X \\
		Y \\
		Z
		\end{array}
  \right)=Z\left(
		\begin{array}{c}
		\frac{u-c_x}{f_x} \\
		\frac{v-c_y}{f_y} \\
		1
		\end{array}\right)
  \end{equation}
% WS12 Klausur - Aufgabe 2 4.
  \item Wie berechnet man den Schnittpunkt $S$ zweier Linien $L_L$ und $L_R$?
	\begin{itemize}
		\item Translation in ein Koordinatensystem
		\item Gleichsetzen, Z-Wert berechnen
		\item Schnittpunt $S$ aus berechnetem Z-Wert berechnen
	\end{itemize}
% WS12 Klausur - Aufgabe 2 5.
  \item Warum kann es sein das sich zwei Linien nicht schneiden?
	\begin{itemize}
		\item Ungenaue Sterokalibrierung
		\item Linsenverzerung
		\item Pixel-Diskretisierung
		\item Ungenaue Lokalisierung der korrespondierenden Punkte
	\end{itemize}
\end{enumerate}

\subsection{Fähigkeitencheck für die Klausur}
\begin{itemize}
	\item Umgang mit Stereokameramodell % Wie bei Übungsblatt 5
	\item Umgang mit Epipolargeometrie % Wie bei Übungsblatt 5
	\item Bildpunkte ausrechnen
\end{itemize}
%#######################################################################################################
%#######################################################################################################
\section{Visuelle Wahrnehmung}
%#######################################################################################################
%#######################################################################################################
\section{Wissen und Planung}
\subsection{Wissen}
\begin{enumerate}
% Selbstausgedachte Fragen:
	\item Aus was besteht eine Logik?
	% Antowort: Symbolmenge, Belegungsmenge, Syntax, Semantik, Folgerungsoperator und elementaren Aussagen wahr und falsch
	\item Was ist eine:
	\begin{enumerate}
		\item Symbolmenge
		\item Belegungsmenge
		\item Syntax
		\item Semantik
		\item Folgerungsoperator
	\end{enumerate}

  \item Was ist eine Wissensbasis?
	\item Was ist Deduktion?
	\item Was ist ein Literal?
	\item Beschreiben Sie die Resolutionsregel
	\item Was ist eine Klausel?
	% Antwort: Eine Klausel ist die Menge der in einer Disjunktion enthaltenen Literale
	\item Wann ist ein Deduktionsalgorithmus korrekt?
	\item Wann ist ein Deduktionsalgorithmus vollständig?
	
	% Ist eine typische Frage in der Klausur
	\item Gegeben ist die folgende Klauselnmenge $WB=\left\{\left\{Q,\neg V,W\right\},\left\{\neg W,X\right\},\left\{\neg Q,W\right\},\left\{\neg X,Y\right\},\left\{V,\neg Y\right\},\left\{\neg W,\neg Y\right\}\right\}$. Zeigen Sie mit dem Resolutionsalgorithmus das $V\Leftrightarrow Y$ ableitbar ist.
	
	\item Der Resolutionsalgorithmus hat welches Problem?
	% Antwort: Erfüllbarkeitsproblem der Aussagenlogik ist NP-vollständig (d.h. vermutlich exponentieller Worst-Case-Aufwand)
	
	\item In welcher Zeit lässt sich mit den Horn-Formeln die Ableitbarkeit beantworten?
	% Antwort: Quadratische Zeit
	
	\item Wie lauten die drei Horn-Klauseln?
	% Antwort: 1. Definition (nur negierte Literale und ein positives Literal) 2. Integrationseinschränkung (nur negierte Literale) 3. Fakt/Axiom (ein positives Literal)
	\item Wozu dienen Integritätseinschränkungen?
	% Antwort: Fehler in der Wissensbasis aufzeigen
	
	% SS12 Klausur - Aufgabe 3
	\item Nennen Sie den wesentlichen Nachteil von Horn-Formel im Vergleich zu prädikatenlogischen Formeln allgemein!
	% Antwort: Horn-Formel sind nur eine Teilmenge der prädikatenlogischen Formeln. D.h. es gibt Boolesche Funktionen, die man nicht durch Hornformeln darstellen kann.
	
	% WS11 Klausur - Aufgabe 3
	\item Gegeben ist folgende Wissensbasis $K=(P \vee \neg B \vee \neg A) \wedge (\neg A \vee \neg P \vee Q) \wedge (\neg C \vee R \vee \neg Q) \wedge A \wedge B$
	
	\begin{enumerate}
		\item Erstellen Sie den Und-/Oder-Graphen
		\item Überprüfen Sie mittels Rückwärtsverkettung, ob sich Aussage R aus der Wissensbasis folgern lässt
		% Antwort: R ist nicht aus WB folgerbar
		\item Was könnte eine Frage für eine Vorwärtsverkettung sein?
	\end{enumerate}
	
	% WS14 Klausur - Aufgabe 3
	\item Prüfen Sie mithilfe des DPLL-Algorithmus, ob die prädikatenlogische Formel: $A \wedge (\neg A \vee \neg D \vee \neg E) \wedge (\neg A \vee D \vee E) \wedge (\neg D \vee E \vee F) \wedge (D \vee \neg F)$ erfüllbar ist. Geben Sie in jedem Schritt an, warum Sie eine Variable mit einem bestimmten Wert belegen.
	
	% Selbstausgedachte
	\item Durch welche Maßnahme kann ein Roboter als Punkt behandelt werden.
	
	\item Was sind die Nachteile der Potentialfeldmethode?
	% Antwort: Lokale Minimas, Keine Terminierung
	
	% WS11 Klausur - Aufgabe 3
	\item Wieso ist der Einsatz einer Heuristik bei vielen Planungsproblemen sinnvoll oder notwendig?
	\item Auch beim A*-Algorithmus wird eine Heuristik verwendet. An welcher Stelle im Algorithmus kommt eine Heuristik zum Einsatz?
	\item Welche Eigenschaften muss sie hierbei erfüllen, damit der A*-Algorithmus den kürzesten Weg zum Ziel tatsächlich finden kann?
\end{enumerate}


\addcontentsline{toc}{chapter}{Literaturverzeichnis}
%\nocite{*} %Auch nicht-zitierte BibTeX-Einträge werden angezeigt.
\bibliographystyle{alpha} %Art der Ausgabe: plain / apalike / amsalpha / ...
%\bibliography{literatur} %Eine Datei 'literatur.bib' wird hierfür benötigt.
\bibliography{bib}


%============= Register ==================================================================

\printindex

%=========================================================================================
%=========================================================================================
%=========================================================================================

\end{document}