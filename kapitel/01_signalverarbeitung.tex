% !TeX root = summary.tex

\chapter{Signalverarbeitung}

Die erste Aufgabe der Signalverarbeitung ist eine Vorverarbeitung. So muss z.B. ein analoges elektrisches Sensorsignal digitalisiert werden bevor es in einem Rechner verarbeitet werden kann. Anschließend wird durch die Signalverarbeitung die relevante Information aus dem Signal extrahiert. Oder in anderen Worten: Unwichtige Informationen sollen herausgefiltert werden. Beispiele für unwichtige Informationen:
\begin{itemize}
	\item Sprecheridentität (sprecherabhängigen Anteile)
	\item Hintergrundgeräusche, Hall, Echo, absolute Lautstärke
	\item Rotationen in Bildern
	\item Helligkeit
\end{itemize}

\subsection{Faltung\index{Faltung}}

In der Mathematik und besonders in der Funktionalanalysis beschreibt die Faltung einen mathematischen Operator, welcher für zwei Funktionen $f$ und $g$ eine dritte Funktion liefert. Diese gibt eine Art "{}Überlappung"{} zwischen $f$ und einer gespiegelten und verschobenen Version von $g$ an.
Definition: $$\begin{array}{cl} (f * g)(t) = \int\limits_{- \infty}^{+ \infty} f(t - \tau) g(\tau) d \tau & \quad \textrm{(kontinuierlich)} \\ (f * g)[i] = \sum\limits_{j = - \infty}^{\infty} f[i - j] g[j] & \quad \textrm{(diskret)} \end{array}$$
Bedeutung: \\
Eine anschauliche Deutung der Faltung ist die Gewichtung einer Funktion mit einer anderen. Der Funktionswert der Gewichtsfunktion an einer Stelle $t$ gibt an, wie stark der um $t$ zurückliegende Wert der gewichteten Funktion in den Wert der Ergebnisfunktion eingeht. \\
Faltung der Zeitfunktion: $$F(f_1(t) * f_2(t)) = F_1(\omega) \cdot F_2(\omega)$$
Multiplikation der Zeitfunktionen: $$F(f_1(t) \cdot f_2(t)) = F_1(\omega) * F_2(\omega)$$
Faltung zweier Funktionen entspricht dem Filtern eines Signals. \\[0,1cm]
Beipiel mit den folgenden zwei Funktionen:
$$f(x) = \left\{ \begin{array}{ccc} A & \textrm{ für } & x \in [-L;L] \\ 0 & \textrm{sonst} & \end{array} \right. \quad \textrm{und} \quad g(x) = \left\{ \begin{array}{ccc} A & \textrm{ für } & x \in [0;2L] \\ 0 & \textrm{sonst} & \end{array} \right.$$

% Bild der beiden Funktionen
\mypic{10}{faltung1}

Die Faltung der beiden Funktionen sieht folgendermaßen aus:

% Bild der Faltung
\mypic{5}{faltung2}

Die Faltung lässt sich wiefolgt ermitteln:
\begin{enumerate}
	\item Symbolisch: Lösung des Integrals
	\item Grafisch: Spiegelung von $f$ und über $g$ schieben. Wert der Faltung für t entspricht der Fläche der Überlappung
\end{enumerate}

\subsection{Grundlegendes}

\begin{description}
\item[Erfassen/Messen von Signalen:] \quad
\begin{itemize}
\item Signal als Funktion: 
\begin{itemize}
	\item Akustik $f(t)$
	\item Bilder $f(x,y)$
	\item Energie als Funktion der Zeit oder des Raumes
	\item Energie: Lautstärke, Helligkeit, Grauwertintensität
	\item Farbe, Stereo, Bildsequenzen
\end{itemize}
\item Eigenschaften: 
\begin{itemize}
	\item Hinlänglich glatt
	\item $0 \leq f(x,y) < \infty$ (wertbeschränkt)
\end{itemize} 
\end{itemize}
\item[Abtastung und Sampling\index{Sampling}:] \quad
\begin{itemize}
\item Man messe das Signal $f(x)$ an verschiedenen Punkten $x$
\item Punkte $x$, in diskreten Abständen, an meist äquidistanten Stellen eines Abtastrasters
\item Akustik: $G = f(0), f(\Delta), f(2 \Delta), \dots, f((N-1)\Delta)$
\item Bild: $$G = \left[ \begin{array}{ccc} f(0,0) & \cdots & f(0,N-1) \\ \vdots & \ddots & \vdots \\ f(N-1,0) & \cdots & f(N-1,N-1) \end{array} \right]$$
\item Meist: $N = 2^m$
\item Bild: Rechtwinklig, schiefwinklig, sechseckige Raster
\item Wie oft? Wie großes Raster? $\to$ Sampling Theorem
\end{itemize}
\item[Quantisierung\index{Quantisierung}:] \quad
\begin{itemize}
\item Im Computer müssen Messwerte quantisiert werden. Die Anzahl der Quantisierungsstufen bestimmt Auflösung.
\item Feinere Auflösung: Bessere Qualität, mehr Speicher
\item Dynamische Abtastung: Starke Übergänge: Fein rastern - grob quantisieren; schwache Übergänge: Grob rastern, fein quantisieren
\end{itemize}
\item[Digitalisierung von Signalen:] \quad
\begin{itemize}
\item CD, Video (DV, DVD), Digitaler Rundfunk (DAB), ISDN
\item Vorteile: Qualität (Bits sind Bits, verlustfreie Übertragung); Kompression; mehrfacher Nutzen von Kommunikationskanälen (Time Division Multiple Access)
\end{itemize}
\item[Diracfunktion\index{Diracfunktion}:]\quad
\begin{itemize}
\item Definition: $$\int\limits_{- \infty}^{+ \infty} f(x) \delta(x - x_0) dx = f(x_0) \quad \textrm{und} \quad \int\limits_{- \infty}^{+ \infty} \delta(x - x_0) dx = \int\limits_{x_0^-}^{x_0^+} \delta(x - x_0) dx = 1$$
\item Hat Fläche 1, an einer beliebig kleinen Umgebung
\item $A \delta(x - x_0)$ Impuls mit Stärke $A$ an der Stelle $x = x_0$
\end{itemize}
\end{description}

\subsection{Fouriertransformation\index{Fouriertransformation}}

\subsubsection*{Idee der Fouriertransformation}
\begin{itemize}
\item Zerlegung eines Signals in eine Summe von komplexen Sinus- und Cosinusfunktionen.
\item Unterschiedliche Frequenzen.
\item Darstellung: Amplituden und Phasen von Frequenz
\end{itemize}

\subsubsection*{Fouriertransformation}
\begin{itemize}
\item Fouriertransfromation: $$F(\omega) = \int\limits_{- \infty}^{\infty} f(t) e^{-i \omega t} dt$$
\item Inverse Fouriertransformation: $$f(t) = \frac{1}{2 \pi} \int\limits_{- \infty}^{\infty} F(\omega) e^{+i \omega t} d \omega$$
\item Betrag des Fourierspektrums: $$|F(\omega)| = \sqrt{\Re(F(\omega))^2 + \Im(F(\omega))^2}$$
\item Phase des Spektrums: $$\phi(F(\omega)) = \textrm{arctan}\frac{\Im(F(\omega))}{\Re(F(\omega))}$$
\end{itemize}

\subsubsection*{Eigenschaften}
\begin{itemize}
\item Linearität: $$F(c_1f_1 + c_2f_2) = c_1F(f_1) + c_2F(f_2)$$
\item Differentiation: $$F(f^{(n)}) = (i \omega)^n F(f)$$
\item Verschiebung: $$\begin{array}{cl} F(f(t-T)) = e^{-i \omega T} F(f) & \quad \textrm{(Zeit)} \\ F(e^{-i \omega_0 t} f(t)) = F(\omega - \omega_0) & \quad \textrm{(Frequenz)} \end{array}$$
\end{itemize}

\subsubsection*{Zusammenhänge}
\begin{center}
\begin{tabular}{rcl}
Signal & $\Rightarrow$ & Transformierte \\ Transformierte & $\Leftarrow$ & Signal \\[0,1cm] diskret & $\Leftrightarrow$ & periodisch \\ reell & $\Leftrightarrow$ & gerade \\ imaginär & $\Leftrightarrow$ & ungerade
\end{tabular}
\end{center}

\subsubsection*{Typische Fouriertransformationen}
\begin{itemize}
\item Sinusfunktion: $$f(x) = a \sin(2 \pi \alpha x) \quad \Leftrightarrow \quad F(\omega) = \frac{a}{2}i \delta(\omega + \alpha) - \frac{a}{2}i \delta(\omega - \alpha)$$
\item Cosinusfunktion: $$f(x) = a \cos(2 \pi \alpha x) \quad \Leftrightarrow \quad F(\omega) = \frac{a}{2} \delta(\omega + \alpha) + \frac{a}{2} \delta(\omega - \alpha)$$
\end{itemize}
(Sinus- und cosinusförmige Signale werden im Frequenzbereich als zwei Impulse wiedergegeben.)
\begin{itemize}
\item Impulszug: $$f(x) = \sum\limits_{\nu = -\infty}^{\infty} \delta(x - \nu T) \quad \Leftrightarrow \quad F(\omega) = \frac{1}{T} \sum\limits_{\nu = -\infty}^{\infty} \delta(\omega - \frac{\nu}{T})$$
\end{itemize}

\subsubsection*{Anmerkungen}
\begin{itemize}
\item Periodisch unendlich ausgedehntes Signal wird im Fourierbereich als endliche Bandbreite dargestellt.
\item Endliches Signal kann zu unendlichem Spektrum führen. Approximation nötig!
\item Je stärker die Übergänge im Signal, desto höher die Frequenz der Komponenten, um so breiter das Spektrum.
\end{itemize}


\subsection{Kurzzeitspektralanalyse\index{Kurzzeitspektralanalyse}}

Problem: Beide kontinuierlich, schwer digital darzustellen
\begin{itemize}
\item Diskrete Zeit (\textsl{discrete time}) Fouriertransformation: \\ Eingabe: diskret, aperiodisch \\ Ergebnis: periodisch, kontinuierlich
\item Diskrete Fouriertransformation (DFT, FFT): \\ Eingabe: periodisch, diskret \\ Ergebnis: diskret, periodisch \\ Spezialfall der Z-Transformation. \\ Periodische Eingabe? Man schneidet ein Fenster aus und tut so, als ob es so periodisch unendlich fortgesetzt wird.
\end{itemize}

Spektrum einer ganzen Aufnahme ist häufig nicht hilfreich bzw. kann die Analyse ja immer nur für ein endliches Intervall erfolgen. Problem: Frequenzauflösung vs. Zeitauflösung.

Beispiel Phoneme: Phoneme haben eine Länge von 10-100ms. Es wird angenommen, dass Sprache stationär ist in einem kurzen Zeitbereich. Es wird deshalb die Frequenzverteilung in kurzen Segmenten analysiert. Eine Herrausforderung ist es eine Abwägung zwischen zeitlicher Auflösung und der Frequenzauflösung zu finden. Es wird deshalb ein überlappender Ausschnitt aus einem Signal verwendet.

Fensterung erfolgt
\begin{itemize}
	\item Es wird Periodizität des ausgesuchten Signals angenommen. Daher DFT.
	\item Problem: Periodizität des gefensterten Signals. Periodische Fortsetzung kann unstetig sein.
\end{itemize}
Weiterer Effekt der Fensterung ist der Leck-Effekt. Ein Ausschneiden eines Stückes aus einem Signal entspricht im zeitbereich einer Multiplikation mit einer Rechteck-Funktion. Im Frequenzbereich entspricht dies einer Faltung des originalen Spektrums mit der sinc-Funktion. Die Ideale Fensterung wäre keine Fensterung. Dies entspricht im frequenzbereich einer Faltung mit einem Dirac. Eine sinc-Funktion weicht also sehr stark von einem Dirac ab. Eine Verringerung des Leck-Effekts lässt sich nur erreichen wenn das Fenster am Rand sehr viele Ableitungen hat die gegen 0 gehen. Ein solches Fenster ist das Hanning-Fenster.

\subsection{Fourierreihen\index{Fourierreihen}}

\subsubsection*{Fourierreihenzerlegung\index{Fourierreihenzerlegung}}
\begin{itemize}
\item Periodisches Signal
\item Signal: $$c_k = \frac{1}{2L} \int\limits_{-L}^{+L} f(x) e^{-i \pi \frac{k}{L} x} dx \quad \textrm{wobei} \quad f(x) = \sum\limits_{k=- \infty}^{+ \infty} c_k e^{i \pi \frac{k}{L} x}$$
\item Komplexe Schreibweise: $$e^{ix} = \cos(x) + i \sin(x)$$
\item Parameter: Amplitude, Phase, Frequenz (Periode)
\end{itemize}

\subsubsection*{Fourierreihe für Rechteckfunktion}
$$f(x) = \frac{1}{2} a_0 + \sum\limits_{n=1}^{\infty} \left( a_n \cos( \frac{2 \pi}{T} nx) + b_n \sin (\frac{2 \pi}{T} nx) \right)$$
Beispiel für Fourierreihenentwicklung: \\
Für $f(x) = \textrm{sign}(\sin(x))$ gilt:
\begin{eqnarray*}
a_n &=& \frac{1}{\pi} \int\limits_{- \pi}^{+ \pi} f(x) \cdot \cos(nx) dx = 0 \\
b_n &=& \frac{1}{\pi} \int\limits_{- \pi}^{+ \pi} f(x) \cdot \sin(nx) dx = \frac{2}{n \pi} (1 - (-1)^n)
\end{eqnarray*}
und somit $$f(x) = \frac{4}{\pi} \left( \sin(x) + \frac{\sin(3x)}{3} + \frac{\sin(5x)}{5} + \cdots \right)$$

\subsection{Aliasing\index{Aliasing}}
In der Signalverarbeitung treten Alias-Effekte beim Digitalisieren analoger Signale auf. \\
Damit das Ursprungssignal korrekt wiederhergestellt werden kann, dürfen im abgetasteten Signal nur Frequenzanteile vorkommen, die weniger als halb so groß wie die Abtastfrequenz sind. Wird dieses Abtasttheorem verletzt, werden Frequenzanteile, die größer sind als die halbe Abtastfrequenz als niedrigere Frequenzen interpretiert. Die hohen Frequenzen geben sich sozusagen als jemand anderes aus, daher die Bezeichnung Alias. Die halbe Abtastfrequenz wird als Nyquist-Frequenz bezeichnet. \\
Falls es nicht zu vermeiden ist, dass hohe Frequenzen im Eingangssignal vorhanden sind, wird das Eingangssignal zur Unterdrückung von Alias-Effekten durch einen Tiefpass gefiltert (Anti-Aliasing-Filter), wobei die aktive Filterwirkung dieses Abschneidens der hohen Frequenzen eindeutiger mit Höhensperre, Höhenfilter, High Cut und Treble Cut beschrieben wird.

\subsubsection*{Abtast/Sampling Theorem}
$$\Delta x \leq \frac{1}{2 \omega}$$
\begin{itemize}
\item Die Samplingfrequenz muss mindestens zweimal so gross sein als die höchste im Signal vorkommende Frequenz $\omega$ (\textsl{Cutoff-frequency}).
\item Um Aliasing zu vermeiden muss das Abtasttheorem eingehalten werden.
\item Ist das Abtasttheorem eingehalten, kann ein Signal durch inverse Fouriertransformation vollständig rekonstruiert werden.
\end{itemize}

\subsubsection*{Abtasten in der Praxis}
\begin{itemize}
\item Abtasten nur in endlichem Intervall möglich.
\item Sprektrum lokal (in kleinem Intervall interessant)
\begin{itemize}
\item Multiplikation des Signals mit Fensterfunktion
\item Faltung des Fenstersprektrums mit Signalspektrum
\item Ungenauigkeiten, genaue Rekonstruktion nicht mehr möglich
\end{itemize}
\item Ausnahme: Signal ist periodisch und bandbegrenzt
\end{itemize}

\subsubsection*{Behebung von Aliasing}
Möglichkeiten zur Behebung von Aliasing:
\begin{itemize}
\item Samplefrequenz erhöhen. Nachteil: erhöhtes Datenaufkommen
\item Bandbegrenzen des Originalsignals durch Tiefpassfilter (Anti-Aliasing Filter). Nachteil: Informationsverlust
\end{itemize}

\subsection{Korrelation\index{Korrelation}}
\begin{itemize}
\item Eindimensionale Kreuzkorrelation: $$R_{f,g}(m) = \sum\limits_{i} f(i) g(i - m)$$
\item Zweidimensionale Kreuzkorrelation: $$R_{f,g}(m) = \sum\limits_i \sum\limits_j f(i,j) g(i-m,j-n)$$
\item Autokorrelation = Kreuzkorrelation mit sich selbst:
\begin{eqnarray*}
A(m) &=& \sum\limits_i f(i) f(i-m) \\ A(m,n) &=& \sum\limits_i \sum\limits_j f(i,j) f(i-m,j-n)
\end{eqnarray*}
\end{itemize}

\subsection{Schablonenanpassung\index{Schablonenanpassung} (\textsl{Template Matching})}
\begin{itemize}
\item Eine einfache Form der Klassifikation.
\item Ziel: Ein Muster zu erkennen das einem abgespeicherten Beispiel ähnlich ist.
\item Maß der Übereinstimmung ist der Absolutbetrag zwischen Muster und Schablone (zentriert an $(m,n)$): $$M_{f,g}(m,n) = \sum\limits_i \sum\limits_j | f(i,j) g(i-m,j-n)|$$
\item Abstand $E(m,n)$ ist definiert als Quadrat der Kreuzkorrelation: $$E_{f,g} (m,n) = \left( \sum\limits_i \sum\limits_j f(i,j) g(i-m,j-n) \right)^2$$
\end{itemize}


\subsection{Fähigkeitencheck für die Klausur}
\begin{itemize}
	\item Bestimmung der Faltung grafisch
	\item Bestimmung der Faltung rechnerisch
	\item Digitalisierung von Signalen, Abtastung, Tritt Aliasing auf?
	\item Filtern mit Filter (Fourietransformation)
	\item Berechnung von Samplingrate, Grenzfrequenz, Frequenzauflösung, Zeitauflösung für DFT
\end{itemize}

\subsection{Fragen zum Kapitel}
\begin{enumerate}
  % SS12 Klausur: Aufgabe 4
	\item Wenn im Frequenzbereich eine Faltung durchgeführt wird, welche Operation tritt dann im Zeitbereich auf?
	% Antwort: Multiplikation
  % SS12 Klausur: Aufgabe 4
	\item Welche Eigenschaft muss ein Signal haben, damit eine Fourierreihenzerlegung für das Signal durchgeführt werden kann?
	% Antwort: Das Signal muss kontinuierlich und periodisch sein.

	% SS12 Klausur: Aufgabe 4
	\item In der Spracherkennung wird häufig das Spektrum eines kurzen Intervalls berechnet. Leider wird dadurch eine Annahme verletzt und es tritt ein Effekt auf. Was ist die Annahme und welcher Effekt tritt auf?
	% Antwort: Es wird angenommen das sich das Signal aus dem Intervall periodisch fortsetzt. Im Spektrum tretten Frequenzen auf die im Signal nicht auftreten. Dies wird als der Leck-Effekt bezeichnet.
	% SS12 Klausur: Aufgabe 5
	\item Was besagt das Abtasttheorem?
	% Antwort: Habe ein kontinuierliches Signal mit $f_{max}$ die größte auftretende Frequenz, so muss die Abtastfrequenz mindestens das doppelte von $f_{max}$ sein um das Signal zu rekonstruieren.
	\item Welches Problem entsteht wenn es nicht beachtet wird?
	% Antwort: Aliasing. Das Spektrum ist verfälscht und das Signal kann nicht rekonstruiert werden.
	\item Mit welchen zwei Methoden kann diesem Problem entgegengewirkt werden? Welche Nachteile haben diese Methoden?
	% Antwort: 1. Erhöhung Abtastfrequenz (Nachteil: Erhöhtes Daten aufkommen). 2. Tiffpassfilterung des Signals (Nachteil: Verlust von Information). 
	\item Skizzieren Sie den Unterschied zwischen einer Rechteck-Fensterfunktion und einer Fensterfunktion die den Leck-Effekt vermindert.
	% WS12 Klausur: Aufgabe 5
	%\item Was ist ein Spektrum? Was ist ein Spektogramm?
\end{enumerate}

\textit{}