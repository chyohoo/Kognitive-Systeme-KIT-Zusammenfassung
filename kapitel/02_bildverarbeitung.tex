% !TeX root = summary.tex

\newcommand{\myfbox}[1]{
\begin{tabular}{|l|}
\hline #1 \\ \hline
\end{tabular}
}

\section{Bildverarbeitung}

\subsection{Kognitionsbegriff}

\subsubsection*{Elemente des Kognitionsbegriffs\dots}

Wahrnehmung (Perzeption); Repräsentation, Modelle, Gedächtnis; Lernen, Verstehen; Denken, Problemlösen; Sprachgebrauch; Selbstreflexion; Sensomotorische Steuerung (Bewusstsein; Emotionen, Gefühle; Motivationen)

\subsubsection*{Elemente eines kognitiven Systems}

\begin{center}
\begin{tabular}{ccccc}
\fbox{Perzeption} & $\Rightarrow$ & \fbox{Kognition} & $\Rightarrow$ & \fbox{Aktion} \\
&& $\Downarrow \quad \Uparrow$ && \\
&& \fbox{Wissensdatenbank} &&
\end{tabular}
\end{center}

\begin{description}
\item[Perzeption\index{Perzeption}] Aufnahme der Umwelt durch Sensoren (Taster, Kameras, Mikrofone, \dots)
\item[Wissensdatenbank\index{Wissensdatenbank}] Repräsentation gelernter Zusammenhänge und Reaktionen darauf
\item[Kognition\index{Kognition}] Auf der Wissensdatenbank arbeitende Algorithmen: Deduktion, Induktion, Lernen
\item[Aktion\index{Aktion}] Ausführende Komponente (Textausgabe, Robotersystem, \dots)
\end{description}

\subsection{Sensorik}

\begin{description}
\item[Sensor\index{Sensor}] System, das eine physikalische Größe und deren Änderung in geeignete (in unserem Fall elektrische) Signale umwandelt.
\item[Ziel] Erfassung der Umwelt (oft in nicht fest definierten Umgebungen)
\item[Probleme] \quad
\begin{itemize}
\item Signalverarbeitung
\item Sensorik liefert nur partielle Information: Wahl der Sensorik
\item Sensorwerte sind vertauscht $\to$ statistische Verarbeitung
\item Verwendung mehrerer Sensortypen in Multisensorsystemen: Fusion der der Meßwerte
\item Modellierung: Abstraktionsstufen des Umweltmodells
\end{itemize}
\item[Sensortypen] \quad
\begin{itemize}
\item Mechanische Sensoren (z.B. Kraftmesser)
\item Temperatursensoren
\item Chemische Sensoren (z.B. H*-Ionen Sensor)
\item Akustische Sensoren
\item Magnetische Sensoren
\item Gassensoren (z.B. Lambda-Sonde)
\item u.a.
\end{itemize}
\end{description}

\subsubsection{Sensor: Struktur}

\begin{description}
\item[Elementarsensor] Aufnahme einer Messgröße und Abbildung auf Signal.
\item[Integrierter Sensor] zusätzliche Signalaufbereitung: Verstärkung, Filterung, Linearisierung, Normierung
\item[Intelligenter Sensor] integrierter Sensor mit rechnergesteuerter Auswertung. Ausgang: verarbeitete Größe
\end{description}

\subsection{Sensoren kognitiver Systeme}

\subsubsection{Sensortechnologie}

Klassifikation von Sensoren nach ihrer Funktion:
\begin{description}
\item[Interner Sensor] Erfassung innerer Zustände. Bsp.: Positionssensoren, Geschwindigkeitssensoren
\item[Externer Sensor] Information über Zustand der Umwelt. Bsp.: Taktile Sensoren, Abstandssensoren, Akustische Sensoren
\end{description}

\subsubsection{Interne Sensorik}

\begin{description}
\item[Einfacher Positionssensor: Potentiometer] Lineare Position oder Drehposition \\ Messung der abgefallenen Spannung $U$ oder des Widerstands $R$, dann $$U = ( R / R_{ges} ) \cdot U_{ges}$$ mit $R_{ges}$ als Gesamtwiderstand und $U_{ges}$ als Gesamtspannung \\ $U$ proportional zur Position $P$. $P = const \cdot U$
\item[Positionssensoren: Opt. Codierer] Information über Position von Gelenken oder über die relative Position eines Fahrzeugs. \\ Prinzip:
\begin{itemize}
\item Lichtstrahl wird auf Photodetektor ausgerichtet
\item auf einer Scheibe angebrachtes codiertes Muster unterbricht periodisch den Strahl
\item die Unterbrechungen werden gemessen und evtl. der Code festgestellt
\end{itemize}
Bauarten: inkrementelle Codierung, absolute Codierung
\item[Geschwindigkeitssensor: Tachogenerator] Gleichstromgenerator mit Ausgangsspannung $V_0$, Winkelgeschwindigkeit $\omega_s$: $$V_0 = K_t \cdot \omega_s$$
Arbeitsprinzip: elektromagnetische Induktion (Faradays Gesetz) (\textsl{Analoges Ausgangssignal benötigt A/D-Wandler zur Weiterverarbeitung})
\item[Beschleunigungsensoren: Silizium-Beschl.sensor] Messprinzip: seismische Masse \\ Mikrosystemtechnik: Ätzen einer trägen Masse und deren Aufhängungen in Silizium. \\ Auslenkung ändert mechanische Spannungen, welche den piezoresistiven Widerstandswert ändern. Widerstandswert messtechnisch erfassbar. Ursprüngliches Einsatzgebiet: Airbagauslösung im Kfz-Bereich. In der Robotik oft: Auslenkung zur Erdanziehung oder Lage im Raum.
\item[Lagesensor] \quad
\begin{itemize}
\item Messung der Orientierung zum Erdmagnetfeld und des Kipp- und Neigungswinkels.
\item Orientierung: magnetischer Kompass
\item Kipp- und Neigungswinkel: Inklinometer
\item Funktionsprinzip der Wasserwaage
\item Nachteil: träge
\item Alternativ: Gyroskop zur Messung der Winkelbeschleunigung (mechanischer Kreisel oder Faseroptisch)
\end{itemize}
\end{description}

\subsubsection{Externe Sensorik}

\begin{description}
\item[Taktile Sensoren] Umweltinformation durch direktes Berühren von Objekten (physikalischer Kontakt zwischen Roboter und Werkstück) \\ Einsatz: Ermittlung
\begin{itemize}
\item geometrischer Größen: Lage, Orientierung, Form
\item physikalischer Größen: Kräfte, Momenten, Druck
\end{itemize}
Hierzu dienen: tastende Sensoren, gleitende Sensoren, Kraft-Momenten-Sensoren \\ Beispiele:
\begin{description}
\item[künstliche Haut:] Array-Anordnungen
\item[Dehnmeßstreifen:] Dehnmeßstreifen verformt sich bei Krafteinwirkung. Veränderung des ohmschen Widerstands
\item[Drucksensoren:] Verkürzung von oder Entstehung neuer Strompfade bei Belastung
\item[Gleitsensoren:] Information über Oberflächenbeschaffenheit und geom. Struktur
\end{description}
\item[Kraft-Momenten-Sensoren] Erfassung der Kräfte und Drehmomente zwischen Effektor und Handhabungsobjekt
\item[Abstandssensoren] Messung von Abstand zwischen Sensor und Gegensatnd (Typen: optische, akustische, Radar)
\end{description}

\subsection{Beispiele}

\textbf{Sensorik bei Robotergelenken}
\begin{itemize}
\item Gelenkwinkelstellung bei Roboterarmen / Laufmaschinen-Beinen
\item Überwachung der Positionsgenauigkeit bei Greifern (Karlsruher Hand, Barrett-Hand) \\
früher: mechanische Anschläge \\ heute: Bildverarbeitung, Achsencoder
\end{itemize}

\textbf{Sensorik bei autonomen Systemen}
\begin{itemize}
\item Positionsbestimmung und Navigation in dynamischer Umgebung. Einsatz von: Laserscanner, Ultraschallsensoren, Bildverarbeitung
\item Bsp.: autonome Systeme Mortimer, Viper (IPR), HelpMate (TRC)
\end{itemize}

\textbf{Sensorik bei Laufmaschinen} \\ Sensorik zu Erfassung von: Hindernissen, Orientierung, Untergrundkontakt, Schwerpunkt







