% !TeX root = summary.tex

\section{Robotik\index{Robotik}}

\subsection{Kognitives System}

Kognition = Interpretation + Planung \\
% Bild Folie 26
\mypic{8}{kognitives_system}

Computersystem mit aktorischer Komponente: "{}Roboter"{}

\subsubsection{Probleme der Robotik}

\begin{itemize}
\item Kinematik: Bewegung, wie, wohin, möglich ?
\item Dynamik: welche Geschwindigkeit, Kraft, Momentum, geht etwas kaputt?
\item Konfigurations-, Hindernis-, Freiraum, Kollisionsvermeidung, \dots
\end{itemize}

\subsection{Mechanische Komponenten}

\subsubsection{Gelenktypen}

\begin{itemize}
\item Rotationsgelenk $R$: Drehachse in $90\degree$ zu den Achsen der angeschlossenen Glieder.
\item Torsionsgelenk $T$: Drehachse parallel zu den Achsen der angeschlossenen Glieder.
\item Revolvergelenk $V$: Eingangsglied parallel, Ausgangsglied rechtwinklig zur Drehachse.
\item Lineargelenk $L$: gleitende Bewegung entlang der Achse.
\end{itemize}

$\rightarrow$ Typ dann z.B TTL (vom Torso zum spezifischeren Gelenk)

\subsubsection{Arbeitsraum}

Anfahrbare Punkte in 3D ($\to$ mind. 3 Freiheitsgrade nötig) [Alle Gelenkwinkel berücksichtigen $\equiv$ Grundform]

\subsubsection{Radkonfigurationen}

\begin{description}
	\item[Differentialantrieb] zwei parallele Räder, einfach aber Radregelung in Echtzeit
	\item[Synchro-Drive] 3 Rad in dem alle Räder gleiche Richtung lenken, einfach, geradeaus garantiert aber mechanisch komplex
	\item[Dreirad] einfach, aber eingeschränkte Mechanik
	\item[Mecanum] 3 Rollen auf Kreisbahn, uneingeschränkt bewegbar aber komplexe Mechanik/Ansteuerung
\end{description}

\subsection{Antriebe}

\begin{itemize}
	\item Fluidisch: Linearantrieb / Schaufelrad, durch Einführen von Öl
	\item Muskelartig: Agonist/Antagonist, pneumatisch (mit Druckluft), billig und einfach aber unpräzise und laut ($\to$ Roboter mit kleiner Kraft) \underline{oder} hydraulisch (mit Öl, große Kräfte, ok Geschwindigkeit, laut, verunreinigend, langsamer (große Roboter)
	\item Elektrisch: Schritt / Servomotoren, kompakt, regelbar aber langsam, wenig Kraft (präzise Roboter) \\
\end{itemize}

\subsection{Getriebe}

Elektromotor: elektrisch -(Magnetfeld, Stromdurchflossener Leiter)-> mechanisch, Eisenkern, Polaritätswechsel mechanisch oder Wechselstrom

\subsection{Sensoren}

Messen physikalischer Größen, Elementarsensor, integrierter Sensor (= Elementar + Aufbereitung), intelligenter Sensor (++ Vorverarbeitung und Auswertung)

\begin{itemize}
	\item Intern vs Extern
	\item Aktiv (Energiezuführend, Ultraschall, Infrarot) und Passiv (kamera, Mikrofon)
\end{itemize}

\section{Robotermodellierung}

Lösen der Probleme der Robotik mittels

% Bild Folie 31
\mypic{3}{roboter_modell}

\subsubsection*{Geometrische Modellierung}

\begin{itemize}
\item dient als Grundlage zur Berechnung der Bewegung, Abstandsmessung, Kollisionserkennung, Kräfte, Momente
\item graphische Darstellung
\end{itemize}

Komplexitätsstufen des geometrischen Modells:
\begin{enumerate}
\item Bounding-Box-Modell\index{Bounding-Box-Modell}
\begin{itemize}
\item einhüllende Quader ($\to$ erste Kollisionserkennung)
\end{itemize}
\item Kantenmodell\index{Kantenmodell}
\begin{itemize}
\item Polygonzüge ($\to$ schnelle Visualisierung)
\end{itemize}
\item Volumenmodell\index{Volumenmodell}
\begin{itemize}
\item genaue Darstellung ($\to$ genaue Kollisionserkennung)
\end{itemize}
\end{enumerate}

\subsubsection*{Kinematische Modellierung}

\begin{description}
\item[Kinematisches Modell] + geometrisches Modell -> Stellung (also Gelenkwerte : Stellung Zusammenhang, Erreichbarkeit)
\item[Kinematische Kette] mehrere mit Gelenk verbundene Körper
\end{description}

Iterative Aufstellung des kinematischen Modells. Für jedes Element Objekt-Koordinatensystem

\begin{center}
\begin{tabular}{l@{\,\,\,}l}
Roboterbasis & OKS$_{Basis}$ \\
Armelement 1 & OKS$_{Arm1}$ \\
\vdots & \vdots \\
Armelement 6 & OKS$_{Arm6}$
\end{tabular}
\end{center}
Hom. Transformation verknüpfen OKS.

\subsubsection*{Denavit-Hartenberg-Transformation}

Die Transformation vom OKS des $i$-ten Armelements auf das OKS des $(i-1)$-ten Armelements erfolgt nach der \textbf{Denavit-Hartenberg (DH)-Konvention}\index{Denavit-Hartenberg-Konvention}:
\begin{enumerate}
\item[(a)] die Koordinatensysteme liegen in den Bewegungsachsen
\item[(b)] die $z_{i-1}$-Achse liegt entlang der Drehachse des $i$-ten Armelements
\item[(c)] die $x_i$-Achse steht senkrecht zur $z_{i-1}$-Achse und zeigt von ihr weg
\item[(d)] die $y_i$-Achse bildet rechtshändiges Koordinatensystem
\end{enumerate}
($i \in \{ Basis, 1 , \dots , n \}$) \\[0,1cm]

In Matrizenschreibweise lautet die Denavit-Hartenberg-Transformation (Bewegung) von $i-1$ auf $i$:
\begin{eqnarray*}
A_{i-1,i} &=& R_{z_{i-1}}(\theta_i) \cdot T_{z_{i-1}}(s_i) \cdot T_{x_i}(a_i) \cdot R_{x_i}(\alpha_i) \\
&=& \left[ \begin{array}{cccc} \cos(\theta_i) & - \sin(\theta_i) \cdot \cos(\alpha_i) & \sin(\theta_i) \cdot \sin(\alpha_i) & a_i \cdot \cos(\theta_i) \\ \sin(\theta_i) & \cos(\theta_i) \cdot \cos(\alpha_i) & - \cos(\theta_i) \cdot \sin(\alpha_i) & a_i \cdot \sin(\theta_i) \\ 0 & \sin(\alpha_i) & \cos(\alpha_i) & s_i \\ 0 & 0 & 0 & 1 \end{array} \right]
\end{eqnarray*}
$\theta_i$ und $s_i$ variabel, abhängig von Bewegung und spezieller Geometrie. \\ $\alpha_i$ und $a_i$ dagegen sind invariante Größen, nur einmal berechnen.

\subsection{Fall: Planung von Greifoperationen}

\begin{itemize}
	\item Griff kategorisierbar: Anwendungsgebiet, Objektform, $\dots$
\end{itemize}
$\rightarrow$ Cutkosky-Hierarchie, 16 einzelne Griffarten, Hierarchiebaum (Kraft/Präzision als Root)

\begin{description}
	\item[Greifarten] Greifen/Loslassen, An/Abrücken ohne/mit Objekt, Verbinden, Transferbewegung
	\item[Griffgenerierung] Analyse $\to$ interne/externe Nebendbed. $\to$ Synthese $\to$ Griff
	\item[interne/externe Nebenbed] Kollisionsfreiheit, Zugänglichkeit / Kollisionsfreie An/Abrückbewegung, Griffmöglichkeit, Stabilität von allem
\end{description}

\subsubsection{Problem of Dimensionality}

Menschliche Hand + Pose = 27 Dimensionen, + Stabilität etc. als analytisches Problem kaum lösbar $\rightarrow$ Simulation

\subsubsection{Greifsimulation}

Hand + Objektmodell laden, Griffkandidat generieren, Griff ausführen und anhand Gütekriterien für Stabilität auswerten

\paragraph{Griffkandidatenerzeugung}

\begin{description}
	\item[Formprimitiven] Approximation durch Formprimitiven, nach bestimmten Regeln Griffkandidat generieren (nur manuelle Formprimitiven möglich
	\item[Superquadriken] Zerlegung in Baum aus Superquadriken, auf denen Griffkandidat berechnen (automatisch möglich)
	\item[Box Decomposition] Min Vol Bounding Boxes zur Objektdarstellung, Griffkandidaten auf einzelnen Boxen generieren (automatisch, auch auf unbekannten Objekten, sehr grob
	\item[Medial-Axis-Transformation] Kugeln maximaler Größe reinschreiben, mittels Kugelmittelpunkte Griffkandidaten (sehr genau, Radien $\equiv$ lokale Dichte des Objekts)
\end{description}

\paragraph{Kraftschluss}

Wie gut widersteht ein Griff externen Einwirkungen ? \\

\begin{itemize}
	\item Bestimme Kontaktpunkte und berechne Reibungskegel
	\item Berechne Grasp Wrench Space GWS als konvexe Hülle
	\item min(Ursprung GWS zu einer Facette GWS) ist Maß für Stabilität
\end{itemize}

\subsection{Interesting}

\begin{description}
	\item[Planungsmodell] Nodes $\equiv$ subgoals, Edges $\equiv$ Transition $\rightarrow$ Consistent representation of state
	\item[Dynamic Interactive Learning] Lernsystem auf Roboter, Lernen interaktive statt mit zweitem Rechner. Lernen durch Vormachen, Modellierung zusammen mit Rechner, Umwandeln der Infos vom Roboter selbst
	\item[Constraint Representation] position constraints from learning or knowing, distance, orientation $\dots$
	\item[Structural Bootstrapping] deduce from existing data, filter additional information
	\item[Neurorobotics] biologydriven approach, improve medicine, neuroscience, computing
\end{description}