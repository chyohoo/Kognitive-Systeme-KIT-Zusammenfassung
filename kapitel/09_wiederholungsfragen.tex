% !TeX root = summary.tex

\chapter{Wiederholungsfragen}
%#######################################################################################################
%#######################################################################################################
\section{Signalverarbeitung}
\begin{enumerate}
  % SS12 Klausur: Aufgabe 4
	\item Wenn im Frequenzbereich eine Faltung durchgeführt wird, welche Operation tritt dann im Zeitbereich auf?
	% Antwort: Multiplikation
  % SS12 Klausur: Aufgabe 4
	\item Welche Eigenschaft muss ein Signal haben, damit eine Fourierreihenzerlegung für das Signal durchgeführt werden kann?
	% Antwort: Das Signal muss kontinuierlich und periodisch sein.

	% SS12 Klausur: Aufgabe 4
	\item In der Spracherkennung wird häufig das Spektrum eines kurzen Intervalls berechnet. Leider wird dadurch eine Annahme verletzt und es tritt ein Effekt auf. Was ist die Annahme und welcher Effekt tritt auf?
	% Antwort: Es wird angenommen das sich das Signal aus dem Intervall periodisch fortsetzt. Im Spektrum tretten Frequenzen auf die im Signal nicht auftreten. Dies wird als der Leck-Effekt bezeichnet.
	% SS12 Klausur: Aufgabe 5
	\item Was besagt das Abtasttheorem?
	% Antwort: Habe ein kontinuierliches Signal mit $f_{max}$ die größte auftretende Frequenz, so muss die Abtastfrequenz mindestens das doppelte von $f_{max}$ sein um das Signal zu rekonstruieren.
	\item Welches Problem entsteht wenn es nicht beachtet wird?
	% Antwort: Aliasing. Das Spektrum ist verfälscht und das Signal kann nicht rekonstruiert werden.
	\item Mit welchen zwei Methoden kann diesem Problem entgegengewirkt werden? Welche Nachteile haben diese Methoden?
	% Antwort: 1. Erhöhung Abtastfrequenz (Nachteil: Erhöhtes Daten aufkommen). 2. Tiffpassfilterung des Signals (Nachteil: Verlust von Information). 
	\item Skizzieren Sie den Unterschied zwischen einer Rechteck-Fensterfunktion und einer Fensterfunktion die den Leck-Effekt vermindert.
	% WS12 Klausur: Aufgabe 5
	%\item Was ist ein Spektrum? Was ist ein Spektogramm?
\end{enumerate}


\subsection{Fähigkeitencheck für die Klausur}
\begin{itemize}
	\item Bestimmung der Faltung grafisch % Wie in Übungsblatt 1 
	\item Bestimmung der Faltung rechnerisch % Wie in Übungsblatt 1
	\item Digitalisierung von Signalen, Abtastung, Tritt Aliasing auf? % Wie in Übungsblatt 1
	\item Filtern mit Filter (Fourietransformation) % Wie in Übungsblatt 1
	\item Berechnung von Samplingrate, Grenzfrequenz, Frequenzauflösung, Zeitauflösung für DFT % Wie in Übungsblatt 1
\end{itemize}
%#######################################################################################################
%#######################################################################################################
\section{Bildverarbeitung}
\begin{enumerate}
  % Zur Lösung der WS14 Klausur - Aufgabe 1.1
	\item Geben Sie die Formeln für die Umrechnung von RGB nach HSI an
	\begin{itemize}
		\item falls $R=G=B$, dann ist $H$ undefiniert
		\item falls $R=G=B=0$, dann ist $S$ undefiniert
		\item $c = arccos  \frac{2R - G - B}{2 \sqrt{(R-G)^2 + (R-B)(G-B)}}$
		\item $H = \left\{ \begin{array}{cl} c & \textrm{ falls } B < G \\ 360\degree - c & \textrm{ sonst} \end{array} \right.$
		\item $S = 1 - \frac{3}{R+G+B} \min (R,G,B)$
		\item $I = \frac{1}{3} (R + G + B)$
	\end{itemize}
	\item Gebe die besonderen Werte für den arccos an.
		\begin{center}
  \begin{tabular}{ l | l | l | l | l | l | l | l | l}
    \hline
    $-1$ & $-\frac{\sqrt{3}}{2}$ & $-\frac{\sqrt{2}}{2}$ & $-\frac{1}{2}$ & $0$ & $\frac{1}{2}$ & $\frac{\sqrt{2}}{2}$ & $\frac{\sqrt{3}}{2}$ & $1$ \\ \hline
    $\pi$ & $\frac{5\pi}{6}$ & $\frac{3\pi}{4}$ & $\frac{2\pi}{3}$ & $\frac{\pi}{2}$ &  $\frac{\pi}{3}$ &  $\frac{\pi}{4}$ &  $\frac{\pi}{6}$ & $0$ \\
		\hline
  \end{tabular}
\end{center}
	% Zur Lösung der WS14 Klausur - Aufgabe 2 a)
	\item Wie berechnet man das Grauwerthistogramm?
	\begin{equation}
	H(x)= \left\{ \begin{array}{cl} Anzahl x=Wert \\ 0 \textrm{ sonst} \end{array} \right.
	\end{equation}
	% Zur Lösung der WS14 Klausur - Aufgabe 2 b)
	\item Wie berechnet man beliebige Quantile?
	
	\begin{itemize}
		\item AnzahlWerte*Quantil
		\item Schauen welcher Wert ist in dieses Intervall von links nach rechts eingeschlossen
	\end{itemize}
	
	% Zur Lösung der WS14 Klausur - Aufgabe 2 c) - Wie bei Übungsblatt 2
	\item Wie berechnet man eine Histogrammdehnung? Welche Auswirkung hat eine Histogrammdehnung?
	
	\begin{itemize}
		\item $f(x)=0$ für linkes Quantil
		\item $f(x)=255$ für rechtes Quantil
		\item $f(x) = \left\{ \begin{array}{cl} 0 & \textrm{ falls } x < \textrm{linkes Quantil} \\ 255 & \textrm{falls} x>\textrm{rechtes Quantil} \\ \frac{255}{rQ-lQ}x-\frac{255 \cdot lQ}{rQ-lQ} & \textrm{ sonst }\end{array} \right.$
	\end{itemize}

  Verbesserung der Spreizung. Anstatt der minimalen bzw. maximalen Intensität werden Quantile verwendet, um tatsächliche Maxima im Histogramm zu erkennen. Ist eine affine Punktoperation.
  
	% Zur Lösung der WS14 Klausur - Aufgabe 2 d) - Wie bei Übungsblatt 2
  \item Wie berechnet man ein Histogrammausgleich? Welche Auswirkung hat ein Histogrammausgleich? \\
	Beim Histogrammausgleich werden Intensitäten erhöht die im Histogramm oft vorkommen. Dadurch werden stark vertretene Grauwerte besser sichtbar. Ist eine homogene Punktoperation aber keine affine Punktoperation. Kann in manchen Fällen auch zu einer Verminderung des Kontrast führen (siehe Übung).

  % Zur Lösung der WS14 Klausur - Aufgabe 2 e)
	\item Wie führt man Region-Growing durch? \\
	
	\begin{itemize}
		\item Wähle einen Startpunkt
		\item Initialisiere eine Liste mit diesem Punkt
		\item Wähle eine Schwelle $\epsilon$
		\item Nehme aus der Liste einen Punkt für den noch nicht geschaut wurde (also nie für eine Punkt mehr als einmal schauen)
		\item Ist der Unterschied der Grauwerte eines der direkten 4 Nachbarn kleiner als die Schwelle $\epsilon$ dann nehme diesen Punkt in die Liste auf.
		\item Mache dies solange bis die Liste nicht mehr wächst.
	\end{itemize}
	
  %SS 12 Klausur
	\item Was ist der maximale Korrelationswert, den die Zero Mean Normalized Cross Correlation (ZNCC) liefern kann?
	%WS 12 Klausur
	\item Warum kann man mit Hilfe der Epipolargeometrie das Sterokorrespondenzproblem schneller lösen?
	%Selbst ausgedacht
	\item Bennen Sie alle affinen Punktoperationen die in der Vorlesung behandelt wurden!
	\item Was versteht man unter Non-Maximum Surpression?
\end{enumerate}

\subsection{Fähigkeitencheck für die Klausur}
\begin{itemize}
	\item Eine Sprzeiung durchführen % Wie bei Übungsblatt 2
	\item Berechnung eines Histogramms, akummuliertes Histogramm % Wie bei Übungsblatt 2
	\item Beweis das Filtermatrix eine Approximation des Gauß-Filters ist % Wie bei Übungsblatt 2
	\item Anwendung des Morphologischen Schließen und Öffnen Operators % Wie bei Übungsblatt 5
	\item Anwendung der Hough-Transformation % Wie bei Übungsblatt 5
	\item Berechnung von Korrelation und Autokorrelation % Wie bei Übungsblatt 5
\end{itemize}
% Fähigkeiten
% 1. Anwendung aller besprochener Filter
% 2. Quaternionen rechnen
% 3. Rechnen mit Rotationen und Translationen
% 4. Rechnungen mit Kameramodell wie in Übungsblatt 2
%#######################################################################################################
%#######################################################################################################
\section{Klassifikation}
\begin{enumerate}
% Selbst ausgedacht
	\item Erklären Sie den Unterschied zwischen Supervised - Unsupervised
	\item Erklären Sie den Unterschied zwischen Parametrisch - Nicht-parametrisch
\end{enumerate}
\section{Spracherkennung}
% Aus SS12 Klausur - Aufgabe 4
\begin{enumerate}
	\item Welche Rolle spielt das Sprachmodell in der Automatischen Spracherkennung? Beschreiben Sie die zwei verschiedenen Ansätze, die in der Vorlesung behandelt wurden und nennen Sie je einen Vor- und einen Nachteil.
% Aus WS12 Klausur - Aufgabe 4	
	\item Benennen Sie alle wichtigen und in der Vorlesung dargestellten Komponenten eines modernen Spracherkenners (grobes Blockschaltbild). Kennzeichnen Sie auch in welcher Form die Datenströme vor und nach den jeweiligen Schritten vorliegen.
	\item Wofür werden HMMs in der Spracherkennung eingesetzt? Welche Annahme wird in der Praxis für die Länge der Markov Ketten getroffen, wenn Sprache mit HMMs modelliert wird?
	\item Stimmhafte Phoneme können in unterschiedlichen Tonhöhen produziert werden. Aus anatomischer Sicht ist dies für ein und dasselbe Phonem möglich, da die drei prinzipiellen Komponenten der Sprachproduktion unabhängig voneinander gesteuert werden können. Wie heißen diese, was ist deren Aufgabe und warum genau können Phoneme in unterschiedlichen Tonhöhen produziert werden?
	\item Was ist ein Formant? Was ist das Vokaldreieck?
\end{enumerate}
%#######################################################################################################
%#######################################################################################################
\section{Maschinelles Lernen}
%#######################################################################################################
%#######################################################################################################
\section{3D-Bildverarbeitung}
\subsection{Geometrische 3D-Transformationen}
\begin{enumerate}
% Selbst ausgedacht:
\item Welche zwei Vorgehensweisen haben sich für die geometrische 3D-Transformation durchgesetzt? \\
\begin{itemize}
	\item Homogene Geometrie
	\item Quaternionen
\end{itemize}
\item Nennen Sie drei Anforderungen an geometrische 3D-Transformationen:
\begin{itemize}
	\item Geschlossene Ausdrücke
	\item Invertierbarkeit
	\item Interpolation
\end{itemize}
\item Wie lautet die Rotationsmatrix für eine Rotation um die $x$-Achse? % Wie SS13 Klausur Aufgabe 2.2a)
\item Wie lautet die Rotationsmatrix für eine Rotation um die $y$-Achse? % WIe WS12 Klausur Aufgabe 5
\item Wie lautet die Rotationsmatrix für eine Rotation um die $z$-Achse?
\item Der Vektor $a$ soll um $180°$ um die $x$-Achse gedreht werden. Stellen Sie das entsprechende Quaternion auf. 
\item Wie lautet die Formel für die Umrechnung von Quaternion zu Rotationsmatrix?
\item Wie lautet die Formel für die Umrechnung von Rotationsmatrix zu Quaternion?
% Antwort: q=(0,(1,0,0)
\item Wie lautet die Formel zur numerischen Berechnung der SLERP zwischen $q$ und $r$?
\item Berechnen Sie SLERP(q,r,0) und SLERP(q,r,1).
% Antwort: SLERP(q,r,0)=q und SLERP(q,r,1)=1
	\item Gegeben sei das Quaternion $q_1=(s,(x,y,z))=(2,(4,-3,0))$. Berechnen Sie das multiplikativ inverse Quaternion $q^{-1}_{1}$. % Übungsblatt 6 von 2012
	% Lösung $q^{-1}_{1}=(\frac{2}{29},(-\frac{4}{29},\frac{3}{29},0))$
	\item Rotieren Sie den Punkt $\vec{x}=(0,0,3)$ mit dem Quaternion $q_2=(\frac{\sqrt{2}}{2},(\frac{\sqrt{2}}{2},0,0))$. % Übungsblatt 6 von 2012
	% Lösung $\vec{x}'=(0,-3,0)$
	\item Wie lautet die Formel zur analytischen Berechnung der SLERP zwischen $q$ und $r$, die in der Vorlesung vorgestellt wurde? % Klausur WS12 - Aufgabe 4 b)
\end{enumerate}


\subsection{Erweitertes Kameramodell}
\begin{enumerate}
% Selbst ausgedacht
	\item Wenn man das in der Vorlesung behandelte Erweiterte Kameramodell mit dem Lochkameramodell vergleicht, welche vier Vereinfachungen werden beim Lochkameramodell gemacht?
% WS12 Klausur - Aufgabe 2 1.
  \item Wie lautet die Formel für die Kalibriermatrix $K$? \\
	\begin{equation}
	K = \myvecnine{f_x}{0}{c_x}{0}{f_y}{c_y}{0}{0}{1}
	\end{equation}
% Selbst ausgedacht
	\item Wie kommt man auf die Kalibriermatrix $K$? \\
	\begin{equation}
		\myvecthree{u \cdot w}{v \cdot w}{w} = K \cdot \myvecthree{X}{Y}{Z}
  \end{equation}
% WS12 Klausur - Aufgabe 2 2.
  \item Wie kommt man von einem Punkt in Weltkoordinaten $A_w$ in die Bildkoordinaten $A_B$
	\begin{equation}
		A_B=K \cdot T \cdot W \cdot A_w
  \end{equation}
	$T$ optional je nach dem ob linke oder rechte Kamera.
% WS12 Klausur - Aufgabe 2 3.
  \item Formel für die Berechnung der Linie $L$ die in einem bestimmten Bildpunkt $B$ resultiert
	\begin{equation}
		L=\left(
		\begin{array}{c}
		X \\
		Y \\
		Z
		\end{array}
  \right)=Z\left(
		\begin{array}{c}
		\frac{u-c_x}{f_x} \\
		\frac{v-c_y}{f_y} \\
		1
		\end{array}\right)
  \end{equation}
% WS12 Klausur - Aufgabe 2 4.
  \item Wie berechnet man den Schnittpunkt $S$ zweier Linien $L_L$ und $L_R$?
	\begin{itemize}
		\item Translation in ein Koordinatensystem
		\item Gleichsetzen, Z-Wert berechnen
		\item Schnittpunt $S$ aus berechnetem Z-Wert berechnen
	\end{itemize}
% WS12 Klausur - Aufgabe 2 5.
  \item Warum kann es sein das sich zwei Linien nicht schneiden?
	\begin{itemize}
		\item Ungenaue Sterokalibrierung
		\item Linsenverzerung
		\item Pixel-Diskretisierung
		\item Ungenaue Lokalisierung der korrespondierenden Punkte
	\end{itemize}
\end{enumerate}

\subsection{Fähigkeitencheck für die Klausur}
\begin{itemize}
	\item Umgang mit Stereokameramodell % Wie bei Übungsblatt 5
	\item Umgang mit Epipolargeometrie % Wie bei Übungsblatt 5
	\item Bildpunkte ausrechnen
\end{itemize}
%#######################################################################################################
%#######################################################################################################
\section{Visuelle Wahrnehmung}
%#######################################################################################################
%#######################################################################################################
\section{Wissen und Planung}
\subsection{Wissen}
\begin{enumerate}
% Selbstausgedachte Fragen:
	\item Aus was besteht eine Logik?
	% Antowort: Symbolmenge, Belegungsmenge, Syntax, Semantik, Folgerungsoperator und elementaren Aussagen wahr und falsch
	\item Was ist eine:
	\begin{enumerate}
		\item Symbolmenge
		\item Belegungsmenge
		\item Syntax
		\item Semantik
		\item Folgerungsoperator
	\end{enumerate}

  \item Was ist eine Wissensbasis?
	\item Was ist Deduktion?
	\item Was ist ein Literal?
	\item Beschreiben Sie die Resolutionsregel
	\item Was ist eine Klausel?
	% Antwort: Eine Klausel ist die Menge der in einer Disjunktion enthaltenen Literale
	\item Wann ist ein Deduktionsalgorithmus korrekt?
	\item Wann ist ein Deduktionsalgorithmus vollständig?
	
	% Ist eine typische Frage in der Klausur
	\item Gegeben ist die folgende Klauselnmenge $WB=\left\{\left\{Q,\neg V,W\right\},\left\{\neg W,X\right\},\left\{\neg Q,W\right\},\left\{\neg X,Y\right\},\left\{V,\neg Y\right\},\left\{\neg W,\neg Y\right\}\right\}$. Zeigen Sie mit dem Resolutionsalgorithmus das $V\Leftrightarrow Y$ ableitbar ist.
	
	\item Der Resolutionsalgorithmus hat welches Problem?
	% Antwort: Erfüllbarkeitsproblem der Aussagenlogik ist NP-vollständig (d.h. vermutlich exponentieller Worst-Case-Aufwand)
	
	\item In welcher Zeit lässt sich mit den Horn-Formeln die Ableitbarkeit beantworten?
	% Antwort: Quadratische Zeit
	
	\item Wie lauten die drei Horn-Klauseln?
	% Antwort: 1. Definition (nur negierte Literale und ein positives Literal) 2. Integrationseinschränkung (nur negierte Literale) 3. Fakt/Axiom (ein positives Literal)
	\item Wozu dienen Integritätseinschränkungen?
	% Antwort: Fehler in der Wissensbasis aufzeigen
	
	% SS12 Klausur - Aufgabe 3
	\item Nennen Sie den wesentlichen Nachteil von Horn-Formel im Vergleich zu prädikatenlogischen Formeln allgemein!
	% Antwort: Horn-Formel sind nur eine Teilmenge der prädikatenlogischen Formeln. D.h. es gibt Boolesche Funktionen, die man nicht durch Hornformeln darstellen kann.
	
	% WS11 Klausur - Aufgabe 3
	\item Gegeben ist folgende Wissensbasis $K=(P \vee \neg B \vee \neg A) \wedge (\neg A \vee \neg P \vee Q) \wedge (\neg C \vee R \vee \neg Q) \wedge A \wedge B$
	
	\begin{enumerate}
		\item Erstellen Sie den Und-/Oder-Graphen
		\item Überprüfen Sie mittels Rückwärtsverkettung, ob sich Aussage R aus der Wissensbasis folgern lässt
		% Antwort: R ist nicht aus WB folgerbar
		\item Was könnte eine Frage für eine Vorwärtsverkettung sein?
	\end{enumerate}
	
	% WS14 Klausur - Aufgabe 3
	\item Prüfen Sie mithilfe des DPLL-Algorithmus, ob die prädikatenlogische Formel: $A \wedge (\neg A \vee \neg D \vee \neg E) \wedge (\neg A \vee D \vee E) \wedge (\neg D \vee E \vee F) \wedge (D \vee \neg F)$ erfüllbar ist. Geben Sie in jedem Schritt an, warum Sie eine Variable mit einem bestimmten Wert belegen.
	
	% Selbstausgedachte
	\item Durch welche Maßnahme kann ein Roboter als Punkt behandelt werden.
	
	\item Was sind die Nachteile der Potentialfeldmethode?
	% Antwort: Lokale Minimas, Keine Terminierung
	
	% WS11 Klausur - Aufgabe 3
	\item Wieso ist der Einsatz einer Heuristik bei vielen Planungsproblemen sinnvoll oder notwendig?
	\item Auch beim A*-Algorithmus wird eine Heuristik verwendet. An welcher Stelle im Algorithmus kommt eine Heuristik zum Einsatz?
	\item Welche Eigenschaften muss sie hierbei erfüllen, damit der A*-Algorithmus den kürzesten Weg zum Ziel tatsächlich finden kann?
\end{enumerate}
